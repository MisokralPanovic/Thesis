\chapter{Discussion and Conclusion}
\section{Discussion}

qpcr data suggest ifits are activated the most in non infected cells dusing infection ... it seems that they interact with IBs ... is the OE and KO results from the paper because of protection prior to infection or because of intreactions with IBs? ... conclude that probs both


lets assume i3 is non reeactive with (p)IBs (or minimaly reactvie) - baseline to what is ifit based interaction

compare the observed phenotypes in a range of IB sizes (small, mid, large) and what interaction with differet ifits are there

\section{Final Remarks and Future Directions}
experiments that we wanted to do but they did not work or we did not have time to do them

ifit inducible cell lines
ifit sh cell lines+

minigenome

stain for inert proteins or gfp/rfp and see if they associate with ib and pibs

measure ibs from non transfected cells (INFECTION TRANSFECTION)


pibs + i2 + i3

% Words in text: 0
% total words: 90 + 509 + 0 + 6100 + 4100 + 5740 + 4260 + 8660 + 0 + 0 = 29,459