\chapter{Final Discussion and Future Directions}
% 350
Human and bovine RSV stand as the predominant causes of lower respiratory tract infections, posing significant health risks to young calves, children under 5, the elderly, and immunocompromised individuals \cite{Falsey2005RespiratoryAdults, Coultas2019RespiratoryAge}. RSV, an enveloped, single-stranded negative-sense RNA virus, is host-restricted \textit{in vivo} \cite{Buchholz2000ChimericVaccine}. In infected cells, RSV forms membrane-less perinuclear cytoplasmic inclusion bodies (IBs), recognized as sites for viral RNA transcription and replication \cite{Rincheval2017FunctionalVirus, Jobe2020RespiratorySignaling, Jobe2023ViralCondensates}. These IBs also manipulate cellular components, either repurposing them for viral benefit (e.g., components of the eIF4F complex) \cite{Jobe2023ViralCondensates} or inhibiting their function (e.g., MAVS and MDA5) \cite{Lifland2012HumanMAVS}. To date there is a limited information oavailable about host interferon stimulated genes that restrict RSV infection. Notable expetions exist such as the IFITM protein family \cite{Smith2019Interferon-InducedMembrane}, IFI44 \cite{McDonald2016ADisease, Li2021IdentificationVirus}, and most importantly for this project the IFIT protein family \cite{Drori2020InfluenzaProteins}. IFIT proteins are known to exert their antiviral action through mechanisms well described in human systems, involving the potentiation of innate immune signalling cascades (IFIT1, IFIT3, and IFIT5) \cite{Li2009ISG56Response, Reynaud2015IFIT1Interferon, Liu2011IFN-InducedTBK1, Zhang2013IFIT5Pathways}, promotion of apoptosis (IFIT2) \cite{Chen2017InhibitionApoptosis, Diamond2013TheProteins}, inhibition of cell cycle progression (IFIT3) \cite{Xiao2006RIG-GProteins}, and detection of non-self single-stranded RNA (IFIT1 and IFIT5) \cite{Abbas2013StructuralProteins, Pichlmair2011IFIT1RNA, Diamond2014IFIT1:Translation, Mears2018BetterResponse}. Currenlty it is unknow which of these functions contribute to the restriction of RSV replication. It is possible that the restriction is not mediated via these functions but to a novel, currently undiscovered mechanism. All of these were shown to impede viral replication, although the precise mechanisms of action are not know. Better understanding of how host facors impede RSV replication can lead to potential new anti RSV therapies. Alongn to this, comparing the restriction factors of different species such as in human and bovine cells can potentially lead us to better understant the RSV host restriction that is observed \textit{in vivo}. In this project we aimed to assess if the IFITs are induced during RSV infection and thus their reported ectopic inhibition of RSV is relevant \textit{in vivo}. Additionally, we sought to unravel the nature of this inhibition, specifically by examining IFITs' interaction with RSV IBs. Lastly, we aimed to understand if this induction and subsequent inhibition are consistent between species by assessing bovine IFIT interaction with bovine RSV, as information on this aspect is lacking in the literature.

% 576
In Chapters \ref{ch:Assessment of Transcriptional Induction of Human IFITs in the Context of RSV} and \ref{ch:Assessment of Transcriptional Induction of Bovine IFITs in the Context of RSV} we have investigated the transcriptional induction of human and bovine IFITs respectively to the activators of the innate immune system, such as IFN\(\alpha\) and LPS, and human and bovine RSV infection. Treatment with IFN\(\alpha\) resulted in induction of all \textit{IFITs} across both the two human cellines and two bovine cell lines that we have tested, however, only A549 cell line indicated induction with magnitude over 30 fold, with the remaining cell lines showing modest induction. BEAS2B cell line has been described in the literature to maintain higher basal expression of interferon-stimulated genes \cite{Seng2014HighResistance}, and the responses from this cell line seem similar to what was observed in bovine cell lines MDBK and BT. This could suggest that MDBK and BT cell lines do as well have higher basal expression of interferon-stimulated genes and thus would explain our observations. With regards to other stimulants, we have observed LPS not being a huge factor in human or bovine \textit{IFIT} induction; IFN\(\gamma\) induces human \textit{IFITs} to the same magnitude and this stimulation is either concentration non-specific or the response was maximal at the lowest concentration tested; and intracellular detection of non-self necleic acids (transfected poly I:C) apperas to be the best inducer these. With regards to the \textit{IFIT} transcriptional responses to RSV infection, we have observed drastic differences between human and bovine cells. We have observed human \textit{IFITs} to be induced by hRSV infection, with this induction magnitude being concentration and time dependant. Along to this, we have also observed that the presence of low amount of replicating hRSV was not sufficient for \textit{hIFIT} induction, suggesting that certain detection threshold needs to be surpassed for \textit{IFIT} transcription to initiate. We elucidated that the human \textit{IFIT} induction by hRSV infection requires functional interferon receptor-mediated signalling, along with active hRSV replication, which was observed in both A549 adn BEAS2B cell lines. Therefore we hypothesised that the human \textit{IFITs} are predominantly induced in uninfected cells by interferon signalling cascades and therefore act prophylactically to protect those cells and limit RSV spread. With regards to bovine \textit{IFIT} induction to bRSV infection, we have failed to oberve any biologically significant induction of these, although the bRSV was sucessfuly replicating in both MDBK and BT cell lines. We have tried a plethora of bRSV concentrations and mutants, such as $\Delta$SH and $\Delta$NSs which as they are lacking the key viral proteins that negatively regulate the cellular immune response to the infection should result in higher antiviral response, but in all of these we still failed to consistently observe biologically significant induction of bovine \textit{IFITs}. Further we investigated the species cross protection and cross activation by infecting the human cells with bRSV and assesing the human \textit{IFIT} induction and vice versa. We have observed both WT and mutant bRSV to significantly induce \textit{hIFITs} in the A549 cell line, while we observed a modest induction in the BEAS2B cell line. On the other hand, we have failed to observe \textit{bIFIT} induction by hRSV infection in both MDBK and BT cell lines. Interestingly, both viruses were able to replicate efficiently in the cell of the other species, as confirmed by the RSV \textit{N} gene quantification. This suggest that the species restriction of RSV that is reported from the literature \textit{in vivo} is not recapitulated in our experiments \textit{in vitro}. This suggest that the species restriction arises from humoral and cell mediated immune reponses, rather than via the diffrences of intracellular antiviral proteins between the species.

% 504
The above mentioned absence of bovine \textit{IFIT} activation could be due to a plethora of factors such as higher basal expression of bovine \textit{IFITs}, which is also hinted at by the decreased sensitivity to interferon treatment; or preferential inhibition of \textit{bIFIT} activation pahtways by both human and bovine RSV, although this restriction is not seen in human samples, although we do not know which is the real reason. The former question could be elucidated by quantifying the total mRNA transcript of \textit{bIFITs} in mock and infected cells and comparing these to the quantification of \textit{hIFIT} mRNA transcript from human cell lines, especially A549. Along these lines, in Chapter \ref{ch:Subcellular Localisation of Endogenous IFIT Proteins in the Context of RSV Inclusion Bodies} we have initialy investigated the differences of subcellular localisation of human and bovine IFIT proteins in mock, IFN$\alpha$, and RSV treated samples. Firstly, we were able to observe both human and bovine IFITs, which behaved in a similar manner. This suggest that the increased basal expression of bovine \textit{IFITs} is correct, as in the absence of this we would expect not to detect a signal. Along to this we have observed a discrepancy between the previously established fold changes to IFN$\alpha$ stimulation and RSV infection, and the observed staining intensities. Based on the results from Chapters \ref{ch:Assessment of Transcriptional Induction of Human IFITs in the Context of RSV} and \ref{ch:Assessment of Transcriptional Induction of Bovine IFITs in the Context of RSV}, we would expect human IFIT staining intensities increase between control and treated samples, while bovine IFIT staining remaining constant between the conditions. We however only observed the signal of bovine IFIT1 to sequentially increase between the mock, bIFN$\alpha$ treated, and bRSV infected samples; while we have seen human IFIT1 and IFIT5 to maintain the staining intensity between mock and hIFN$\alpha$ treated samples, but display increased staining in hRSV infected samples; and have observed human IFIT3 and bovine IFIT5 to display increase in signal between mock and IFN$\alpha$ treated samples, which is then maintained during infection. It is especially intreeuging to see human and bovine IFIT5 to so clearly increase in their signal as these two were consistently the worst responding \textit{IFITs} in previous chapters. We have tried to quantified these differences in detected fluorescences using the CellProfiler software \cite{McQuin2018CellProfilerBiology}, where we would segment the acquired images for cellular obsects and thus would obtain the mean flourescences per cellular object in an automated, unbiased manner. We have however encountered an issue where there was a large variability of the basal aquired signal intensity, which was most evident using the nuclear stain. Due to this reason this analysis was ommited from this thesis. Regardless, the preliminary analysis did not show us increases in intensity that would be correlative with that was observed with regards to the relative transcript abundance. This could be due to bad practices during data acquisition using the confocal microscope, or it can have a biological culprint. It has been described in the literature that in general protein abundance should scale with mRNA abundance, however many additional factors such as differential translation, protein degradation, contextual confounders and pervasive protein-level buffering would lead to actual quantities of mRNA and protein to not correlate \cite{Liu2016OnAbundance, Buccitelli2020MRNAsControl}, which could explain the differences we observed.

%%% another 700
% both chapters together again
Therefore clearly both human and bovine IFIT proteins are present during RSV infection and thus can in theory enact their anti-RSV activity, which nature is currently unknown. We have tried to start to discect this by investigating the changes in subcellular localisation of the IFIT proteins.

% if 1 and 5 (cos similar)

%if 3 and 2

% ifit2 story
n-term tagged i2 detected by i2a and i2b

% investigate the ifit heteromultimer importance
sh cell lines or crispr cell lines to look at ifit multimer complex importance
pibs + i2 + i3

confocal stuff issues

super res micro with confocal

% if stable cell lines for infection/trasnfection
we tried this; issue with i2f apoptotic factor; i3 stopping cell cycle; i1 stopping translation
better to do inducible cell line

% proper controls if
stain for inert proteins or gfp/rfp and see if they associate with ib and pibs
measure ibs from non transfected cells (INFECTION TRANSFECTION)

%%% another 700
% final model of what is happening
ifits gets activated in yet uninfected cells. after infection of these i1 i2 i5 are recriuted to IBs, and this recruitment is inhibited if their concentration is not sufficient. possibly 3 gets recruited if sufficient amount of 1 and 2 is present in these structures. initialy these ifits are within the ib and as they mature and produce ibags they get pushed out within the ib edge, where the rsv genome is located. this all culminates in possibly reducing the maximal size and normal viral physiology of the IBs.

% what causes recrlitment
\cite{Oliveira2013HumanCells} - IP of n p and m2/1 doesnt show ifits as interactors
IFITI IP and Mass spec

% do intra IB ifits alter rsv replication?
minigenome \cite{Teng2016UseTranscription}


% final statement
asdasd


% Words in text: 0
% total words: 90 + 509 + 800? + 7,652 + 4,067 + 7,280 + 5,713 + 11,616 + 19,697 + 2,000-2,500? = 56,624
% -------------------------------------------------------------------------------------------
% random notes that will be useful later
%IBs containing visible IBAGs were significantly larger than those without 6.4 um2 versus 2.3 um2 (Rincheval et al., 2017; hep2)
%Although small IBs (-+2.5 um2) were observed as early as 6 h p.i., they did not colocalize with detectable levels of p65 (FJ thesis)
%Our pseudo-IBs were also mostly spherical and at 24 h post-transfection, measured up to 6.9 um2 , which is considerably less than that of the conventional IBs observed in infected cells. (FJ thesis)
%Visible aggregates measured above 3 um2 with the average size increasing as infection progressed (Fig 4.5B). At 48 h p.i., p65 aggregations had a mean area of 22.18 um2 but smaller puncta (<10 um2 ) were also observed, most likely the result of nascent infections in nearby cells. (FJ thesis)
%Their mean area also increased to 8.99 um2 by 24 h p.i. (FJ thesis)

%6h only N positive, mean 1; 16h n positive mean 1, n + p65 10; 24h n pos mean 1, n + p65 two clusters -> 2.5, 17 (FJ thesis)


