\chapter{Final Discussion and Future Directions}
%discussion
%%% another 700
Human and bovine RSV stand as the predominant causes of lower respiratory tract infections, posing significant health risks to young calves, children under 5, the elderly, and immunocompromised individuals \cite{Falsey2005RespiratoryAdults, Coultas2019RespiratoryAge}. RSV, an enveloped, single-stranded negative-sense RNA virus, is host-restricted \textit{in vivo} \cite{Buchholz2000ChimericVaccine}. In infected cells, RSV forms membrane-less perinuclear cytoplasmic inclusion bodies (IBs), recognized as sites for viral RNA transcription and replication \cite{Rincheval2017FunctionalVirus, Jobe2020RespiratorySignaling, Jobe2023ViralCondensates}. These IBs also manipulate cellular components, either repurposing them for viral benefit (e.g., components of the eIF4F complex) \cite{Jobe2023ViralCondensates} or inhibiting their function (e.g., MAVS and MDA5) \cite{Lifland2012HumanMAVS}. To date there is a limited information oavailable about host interferon stimulated genes that restrict RSV infection. Notable expetions exist such as the IFITM protein family \cite{Smith2019Interferon-InducedMembrane}, IFI44 \cite{McDonald2016ADisease, Li2021IdentificationVirus}, and most importantly for this project the IFIT protein family \cite{Drori2020InfluenzaProteins}. IFIT proteins are known to exert their antiviral action through mechanisms well described in human systems, involving the potentiation of innate immune signalling cascades (IFIT1, IFIT3, and IFIT5) \cite{Li2009ISG56Response, Reynaud2015IFIT1Interferon, Liu2011IFN-InducedTBK1, Zhang2013IFIT5Pathways}, promotion of apoptosis (IFIT2) \cite{Chen2017InhibitionApoptosis, Diamond2013TheProteins}, inhibition of cell cycle progression (IFIT3) \cite{Xiao2006RIG-GProteins}, and detection of non-self single-stranded RNA (IFIT1 and IFIT5) \cite{Abbas2013StructuralProteins, Pichlmair2011IFIT1RNA, Diamond2014IFIT1:Translation, Mears2018BetterResponse}. Currenlty it is unknow which of these functions contribute to the restriction of RSV replication. It is possible that the restriction is not mediated via these functions but to a novel, currently undiscovered mechanism. All of these were shown to impede viral replication, although the precise mechanisms of action are not know. Better understanding of how host facors impede RSV replication can lead to potential new anti RSV therapies. Alongn to this, comparing the restriction factors of different species such as in human and bovine cells can potentially lead us to better understant the RSV host restriction that is observed \textit{in vivo}. In this project we aimed to assess if the IFITs are induced during RSV infection and thus their reported ectopic inhibition of RSV is relevant \textit{in vivo}. Additionally, we sought to unravel the nature of this inhibition, specifically by examining IFITs' interaction with RSV IBs. Lastly, we aimed to understand if this induction and subsequent inhibition are consistent between species by assessing bovine IFIT interaction with bovine RSV, as information on this aspect is lacking in the literature.

%%% another 700
% combined qpcr data
In Chapters \ref{ch:Assessment of Transcriptional Induction of Human IFITs in the Context of RSV} and \ref{ch:Assessment of Transcriptional Induction of Bovine IFITs in the Context of RSV} we have investigated the transcriptional induction of human and bovine IFITs respectively to the activators of the innate immune system, such as IFN\(\alpha\) and LPS, and human and bovine RSV infection. Treatment with IFN\(\alpha\) resulted in induction of all \textit{IFITs} across both the two human cellines and two bovine cell lines that we have tested, however, only A549 cell line indicated induction with magnitude over 30 fold, with the remaining cell lines showing modest induction. BEAS2B cell line has been described in the literature to maintain higher basal expression of interferon-stimulated genes \cite{Seng2014HighResistance}, and the responses from this cell line seem similar to what was observed in bovine cell lines MDBk and BT. This could suggest that MDBK and BT cell lines do as well have higher basal expression of interferon-stimulated genes and thus would explain our observations. With regards to other stimulants, we have observed LPS not being a huge factor in human or bovine \textit{IFIT} induction; IFN\(\gamma\) induces human \textit{IFITs} to the same magnitude and this stimulation is either concentration non-specific or the response was maximal at the lowest concentration tested; and intracellular detection of non-self necleic acids (transfected poly I:C) apperas to be the best inducer these. 



human cell lines roxo theory

bovine cells do not seem to get induced by neither virus


qpcr data suggest ifits are activated the most in non infected cells dusing infection

% fit qpcr with initial if and speculate bovine stuff
in human we probs did not see huge differences between mock, ifn, and infected
we tried to quantify it using cell profiler \cite{McQuin2018CellProfilerBiology}
problems with differential basal intensity beween images with the same gain
but prelimiary results do not say 200 fold

mRNA does not always relte to protein \cite{Liu2016OnAbundance, Buccitelli2020MRNAsControl}

% for qpcr story - future steps
quantify and compare the absolute amount of copies per unit of measurement between human and bovine cells (in at least a few cell lines) to see if bIFIT and BEAS2B hIFITs are basaly more expressed

%%% another 700
% aim 2
how do they do the negative action?

% if 1 and 5 (cos similar)
asdf

%if 3 and 2
asdf

%%% another 700
% if compare the observed phenotypes in a range of IB sizes (small, mid, large) and what interaction with differet ifits are there
asdf

% final model of what is happening
asdf


... it seems that they interact with IBs ... is the OE and KO results from the paper because of protection prior to infection or because of intreactions with IBs? ... conclude that probs both


% future directions

%%% another 700
% if stable cell lines for infection/trasnfection
we tried this; issue with i2f apoptotic factor; i3 stopping cell cycle; i1 stopping translation
better to do inducible cell line

% ifit2 story
n-term tagged i2 detected by i2a and i2b

% investigate the ifit heteromultimer importance
sh cell lines or crispr cell lines to look at ifit multimer complex importance
pibs + i2 + i3

%%% another 700
% proper controls if
stain for inert proteins or gfp/rfp and see if they associate with ib and pibs
measure ibs from non transfected cells (INFECTION TRANSFECTION)

% what causes recrlitment
\cite{Oliveira2013HumanCells} - IP of n p and m2/1 doesnt show ifits as interactors
IFITI IP and Mass spec

% do intra IB ifits alter rsv replication?
minigenome

% final statement
asdasd

% Words in text: 0
% total words: 90 + 509 + 800? + 7,652 + 4,067 + 7,280 + 5,713 + 11,616 + 19,697 + 2,000-2,500? = 56,624
% -------------------------------------------------------------------------------------------
% random notes that will be useful later
%IBs containing visible IBAGs were significantly larger than those without 6.4 um2 versus 2.3 um2 (Rincheval et al., 2017; hep2)
%Although small IBs (-+2.5 um2) were observed as early as 6 h p.i., they did not colocalize with detectable levels of p65 (FJ thesis)
%Our pseudo-IBs were also mostly spherical and at 24 h post-transfection, measured up to 6.9 um2 , which is considerably less than that of the conventional IBs observed in infected cells. (FJ thesis)
%Visible aggregates measured above 3 um2 with the average size increasing as infection progressed (Fig 4.5B). At 48 h p.i., p65 aggregations had a mean area of 22.18 um2 but smaller puncta (<10 um2 ) were also observed, most likely the result of nascent infections in nearby cells. (FJ thesis)
%Their mean area also increased to 8.99 um2 by 24 h p.i. (FJ thesis)

%6h only N positive, mean 1; 16h n positive mean 1, n + p65 10; 24h n pos mean 1, n + p65 two clusters -> 2.5, 17 (FJ thesis)


