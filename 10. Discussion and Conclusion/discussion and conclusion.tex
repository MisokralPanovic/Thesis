\chapter{Final Remarks and Future Directions}
%Discussion

%reacp intro ifits
intro ifit

%recap intro rsv
intro rsv

%recap aims
recapitulate aims

% combined qpcr data
qpcr data suggest ifits are activated the most in non infected cells dusing infection

% fit qpcr with initial if and speculate bovine stuff
asdf

% if 1 and 5 (cos similar)
asdf

%if 3 and 2
asdf

% if compare the observed phenotypes in a range of IB sizes (small, mid, large) and what interaction with differet ifits are there
asdf

% final model of what is happening
asdf


... it seems that they interact with IBs ... is the OE and KO results from the paper because of protection prior to infection or because of intreactions with IBs? ... conclude that probs both


%Future Directions

% for qpcr story
quantify and compare the absolute amount of copies per unit of measurement between human and bovine cells (in at least a few cell lines) to see if bIFIT and BEAS2B hIFITs are basaly more expressed

% if stable cell lines for infection/trasnfection
we tried this; issue with i2f apoptotic factor; i3 stopping cell cycle; i1 stopping translation
better to do inducible cell line

% ifit2 story
n-term tagged i2 detected by i2a and i2b

% investigate the ifit heteromultimer importance
sh cell lines or crispr cell lines to look at ifit multimer complex importance
pibs + i2 + i3

% proper controls if
stain for inert proteins or gfp/rfp and see if they associate with ib and pibs
measure ibs from non transfected cells (INFECTION TRANSFECTION)

% what causes recrlitment
\cite{Oliveira2013HumanCells} - IP of n p and m2/1 doesnt show ifits as interactors
IFITI IP and Mass spec

% do intra IB ifits alter rsv replication?
minigenome

% Words in text: 0
% total words: 90 + 509 + 0 + 7,652 + 4,067 + 7,280 + 5,713 + 11,616 + 19,697 + 0 = 56,624

% random notes that will be useful later
IBs containing visible IBAGs were significantly larger than those without 6.4 um2 versus 2.3 um2 (Rincheval et al., 2017; hep2)
Although small IBs (-+2.5 um2) were observed as early as 6 h p.i., they did not colocalize with detectable levels of p65 (FJ thesis)
Our pseudo-IBs were also mostly spherical and at 24 h post-transfection, measured up to 6.9 um2 , which is considerably less than that of the conventional IBs observed in infected cells. (FJ thesis)
Visible aggregates measured above 3 um2 with the average size increasing as infection progressed (Fig 4.5B). At 48 h p.i., p65 aggregations had a mean area of 22.18 um2 but smaller puncta (<10 um2 ) were also observed, most likely the result of nascent infections in nearby cells. (FJ thesis)
Their mean area also increased to 8.99 um2 by 24 h p.i. (FJ thesis)

6h only N positive, mean 1; 16h n positive mean 1, n + p65 10; 24h n pos mean 1, n + p65 two clusters -> 2.5, 17 (FJ thesis)