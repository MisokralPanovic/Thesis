\chapter{Final Discussion and Future Directions}
% 316
Human and bovine RSV stand as the predominant causes of lower respiratory tract infections, posing significant health risks to young calves, children under 5, the elderly, and immunocompromised individuals \cite{Falsey2005RespiratoryAdults, Coultas2019RespiratoryAge}. RSV, characterised as an enveloped, single-stranded negative-sense RNA virus, exhibits host restriction \textit{in vivo} \cite{Buchholz2000ChimericVaccine}. Within infected cells, RSV forms membrane-less perinuclear cytoplasmic inclusion bodies, recognised as hubs for viral RNA transcription and replication \cite{Rincheval2017FunctionalVirus, Jobe2020RespiratorySignaling, Jobe2023ViralCondensates}. These IBs also manipulate cellular components, either repurposing them for viral benefit (e.g., components of the eIF4F complex) \cite{Jobe2023ViralCondensates} or inhibiting their function (e.g., MAVS and MDA5) \cite{Lifland2012HumanMAVS}. Despite limited information on host interferon-stimulated genes restricting RSV infection, exceptions include the IFITM protein family \cite{Smith2019Interferon-InducedMembrane}, IFI44 \cite{McDonald2016ADisease, Li2021IdentificationVirus}, and notably, the IFIT protein family \cite{Drori2020InfluenzaProteins}. IFIT proteins exert their antiviral effects through a multitude of mechanisms, involving potentiation of innate immune signalling cascades (IFIT1, IFIT3, and IFIT5) \cite{Li2009ISG56Response, Reynaud2015IFIT1Interferon, Liu2011IFN-InducedTBK1, Zhang2013IFIT5Pathways}, promotion of apoptosis (IFIT2) \cite{Chen2017InhibitionApoptosis, Diamond2013TheProteins}, inhibition of cell cycle progression (IFIT3) \cite{Xiao2006RIG-GProteins}, and detection of non-self single-stranded RNA (IFIT1 and IFIT5) \cite{Abbas2013StructuralProteins, Pichlmair2011IFIT1RNA, Diamond2014IFIT1:Translation, Mears2018BetterResponse}. Currently, it is unknown which of these functions contribute to the restriction of RSV replication. It is possible that the restriction is not mediated via these functions but through a novel, currently undiscovered mechanism. All of these were shown to impede viral replication, although the precise mechanisms of action are not known. A better comprehension of how host factors impede RSV replication may lead to potential anti-RSV therapies. Furthermore, comparing restriction factors across species, such as in human and bovine cells, could enhance understanding of RSV host restriction \textit{in vivo}. This project aimed to assess if IFITs are induced during RSV infection, exploring the relevance of their reported ectopic inhibition of RSV \textit{in vivo}. Additionally, we sought to uncover the nature of this inhibition, specifically by examining IFITs' interaction with RSV IBs. Lastly, we aimed to understand if this induction and subsequent inhibition are consistent between species by assessing bovine IFIT interaction with bovine RSV, as information on this aspect is lacking in the literature.

% 408
In Chapters \ref{ch:Assessment of Transcriptional Induction of Human IFITs in the Context of RSV} and \ref{ch:Assessment of Transcriptional Induction of Bovine IFITs in the Context of RSV}, we investigated the transcriptional induction of human and bovine IFITs in response to activators of the innate immune system, including IFN\(\alpha\) and LPS, as well as human and bovine RSV infection. Treatment with IFN\(\alpha\) resulted in the induction of all \textit{IFITs} across the two human cell lines and two bovine cell lines tested. Notably, the A549 cell line exhibited an induction magnitude exceeding 30-fold, while other cell lines showed more modest induction. The BEAS2B cell line, known for maintaining higher basal expression of interferon-stimulated genes \cite{Seng2014HighResistance}, displayed responses similar to bovine cell lines MDBK and BT, suggesting elevated basal expression in the latter. Regarding other stimulants, LPS did not significantly induce human or bovine \textit{IFITs}. IFN\(\gamma\) induced human \textit{IFITs} to a similar magnitude, with either concentration non-specificity or maximal response observed at the lowest concentration tested. The intracellular detection of non-self nucleic acids (transfected poly I:C) appeared to be the most potent inducer.

Examining \textit{IFIT} transcriptional responses to RSV infection revealed substantial differences between human and bovine cells. Human \textit{IFITs} were induced by hRSV infection in a concentration- and time-dependent manner. Interestingly, a low amount of replicating hRSV was insufficient for \textit{hIFIT} induction, indicating a detection threshold requirement for \textit{IFIT} transcription initiation. Human \textit{IFIT} induction by hRSV infection was found to depend on functional interferon receptor-mediated signalling and active hRSV replication, observed in both A549 and BEAS2B cell lines. Consequently, we hypothesised that human \textit{IFITs} are predominantly induced in uninfected cells through interferon signalling cascades, acting prophylactically to protect cells and limit RSV spread. In contrast, bovine \textit{IFIT} induction in response to bRSV infection failed to exhibit biologically significant results, despite successful bRSV replication in MDBK and BT cell lines. Multiple bRSV concentrations and mutants (e.g., $\Delta$SH and $\Delta$NSs) were tested, but no consistent induction of bovine \textit{IFITs} was observed. Cross-species infection experiments, infecting human cells with bRSV and vice versa, revealed significant induction of \textit{hIFITs} by both WT and mutant bRSV in A549 cells, with modest induction in BEAS2B cells. However, \textit{bIFIT} induction by hRSV infection in MDBK and BT cell lines was not observed. Intriguingly, both viruses efficiently replicated in cells of the other species, as confirmed by RSV \textit{N} gene quantification. This suggests that the reported species restriction of RSV \textit{in vivo} is not recapitulated in our \textit{in vitro} experiments. Consequently, this implies that species restriction arises from humoral and cell-mediated immune responses rather than differences in intracellular antiviral proteins between species.

% 462
The absence of bovine \textit{IFIT} activation could be attributed to various factors, such as higher basal expression of bovine \textit{IFITs}, as indicated by decreased sensitivity to interferon treatment. Another possibility is the preferential inhibition of \textit{bIFIT} activation pathways by both human and bovine RSV, although this restriction is not observed in human samples. Thus the exact reason remains unclear. To address the question of higher basal expression, quantifying the total mRNA transcript levels of \textit{bIFITs} in mock and infected cells and comparing them with the quantification of \textit{hIFIT} mRNA transcripts, especially in human cell lines like A549, could provide insights. In Chapter \ref{ch:Subcellular Localisation of Endogenous IFIT Proteins in the Context of RSV Inclusion Bodies} we have initially investigated the differences of subcellular localisation of human and bovine IFIT proteins in mock, IFN$\alpha$, and RSV treated samples. Firstly, we were able to observe both human and bovine IFITs, which behaved in a similar manner. This suggests that the increased basal expression of bovine \textit{IFITs} is correct, as in the absence of this we would expect not to detect any signal. Along to this we have observed a discrepancy between the previously established fold changes to IFN$\alpha$ stimulation and RSV infection, and the observed staining intensities. Based on the results from Chapters \ref{ch:Assessment of Transcriptional Induction of Human IFITs in the Context of RSV} and \ref{ch:Assessment of Transcriptional Induction of Bovine IFITs in the Context of RSV}, we would expect human IFIT staining intensities increase between control and treated samples, while bovine IFIT staining remaining constant between the conditions. We however only observed the signal of bovine IFIT1 to sequentially increase between the mock, bIFN$\alpha$ treated, and bRSV infected samples; while we have seen human IFIT1 and IFIT5 to maintain the staining intensity between mock and hIFN$\alpha$ treated samples, but display increased staining in hRSV infected samples; and have observed human IFIT3 and bovine IFIT5 to display increase in signal between mock and IFN$\alpha$ treated samples, which is then maintained during infection. Notably, human and bovine IFIT5, consistently the least responsive in previous chapters, exhibited clear increases in signal. 

Efforts were made to quantify these differences in detected fluorescence using CellProfiler software \cite{McQuin2018CellProfilerBiology}. The aim was to segment acquired images for cellular objects, enabling the automated and unbiased determination of mean fluorescence per cellular object. However, challenges arose, particularly in the variability of the basal acquired signal intensity,  which was most evident with the nuclear stain. Consequently, this analysis was omitted from this thesis. Despite this, the preliminary analysis did not reveal increases in intensity correlating with the observed differences in relative transcript abundance. Potential explanations include suboptimal practices during data acquisition using the confocal microscope or underlying biological factors. Literature suggests that, while protein abundance is generally expected to scale with mRNA abundance, various additional factors such as differential translation, protein degradation, contextual confounders, and pervasive protein-level buffering may lead to a lack of correlation between mRNA and protein quantities \cite{Liu2016OnAbundance, Buccitelli2020MRNAsControl}. These considerations could elucidate the disparities observed in this study.

%%% another 700
% both chapters together again
Therefore clearly both human and bovine IFIT proteins are present during RSV infection and thus can in theory enact their anti-RSV activity, which nature is currently unknown. We have tried to start to discect this by investigating the changes in subcellular localisation of the IFIT proteins, through which we have noticed that certain IFIT proteins seem to colocalise with the inclusion bodies in RSV infected cells. To better understand these occurances we have thoroughly assayed the IFIT interaction phenotypes with human and bvoine RSV IBs, as detected in A549, BEAS2B, and MDBK cell lines, along with assessing if the IFIT prteins have differential propensity of interaction with IBs based on their size, and thus maturity. Overall, we observed diverse interaction phenotypes, which encompassed strong interactions, manifested by inclusion within the IB structure and colocalisation with the IB edge, as well as weak interactions, including diffusion and edge exclusion phenotypes. Conversely, the exclusion phenotype indicated that IFITs lack access to these structures.


% if 1 and 5 (cos similar)

%if 3 and 2

% ifit2 story
n-term tagged i2 detected by i2a and i2b

% investigate the ifit heteromultimer importance
sh cell lines or crispr cell lines to look at ifit multimer complex importance
pibs + i2 + i3

confocal stuff issues

super res micro with confocal

% if stable cell lines for infection/trasnfection
we tried this; issue with i2f apoptotic factor; i3 stopping cell cycle; i1 stopping translation
better to do inducible cell line

% proper controls if
stain for inert proteins or gfp/rfp and see if they associate with ib and pibs
measure ibs from non transfected cells (INFECTION TRANSFECTION)

%%% another 700
% final model of what is happening
ifits gets activated in yet uninfected cells. after infection of these i1 i2 i5 are recriuted to IBs, and this recruitment is inhibited if their concentration is not sufficient. possibly 3 gets recruited if sufficient amount of 1 and 2 is present in these structures. initialy these ifits are within the ib and as they mature and produce ibags they get pushed out within the ib edge, where the rsv genome is located. this all culminates in possibly reducing the maximal size and normal viral physiology of the IBs.

% what causes recrlitment
\cite{Oliveira2013HumanCells} - IP of n p and m2/1 doesnt show ifits as interactors
IFITI IP and Mass spec

% do intra IB ifits alter rsv replication?
minigenome \cite{Teng2016UseTranscription}


% final statement
asdasd


% Words in text: 0
% total words: 90 + 509 + 800? + 7,652 + 4,067 + 7,280 + 5,713 + 11,616 + 19,697 + 2,000-2,500? = 56,624
% total figures: 8 + 5 + 13 + 13 + 34 + 60 = 133
% -------------------------------------------------------------------------------------------
% random notes that will be useful later
%IBs containing visible IBAGs were significantly larger than those without 6.4 um2 versus 2.3 um2 (Rincheval et al., 2017; hep2)
%Although small IBs (-+2.5 um2) were observed as early as 6 h p.i., they did not colocalize with detectable levels of p65 (FJ thesis)
%Our pseudo-IBs were also mostly spherical and at 24 h post-transfection, measured up to 6.9 um2 , which is considerably less than that of the conventional IBs observed in infected cells. (FJ thesis)
%Visible aggregates measured above 3 um2 with the average size increasing as infection progressed (Fig 4.5B). At 48 h p.i., p65 aggregations had a mean area of 22.18 um2 but smaller puncta (<10 um2 ) were also observed, most likely the result of nascent infections in nearby cells. (FJ thesis)
%Their mean area also increased to 8.99 um2 by 24 h p.i. (FJ thesis)

%6h only N positive, mean 1; 16h n positive mean 1, n + p65 10; 24h n pos mean 1, n + p65 two clusters -> 2.5, 17 (FJ thesis)


