

In Chapter \ref{ch:Subcellular Localisation of Endogenous IFIT Proteins in the Context of RSV Inclusion Bodies} we have initially investigated the differences of subcellular localisation of human and bovine IFIT proteins in mock, IFN$\alpha$, and RSV treated samples. Firstly, we were able to observe both human and bovine IFITs, which behaved in a similar manner. This suggests that the increased basal expression of bovine \textit{IFITs} is correct, as in the absence of this we would expect not to detect any signal. Along to this we have observed a discrepancy between the previously established fold changes to IFN$\alpha$ stimulation and RSV infection, and the observed staining intensities. Based on the results from Chapters \ref{ch:Assessment of Transcriptional Induction of Human IFITs in the Context of RSV} and \ref{ch:Assessment of Transcriptional Induction of Bovine IFITs in the Context of RSV}, we would expect human IFIT staining intensities increase between control and treated samples, while bovine IFIT staining remaining constant between the conditions. We however only observed the signal of bovine IFIT1 to sequentially increase between the mock, bIFN$\alpha$ treated, and bRSV infected samples; while we have seen human IFIT1 and IFIT5 to maintain the staining intensity between mock and hIFN$\alpha$ treated samples, but display increased staining in hRSV infected samples; and have observed human IFIT3 and bovine IFIT5 to display increase in signal between mock and IFN$\alpha$ treated samples, which is then maintained during infection. Notably, human and bovine IFIT5, consistently the least responsive in previous chapters, exhibited clear increases in signal. 

Efforts were made to quantify these differences in detected fluorescence using CellProfiler software \cite{McQuin2018CellProfilerBiology}. The aim was to segment acquired images for cellular objects, enabling the automated and unbiased determination of mean fluorescence per cellular object. However, challenges arose, particularly in the variability of the basal acquired signal intensity, which was most evident with the nuclear stain. Consequently, this analysis was omitted from this thesis. Despite this, the preliminary analysis did not reveal increases in intensity correlating with the observed differences in relative transcript abundance. Potential explanations include suboptimal practices during data acquisition using the confocal microscope or underlying biological factors. Literature suggests that, while protein abundance is generally expected to scale with mRNA abundance, various additional factors such as differential translation, protein degradation, contextual confounders, and pervasive protein-level buffering may lead to a lack of correlation between mRNA and protein quantities \cite{Liu2016OnAbundance, Buccitelli2020MRNAsControl}. These considerations could elucidate the disparities observed in this study.

% 500
Therefore clearly both human and bovine IFIT proteins are present during RSV infection and thus can in theory enact their anti-RSV activity, which nature is currently unknown. We have tried to start to discect this by investigating the changes in subcellular localisation of the IFIT proteins, through which we have noticed that certain IFIT proteins seem to colocalise with the inclusion bodies in RSV infected cells. To better understand these occurances we have thoroughly assayed the IFIT interaction phenotypes with human and bvoine RSV IBs, as detected in A549, BEAS2B, and MDBK cell lines, along with assessing if the IFIT prteins have differential propensity of interaction with IBs based on their size, and thus maturity. Overall, we observed diverse interaction phenotypes, which encompassed strong interactions, manifested by inclusion within the IB structure and colocalisation with the IB edge, as well as weak interactions, including diffusion and edge exclusion phenotypes. Conversely, the exclusion phenotype indicated that IFITs lack access to these structures, which was the most commonly observed phenotype in all IFITs. In Chapter \ref{ch:Investigating the Nature of IFIT and IB Interactions} we have further investigated the nature of IFIT and IB interaction by observing the interaction phenotypes of IFITs with RSV pseudo incusion bodies, which in this context act as simpliefied models of RSV IBs. We have also investigated the maximal propensity of interaction of IFIT proteins with RSV incusion bodies by overexpressing them in RSV infected cells. All of these allowed us to gain initial understanding and to hypothetise initial models of action if IFIT protein interaction with RSV inclusion bodies. 

We have observed strong interaction phenotypes to occur for endogenous human and bovine IFIT1 and IFIT5 during human and bovine RSV infection respectively, although they were more coomonly observed with IFIT1. However we have observed both to display more strong interaction phenotypes with regards to both the ineraction with pseudo inclusion bodies, as well as during exogenous expression of these proteins during human and bovine RSV infection.




However, if a sufficient concentration of IFIT1 is present in the cells, as is the case with exogenously expressed IFIT1, this overcomes the inhibition, resulting in IFIT1 mainly exhibiting strong interaction phenotypes with RSV inclusion bodies.

This suggests that factors within the RSV hinder IFIT5 from associating with these structures during infection. However, in a system where IFIT5 levels are increased, this saturates the inhibition pathways. Notably, we failed to observe large IBs (>7 \(\mu \mbox{m}^2\)), implying that the elevated presence of human IFIT5 may prevent the formation of large IBs.

% if 1 and 5 (cos similar)

%if 3 and 2

% ifit2 story
n-term tagged i2 detected by i2a and i2b

% investigate the ifit heteromultimer importance
sh cell lines or crispr cell lines to look at ifit multimer complex importance
pibs + i2 + i3

confocal stuff issues

super res micro with confocal

% if stable cell lines for infection/trasnfection
we tried this; issue with i2f apoptotic factor; i3 stopping cell cycle; i1 stopping translation
better to do inducible cell line

% proper controls if
stain for inert proteins or gfp/rfp and see if they associate with ib and pibs
measure ibs from non transfected cells (INFECTION TRANSFECTION)

%%% another 700
% final model of what is happening
ifits gets activated in yet uninfected cells. after infection of these i1 i2 i5 are recriuted to IBs, and this recruitment is inhibited if their concentration is not sufficient. possibly 3 gets recruited if sufficient amount of 1 and 2 is present in these structures. initialy these ifits are within the ib and as they mature and produce ibags they get pushed out within the ib edge, where the rsv genome is located. this all culminates in possibly reducing the maximal size and normal viral physiology of the IBs.

% what causes recrlitment
\cite{Oliveira2013HumanCells} - IP of n p and m2/1 doesnt show ifits as interactors
IFITI IP and Mass spec

% do intra IB ifits alter rsv replication?
minigenome \cite{Teng2016UseTranscription}


% final statement
asdasd


% Words in text: 0
% total words: 90 + 509 + 800? + 7,652 + 4,067 + 7,280 + 5,788 + 11,616 + 19,697 + 2,000-2,500? = 56,624
% total figures: 8 + 5 + 13 + 13 + 34 + 60 = 133