\chapter{Final Discussion and Future Directions}
%this much per paragraph
%Past yeast two-hybrid screens have identified additional interactions with proteins such as TID1 (unpublished) and HX-1 (Johns et al., 2010b) (discussed in 1.3.3.3 and 1.3.3.4). Recently N pro and us10, a ribosomal protein implicated in regulation of TLR3 expression, were found to interact (Lv et al., 2017b). It is unclear how modulation of any of these proteins would affect TLR3 and RIG-I-mediated apoptosis in porcine cells such as PK-15 considering how it is clearly IRF3-mediated. Knockout of the genes encoding each would help determine their involvement in apoptosis mediated by each PRR. Their functions might be accessory to the IRF3/Bax-mediated pathway of apoptosis discussed in the present work or may modulate cellular responses to infection with CSFV. Due to time constraints knockouts of TID1, HAX-1 and us10 were not generated, however they would provide an interesting avenue of study to pursue.
%discussion
%intro
Human and bovine RSV stand as the predominant causes of lower respiratory tract infections, posing significant health risks to young calves, children under 5, the elderly, and immunocompromised individuals. RSV, an enveloped, single-stranded negative-sense RNA virus, is host-restricted \textit{in vivo}. In infected cells, RSV forms membrane-less perinuclear cytoplasmic inclusion bodies (IBs), recognized as sites for viral RNA transcription and replication. These IBs also manipulate cellular components, either repurposing them for viral benefit (e.g., components of the eIF4F complex) or inhibiting their function (e.g., MAVS and MDA5). To date there is a limited information oavailable about host interferon stimulated genes that restrict RSV infection. Notable expetions exist such as the IFITM protein family \cite{Smith2019Interferon-InducedMembrane}, IFI44 \cite{McDonald2016ADisease, Li2021IdentificationVirus}, and most importantly for this project the IFIT protein family \cite{Drori2020InfluenzaProteins}. All of these were shown to impede viral replication, although the precise mechanisms of action are not know. Better understanding of how host facors impede RSV replication can lead to potential new anti RSV therapies. Alongn to this, comparing the restriction factors of different species such as in human and bovine cells can potentially lead us to better understant the RSV host restriction that is observed \textit{in vivo}. 

drori experiment
ifit functions

ifit restriction of rsv

Drori \textit{et al.} established

%recap aims
we know that i123 are anti rsv...
but is this relevant in vitro and do they express?
- understanding the intricacises of induction

% combined qpcr data
qpcr data suggest ifits are activated the most in non infected cells dusing infection

% fit qpcr with initial if and speculate bovine stuff
asdf

% for qpcr story - future steps
quantify and compare the absolute amount of copies per unit of measurement between human and bovine cells (in at least a few cell lines) to see if bIFIT and BEAS2B hIFITs are basaly more expressed

% aim 2
how do they do the negative action?

% if 1 and 5 (cos similar)
asdf

%if 3 and 2
asdf

% if compare the observed phenotypes in a range of IB sizes (small, mid, large) and what interaction with differet ifits are there
asdf

% final model of what is happening
asdf


... it seems that they interact with IBs ... is the OE and KO results from the paper because of protection prior to infection or because of intreactions with IBs? ... conclude that probs both


% future directions

% if stable cell lines for infection/trasnfection
we tried this; issue with i2f apoptotic factor; i3 stopping cell cycle; i1 stopping translation
better to do inducible cell line

% ifit2 story
n-term tagged i2 detected by i2a and i2b

% investigate the ifit heteromultimer importance
sh cell lines or crispr cell lines to look at ifit multimer complex importance
pibs + i2 + i3

% proper controls if
stain for inert proteins or gfp/rfp and see if they associate with ib and pibs
measure ibs from non transfected cells (INFECTION TRANSFECTION)

% what causes recrlitment
\cite{Oliveira2013HumanCells} - IP of n p and m2/1 doesnt show ifits as interactors
IFITI IP and Mass spec

% do intra IB ifits alter rsv replication?
minigenome

% final statement
asdasd

% Words in text: 0
% total words: 90 + 509 + 0 + 7,652 + 4,067 + 7,280 + 5,713 + 11,616 + 19,697 + 0 = 56,624
% -------------------------------------------------------------------------------------------
% random notes that will be useful later
%IBs containing visible IBAGs were significantly larger than those without 6.4 um2 versus 2.3 um2 (Rincheval et al., 2017; hep2)
%Although small IBs (-+2.5 um2) were observed as early as 6 h p.i., they did not colocalize with detectable levels of p65 (FJ thesis)
%Our pseudo-IBs were also mostly spherical and at 24 h post-transfection, measured up to 6.9 um2 , which is considerably less than that of the conventional IBs observed in infected cells. (FJ thesis)
%Visible aggregates measured above 3 um2 with the average size increasing as infection progressed (Fig 4.5B). At 48 h p.i., p65 aggregations had a mean area of 22.18 um2 but smaller puncta (<10 um2 ) were also observed, most likely the result of nascent infections in nearby cells. (FJ thesis)
%Their mean area also increased to 8.99 um2 by 24 h p.i. (FJ thesis)

%6h only N positive, mean 1; 16h n positive mean 1, n + p65 10; 24h n pos mean 1, n + p65 two clusters -> 2.5, 17 (FJ thesis)


