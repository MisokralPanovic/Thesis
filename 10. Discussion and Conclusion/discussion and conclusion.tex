\chapter{Discussion and Conclusion}
\section{Discussion}

intro ifit
intro rsv
recapitulate aims

qpcr data suggest ifits are activated the most in non infected cells dusing infection ... it seems that they interact with IBs ... is the OE and KO results from the paper because of protection prior to infection or because of intreactions with IBs? ... conclude that probs both


compare the observed phenotypes in a range of IB sizes (small, mid, large) and what interaction with differet ifits are there

do human ifits qpcr + connfocal
do bovine ifits  + confocal
compare the two


\section{Final Remarks and Future Directions}
experiments that we wanted to do but they did not work or we did not have time to do them


about stable and inducible cell lines and our try to make them but failing
ifit inducible cell lines
ifit sh cell lines+

minigenome

stain for inert proteins or gfp/rfp and see if they associate with ib and pibs

measure ibs from non transfected cells (INFECTION TRANSFECTION)


pibs + i2 + i3

IFITI IP and Mass spec
\cite{Oliveira2013HumanCells} - IP of n p and m2/1 doesnt show ifits as interactors

transfecting bi2f24 + bngfp + p inhibits bpIBs (compared to hi2f or bi2f)

n-term tagged i2 detected by i2a and i2b

% Words in text: 0
% total words: 90 + 509 + 0 + 6100 + 4100 + 5740 + 4260 + 8660 + 13,739 + 0 = 43,198

% random notes that will be useful later
IBs containing visible IBAGs were significantly larger than those without 6.4 um2 versus 2.3 um2 (Rincheval et al., 2017; hep2)
Although small IBs (-+2.5 um2) were observed as early as 6 h p.i., they did not colocalize with detectable levels of p65 (FJ thesis)
Our pseudo-IBs were also mostly spherical and at 24 h post-transfection, measured up to 6.9 um2 , which is considerably less than that of the conventional IBs observed in infected cells. (FJ thesis)
Visible aggregates measured above 3 um2 with the average size increasing as infection progressed (Fig 4.5B). At 48 h p.i., p65 aggregations had a mean area of 22.18 um2 but smaller puncta (<10 um2 ) were also observed, most likely the result of nascent infections in nearby cells. (FJ thesis)
Their mean area also increased to 8.99 um2 by 24 h p.i. (FJ thesis)

6h only N positive, mean 1; 16h n positive mean 1, n + p65 10; 24h n pos mean 1, n + p65 two clusters -> 2.5, 17 (FJ thesis)