Having established and validated the tools for the quantification of bovine \textit{IFITs} and elucidated the optimal assay termination time points for bRSV infections, we were equipped to replicate the analysis conducted in Chapter \ref{ch:Assessment of Transcriptional Induction of Human IFITs in the Context of RSV}. Our goal was to systematically assess the induction of \textit{bIFITs}, starting with different activators of the innate immune system, followed by bRSV infection, and finally, evaluating cross-species defense against hRSV infection. The relative induction levels were assessed using RT-qPCR methodology, as described in Section \ref{sec:Quantitative Real Time/Reverse Transcription PCR}. Briefly, cells were cultured in 12-well plates and exposed to the respective stimulants. At the endpoint of the experiments, RNA extraction was performed, followed by complementary DNA (cDNA) synthesis and transcript quantification through qPCR. The expression of \textit{hRSV N}, \textit{bRSV N}, and \textit{bMx1} genes was quantified using the \(\Delta\)\(\Delta\)Ct method, standardized to the bovine \textit{GAPDH} gene, and further normalized against mock-treated samples. As described previously, bovine \textit{IFIT} genes copy numbers were extrapolated from standard curves, created freshly per experiment, normalized relative to the mock copy numbers, and further standardized to the relative levels of bovine \textit{GAPDH} detected per experimental condition. The statistical analysis adhered to the procedures outlined in Section \ref{sec:Statistical Analysis}. It's important to note that the selection of the appropriate statistical test was contingent upon the assessment of data distribution normality and equality of variance, considerations that will be elaborated upon in the subsequent sections of this chapter.

\subsection{Bovine \textit{IFIT} Responses to Activators of Innate Immune Response} \label{subsec:Bovine IFIT Responses to Activators of Innate Immune Response}
The Madin-Darby bovine kidney (MDBK) cell line, derived from bovine renal epithelium in 1958 (\cite{Madin1958EstablishedOrigin}), serves as a well-established model system in bovine virology studies. We assessed the cells for the induction potential of bovine \textit{IFITs} along with bovine \textit{Mx1} using bovine interferon alpha and LPS (bovine interferon-gamma was not commercially available). Bovine \textit{Mx1} was included in the analyses due to its status as an Interferon-Stimulated Gene (ISG), widely reported in immunology and virology studies, and as we observed minimal bovine \textit{IFIT} responses throughout the study, ensuring the cell lines had functioning internal pathways for ISG induction. Figure \ref{fig:MDBK responses to bIFNa} displays the \textit{bIFIT} and \textit{bMx1} responses to stimulation with bIFN\(\alpha\) at a concentration of 5 ng/mL (equivalent to 1,000 UI/mL of hIFN\(\alpha\)) for either 3 or 6 hours.

The data reveals that all assayed genes were significantly upregulated biologically within 3 hours of induction, although the magnitude varied substantially. For the bovine \textit{IFITs}, all exhibited higher induction at 3 hours compared to 6 hours. The \textit{bIFITs} displayed a similar induction pattern to that observed in A549 cells, with \textit{bIFIT1} showing the strongest response, \textit{bIFIT2} and \textit{bIFIT3} displaying medium responses of similar amplitudes, and \textit{bIFIT5} exhibiting the lowest response. Notably, \textit{bIFIT1} showed the highest induction at 20-fold and 5-fold for 3 and 6 hours of incubation, respectively, and was the only bovine \textit{IFIT} that exhibited biologically significant induction at 6 hours. \textit{bIFIT2} and \textit{bIFIT3} were induced by 6-fold and 3.5-fold for 3 and 6-hour incubations with bIFN\(\alpha\) respectively, while \textit{bIFIT5} was induced by 4-fold and 2-fold for the same respective periods. Regarding \textit{bMx1}, its response surpassed that of the bovine \textit{IFITs}, showcasing a time-dependent induction increasing from 32-fold to 120-fold as the treatment continued. This shows that MDBK cells are capable of \textit{bIFIT} and \textit{bMx1} induction, albeit with differing temporal dynamics. However, these responses are modest, especially compared to the previously observed human \textit{IFIT} responses to IFN\(\alpha\). As a side note, while the datasets for \textit{bIFIT3} and \textit{bIFIT5} exhibited a normal distribution of data with equal variances, datasets for \textit{bIFIT1}, \textit{bIFIT2}, and \textit{bMx1} exhibited normal distributions with unequal variances.

\begin{figure}
    \centering
    \includegraphics[width=1\linewidth]{07. Chapter 2/Figs/02. Induction/01. mdbk_treat_bifna.pdf}
    \caption[\textit{bIFIT} Gene Expression in MDBK Cells in Response to bIFN\(\alpha\) Stimulation.]{\textbf{\textit{bIFIT} Gene Expression in MDBK Cells in Response to bIFN\(\alpha\) Stimulation.} (a) \textit{bIFIT1}, (b) \textit{bIFIT2}, (c) \textit{bIFIT3}, (d) \textit{bIFIT5}, and (e) \textit{bMx1} gene expression levels were assessed using quantitative real-time PCR (qPCR) in MDBK cells following stimulation with bovine interferon alpha (IFN\(\alpha\)) at a concentration of 5 ng/mL for a treatment duration of either 3 or 6 hours. Relative expression values are normalized to standardized mock-treated samples. Median values are represented by red circles. The black dotted line represents mock expression levels, while the red dotted lines indicate biologically significant induction thresholds. Numeric values indicate the p-values compared to mock-treated samples.}
    \label{fig:MDBK responses to bIFNa}
\end{figure}

Furthermore, we aimed to investigate the involvement of Toll-like Receptor 4 (TLR4) in bovine \textit{bIFIT} induction, previously observed to play a role in the A549 cell line (refer to Figure \ref{fig:A549 Response to LPS} in Section \ref{subsec:Human IFIT Responses to Activators of Innate Immune Response}). MDBK cells were incubated with LPS for 6 hours at concentrations of 0.5, 1, 2.5, 5, and 10 ng/mL. Subsequently, the cells were lysed, and their RNA was extracted, converted to cDNA, and quantified by qPCR as previously described. Our findings indicate that neither \textit{bMx1} nor \textit{bIFITs} were induced to biologically significant levels by LPS within the tested concentration range. Most time points revealed no relative change, while 2.5 ng/mL resulted in a 50\% reduction in the levels of \textit{bIFIT1}, \textit{bIFIT2}, and \textit{bIFIT3}. Evidently, the data suggests that LPS, at the tested concentrations, does not induce \textit{bMx1} or \textit{bIFITs}; if anything, it causes a slight downregulation of their expression. As a side note, while the datasets for \textit{bIFIT3}, \textit{bIFIT5}, and \textit{bMx1} exhibited a normal distribution of data with equal variances, the datasets for \textit{bIFIT1} and \textit{bIFIT2} showed normal distributions with unequal variances.


\begin{figure}
    \centering
    \includegraphics[width=1\linewidth]{07. Chapter 2/Figs/02. Induction/02. mdbk_treat_lps.pdf}
    \caption[\textit{bIFIT} Gene Expression in MDBK Cells in Response to LPS Stimulation.]{\textbf{\textit{bIFIT} Gene Expression in MDBK Cells in Response to LPS Stimulation.} (a) \textit{bIFIT1}, (b) \textit{bIFIT2}, (c) \textit{bIFIT3}, (d) \textit{bIFIT5}, and (e) \textit{bMx1} gene expression levels were assessed using quantitative real-time PCR (qPCR) in MDBK cells following stimulation with bacterial LPS at a concentration of 0.5, 1, 2.5, 5, and 10 ng/mL for a treatment duration of 6 hours. Relative expression values are normalized to standardized mock-treated samples. Median values are represented by red circles. The black dotted line represents mock expression levels, while the red dotted lines indicate biologically significant induction thresholds. Numeric values indicate the p-values compared to mock-treated samples.}
    \label{fig:MDBK responses to LPS}
\end{figure}

To validate the MDBK induction data, the BT cell line was employed. Originating from Bovine viral diarrhea virus (BVDV) negative bovine nasal turbinate cells, isolated in 1974 from a neonatal Holstein cow (\cite{McClurkin1974ComparisonVirus}), these cells, described in Section \ref{Growth Curves of Bovine RSV in Bovine Cell Lines}, facilitate bRSV replication and originate from a more physiologically significant site in the bRSV life cycle compared to the MDBK cell line. BT cells were incubated with 5 ng/mL of bIFN\(\alpha\) for either 3 hours or 24 hours. Figure \ref{fig:BT responses to bifna} depicts the results. At 3 hours of treatment, all genes, except \textit{bIFIT2}, were induced to biologically significant levels. However, after 24 hours, the relative mRNA levels returned to basal levels. Specifically, \textit{bIFIT1} displayed the highest response with a 20-fold induction, followed by \textit{bMx1} with a 16-fold increase. \textit{bIFIT3} showed an 8-fold induction, while \textit{bIFIT5} exhibited a 4-fold increase. Notably, while the dataset for \textit{bIFIT2} displayed a normal distribution of data with equal variances, all other datasets showed normal distributions with unequal variances. In summary, it can be concluded that the BT cell line is sensitive to interferon and competent in \textit{bIFIT} induction. This data aligns with the MDBK data (Figure \ref{fig:MDBK responses to bIFNa}), revealing acute responses of \textit{bIFITs} to bIFN\(\alpha\) stimulation, although the induction is not sustained by 24 hours. Differences include the loss of induction persistence of \textit{bMx1} and the absence of bIFN\(\alpha\) sensitivity in \textit{bIFIT2}.

\begin{figure}
    \centering
    \includegraphics[width=1\linewidth]{07. Chapter 2/Figs/02. Induction/08. bt_bifna.pdf}
    \caption[\textit{bIFIT} Gene Expression in BT Cells in Response to bIFN\(\alpha\) Stimulation.]{\textbf{\textit{bIFIT} Gene Expression in BT Cells in Response to bIFN\(\alpha\) Stimulation.} (a) \textit{bIFIT1}, (b) \textit{bIFIT2}, (c) \textit{bIFIT3}, (d) \textit{bIFIT5}, and (e) \textit{bMx1} gene expression levels were assessed using quantitative real-time PCR (qPCR) in BT cells following stimulation with bovine interferon alpha (IFN\(\alpha\)) at a concentration of 5 ng/mL for a treatment duration of either 3 or 24 hours. Relative expression values are normalized to standardized mock-treated samples. Median values are represented by red circles. The black dotted line represents mock expression levels, while the red dotted lines indicate biologically significant induction thresholds. Numeric values indicate the p-values compared to mock-treated samples.}
    \label{fig:BT responses to bifna}
\end{figure}

\subsection{Bovine \textit{IFITs} Responses to bRSV} \label{subsec:Bovine IFITs Responses to bRSV}
Confirming the competence of the selected cell lines for \textit{bIFIT} induction, our goal was to evaluate the impact of bRSV infection on \textit{bIFIT} induction, particularly concerning varying viral MOIs and infection durations. To achieve this, MDBK cells were infected with crudely extracted bRSV at MOIs of 0.1, 1, and 2 for durations of 24 and 48 hours post-infection (HPI). The viruses employed in these experiments were prepared and quantified as detailed in Section \ref{subsec:Virus Propagation and Production} and Section \ref{subsec:Virus Quantification by TCID50 Assay}.

\begin{figure}
    \centering
    \includegraphics[width=1\linewidth]{07. Chapter 2/Figs/02. Induction/03. mdbk_brsv_timepoints.pdf}
    \caption[MDBK \textit{bIFIT} Response to bRSV Infection as a Function of Time and MOI.]{\textbf{MDBK \textit{bIFIT} Response to bRSV Infection as a Function of Time and MOI.} (a) \textit{bIFIT1}, (b) \textit{bIFIT2}, (c) \textit{bIFIT3}, (d) \textit{bIFIT5}, (e) \textit{bMx1} and (f) \textit{bRSV N} gene expression levels were assessed using quantitative real-time PCR (qPCR) in MDBK cell line following infection with bovine RSV at MOI of either 0.1, 1, or 2 for either 24 or 48 hours post-infection. Relative expression values are normalized to standardized mock-treated samples. Median values are represented by red circles. The black dotted line represents mock expression levels, while the red dotted lines indicate biologically significant induction thresholds. Numeric values indicate the p-values compared to mock-treated samples.}
    \label{fig:MDBK responses to bRSV timepoints}
\end{figure}

The results of this experiment are depicted in Figure \ref{fig:MDBK responses to bRSV timepoints}. Evidently, while bRSV demonstrated successful replication, as indicated by the notably high relative \textit{bRSV N} values across all tested MOIs and HPIs (panel f), there were no biologically significant alterations in the mRNA levels of either \textit{bMx1} or \textit{bIFITs}. Specifically, while the mRNA levels of \textit{bIFIT3} and \textit{bIFIT5} remained unchanged under all conditions, minor positive and negative changes were observed in the expression of the other genes. Notably, \textit{bIFIT1} levels doubled for infections at 0.1 and 1 MOI at 48 HPI but decreased by half at MOI 2 at 48 HPI. \textit{bIFIT2} mRNA levels showed no significant changes except for the 2 MOI infection at 48 HPI. Additionally, \textit{bMx1} exhibited a two-fold induction in the case of 0.1 MOI infection at 24 HPI and a 50\% downregulation after 2 MOI infection at 48 HPI. From a statistical standpoint, the datasets for \textit{bIFIT3} and \textit{bMx1} showcased normal distributions and equal variances, while the others exhibited normal distributions with unequal variances. The minimal responses to bRSV infection are intriguing, particularly considering our human data from Chapter \ref{ch:Assessment of Transcriptional Induction of Human IFITs in the Context of RSV}, which indicates that the \textit{hIFIT} responses are predominantly mediated by interferon alpha. Furthermore, as discussed in Section \ref{subsec:Bovine IFIT Responses to Activators of Innate Immune Response}, we are aware that \textit{bIFITs}, especially \textit{bMx1}, respond to bovine interferon alpha stimulation. It is plausible that certain constituents within the bRSV or specific cytokines or chemokines in the crudely extracted bRSV preparations might impede the cascades necessary for \textit{bIFIT} induction.

\begin{figure}
    \centering
    \includegraphics[width=1\linewidth]{07. Chapter 2/Figs/02. Induction/04. mdbk_brsv_uv_roxo.pdf}
    \caption[Impact of Ultra-Purification, UV-Inactivation, and INFR Inhibition on \textit{bIFIT} Induction in MDBK Cells Following bRSV Infection.]{\textbf{Impact of Ultra-Purification, UV-Inactivation, and INFR Inhibition on \textit{bIFIT} Induction in MDBK Cells Following bRSV Infection.} (a) \textit{bIFIT1}, (b) \textit{bIFIT2}, (c) \textit{bIFIT3}, (d) \textit{bIFIT5}, (e) \textit{bMx1}, and (f) \textit{bRSV N} gene expression levels were assessed using quantitative real-time PCR (qPCR) in MDBK cell line following infection with ultra-purified bRSV at MOI 1 for 24 hours. The cells were subjected to three different conditions: virus infection alone (top row), virus infection in the presence of 5 nM of ruxolitinib (interferon receptor inhibitor) throughout the infection (middle row), or UV-inactivated bRSV infection (bottom row). Relative expression values are normalized to standardized mock-treated samples. Median values are represented by red circles. The black dotted line represents mock expression levels, while the red dotted lines indicate biologically significant induction thresholds. Numeric values indicate the p-values compared to mock-treated samples.}
    \label{fig:The effect of ultra-purification, UV-inactivation and INFR inhibition on hIFIT induction following hRSV infection in MDBK}
\end{figure}

To investigate if possible contaminants are preventing the gene induction, we infected the MDBK cells with ultrapurified bRSV, prepared by ultra-centrifugation on a discontinuous sucrose cushion, as described in Section \ref{subsec:Virus Propagation and Production}. Along to this, we wanted to assess the effect of pharmacological inhibition of interferon receptor and physical bRSV inactivation on \textit{bIFIT} and \textit{bMx1} induction. We did this as we previously decribed in Section \ref{subsec:Human IFITs Responses to Human RSV}, Figure \ref{fig:The effect of ultra-purification, UV-inactivation and INFR inhibition on hIFIT induction following hRSV infection in BEAS2B} how basal interferon receptor activation is required for the maintainance of basal \textit{hIFIT} mRNA expression. The data can be observed in Figure \ref{fig:The effect of ultra-purification, UV-inactivation and INFR inhibition on hIFIT induction following hRSV infection in MDBK}. We can see that the viral purification status did not influence the induction of \textit{bIFITs}, however, ultra-purification caused a 7-fold induction of \textit{bMx1}. This was reversed completely by the presence of interferon receptor inhibitor, ruxolitinib. Its presence also slightly decreased the abundance of all \textit{bIFIT} mRNA, although not to a biologically significant levels. Interestingly, the relative median \textit{bRSV N} mRNA value was maintained between the first two conditions. This is in contrast to what we observed in human samples where the presence of ruxolitinib increased the relative median \textit{bRSV N} mRNA. Lastly, we can see that UV-inactivation of bRSV caused only small changes for all genes of magnitude of circa \(\pm\)2. Taken together, we can see that \textit{bMx1} induction response mirrors to what was observed with human RSV in A549 and BEAS2B cell lines in Chapter \ref{ch:Assessment of Transcriptional Induction of Human IFITs in the Context of RSV}. On the other hand, we did not see any significant alterations of \textit{bIFIT} mRNA levels. This suggest that some cellular contaminant present in crude extracted bRSV preparations were suppressing the induction of \textit{bMx1}, but their absence was irrelevant to the potential \textit{bIFIT} induction inhibition. As a side note, while \textit{bRSV N} and \textit{bMx1} datasets exhibited normal distribution of data with unequal variances, \textit{bIFIT1}, \textit{bIFIT2}, \textit{bIFIT3}, and \textit{bIFIT5} datasets exhibited normal distributions with equal variances.

%%%%%%%%%%%%%%%%%%%%%%%%%%%%%%%%%%%%%%%%%%%%%%%%%%%%%%%%%%%%%%%%%%%%%%%%%%%%%%%%%%%%%%%%%%%%%%%%

\begin{figure}
    \centering
    \includegraphics[width=1\linewidth]{07. Chapter 2/Figs/02. Induction/05. mdbk_brsv_moi1_dsh.pdf}
    \caption[MDBK \textit{bIFIT} Response to WT and \(\Delta\)SH bRSV Infection.]{\textbf{MDBK \textit{bIFIT} Response to WT and \(\Delta\)SH bRSV Infection.} (a) \textit{bIFIT1}, (b) \textit{bIFIT2}, (c) \textit{bIFIT3}, (d) \textit{bIFIT5}, (e) \textit{bMx1}, and (f) \textit{bRSV N} gene expression levels were assessed using quantitative real-time PCR (qPCR) in MDBK cell line following infection with WT or \(\Delta\)SH bRSV at MOI 1 for 24 hours post-infection. Relative expression values are normalized to standardized mock-treated samples. Median values are represented by red circles. The black dotted line represents mock expression levels, while the red dotted lines indicate biologically significant induction thresholds. Numeric values indicate the p-values compared to mock-treated samples.}
    \label{fig:MDBK responses to dSH}
\end{figure}

Next we wanted to establish if any of the bRSV constituents casues the inhibition of \textit{bIFIT} induction or if they are not physioligically involved during bRSV infection. As presented in Chapter \ref{ch:Introduction} Section \ref{subsec:Composition}, RSV proteins SH, NS1 and NS2 are known for their inhibitory action on the innate immune pathways through the interaction with their constituents (add citation and alter the information). We hypothetised that these proteins prevent the induction of \textit{bIFITs}. In the Viral Glycoproteins Group in the Pirbright Institute, we had a panel of bRSV deletion mutants, namely bRSV \(\Delta\)SH, \(\Delta\)NS1, \(\Delta\)NS2, and the double deletion mutant \(\Delta\)NS1/2 and we used these to investigate this hypothesis. The viruses were propagated and tittered as described in Section \ref{subsec:Virus Propagation and Production} and Section \ref{subsec:Virus Quantification by TCID50 Assay}. Briefly, infected cells were subjected to sonication, followed by centrifugation to separate cell debris, ultimately yielding virus-containing supernatants that were quantified through titration assays. The lack non structural proteins culminated in lower end titres. Due to this, we firstly infected the cells with MOI 1 WT and \(\Delta\)SH bRSV for 24 hours, and in subsequent experiment we infected the cells with 0.001 MOI \(\Delta\)NSs and WT bRSV for the duration of 24 hours.

Figure \ref{fig:MDBK responses to dSH} shows the relative mRNA changes of \textit{bIFITs} and \textit{bMx1}, extracted from MDBK cells 24 HPI with MOI 1 WT and \(\Delta\)SH bRSV. We can see that both viruses were successfuly replicationg, although \(\Delta\)SH bRSV reached higher median relative mRNA levels at the end point of the experiment compared to the WT bRSV. This is unusual, as this virus should have decreased fitness. Regardless, we can observe that \textit{bIFITs} and \textit{bMx1} responses to crudly extracted WT bRSV were consistent with what was observed presiously, meaning we see minimal responses to the infection (Figure \ref{fig:MDBK responses to bRSV timepoints}). \(\Delta\)SH bRSV infection did not cause any differential responses for any genes tested compared to WT bRSV infection but for \textit{bIFIT2}, which median relative mRNA abundance increased to biologically significant levels at 4-fold. As a side note, while  and \textit{bMx1} dataset exhibited normal distribution of data with equal variances, \textit{bIFIT1}, \textit{bIFIT2}, \textit{bIFIT3}, \textit{bIFIT5}, \textit{bRSV N} datasets exhibited normal distributions with unequal variances.

\begin{figure}
    \centering
    \includegraphics[width=1\linewidth]{07. Chapter 2/Figs/02. Induction/06. mdbk_brsv_low_moi.pdf}
    \caption[MDBK \textit{bIFIT} Response to Low MOI bRSV Infections.]{\textbf{MDBK \textit{bIFIT} Response to Low MOI bRSV Infections.} (a) \textit{bIFIT1}, (b) \textit{bIFIT2}, (c) \textit{bIFIT3}, (d) \textit{bIFIT5}, (e) \textit{bMx1}, and (f) \textit{bRSV N} gene expression levels were assessed using quantitative real-time PCR (qPCR) in MDBK cell line following infection with WT or \(\Delta\)NS1, \(\Delta\)NS2, and \(\Delta\)NS1/2 bRSV at MOIs of 0.001 for 24 hours post-infection. Relative expression values are normalized to standardized mock-treated samples. Median values are represented by red circles. The black dotted line represents mock expression levels, while the red dotted lines indicate biologically significant induction thresholds. Numeric values indicate the p-values compared to mock-treated samples.}
    \label{fig:MDBK responses to low MOI mutant bRSV}
\end{figure}

As mentioned above, further we used 0.001 MOI \(\Delta\)NS1, \(\Delta\)NS2, and \(\Delta\)NS1/2 bRSV along with 0.001 MOI WT bRSV control to assess their involvement in the potential suppresion of \textit{bIFIT} induction. As we can see in Figure \ref{fig:MDBK responses to low MOI mutant bRSV}, none of the genes of interested are influenced by WT infection, which is consistent with our prevoius observations. With regards to the mutant viruses, \textit{bIFIT5} was not responsive to either of them, while all the other genes were induced slighlty to at least one condition. \(\Delta\)NS1 infection influenced only \textit{bIFIT2}, which was induced nearly to the biologically significant levels at around 3.8-fold. The absence of NS2 protein caused small induction in all genes tested with the exeption of \textit{bIFIT5}, more precisely circa 2-fold increase for \textit{bIFIT1}, \textit{bIFIT3}, and \textit{bMx1} and 3.8-fold induction for \textit{bIFIT2}. The infection with double deletion mutant \(\Delta\)NS1/2 bRSV casued the best response with 3.8-fold induction of \textit{bIFIT1} and \textit{bIFIT3}, 3-fold induction of \textit{bMx1}, and most importantly, a biologically significant induction of \textit{bIFIT2} at 5-fold. As a side note, while \textit{bIFIT5}, \textit{bRSV N} and \textit{bMx1} datasets exhibited normal distribution of data with unequal variances, \textit{bIFIT1}, \textit{bIFIT2}, and \textit{bIFIT3} datasets exhibited normal distributions with equal variances. This data suggest that, with the exeption of \textit{bIFIT5}, bRSV NS protein negativelly influence the induction of \textit{bIFITs} and \textit{bMx1}, with NS2 seemingly being a more potent inhibitor.

%%%%%%%%%%%%%%%%%%%%%%%%%%%%%%%%%%%%%%%%%%%%%%%%%%%%%%%%%%%%%%%%%%%%%%%%%%%%%%%%%%%%%%%%%%%%%%%%

\begin{figure}
    \centering
    \includegraphics[width=1\linewidth]{07. Chapter 2/Figs/02. Induction/09. bt_brsv.pdf}
    \caption[BT \textit{bIFIT} Response to WT and \(\Delta\)SH bRSV Infection.]{\textbf{BT \textit{bIFIT} Response to WT and \(\Delta\)SH bRSV Infection.} (a) \textit{bIFIT1}, (b) \textit{bIFIT2}, (c) \textit{bIFIT3}, (d) \textit{bIFIT5}, (e) \textit{bMx1}, and (f) \textit{bRSV N} gene expression levels were assessed using quantitative real-time PCR (qPCR) in BT cell line following infection with WT or \(\Delta\)SH bRSV at MOI 1 for 24 hours post-infection. Relative expression values are normalized to standardized mock-treated samples. Median values are represented by red circles. The black dotted line represents mock expression levels, while the red dotted lines indicate biologically significant induction thresholds. Numeric values indicate the p-values compared to mock-treated samples.}
    \label{fig:BT responses to bRSV}
\end{figure}

Lastly, we set to partially validate the bRSV infection data using BT cell line. We infected the cells with crudely extracted WT and \(\Delta\)SH bRSV at MOI 1 for 24 hours. The underlying data is displayed in Figure \ref{fig:BT responses to bRSV}. As a side note, all datasets were of normal distribution and equal variance other than \textit{bIFIT1} and \textit{bRSV N}. In general, we see a reduction in mRNA detected as a result of infection, regardless of the virus used. In more detail, WT bRSV infection causes \(2^{-0.5}\)-fold decrease for \textit{bIFIT1} and \textit{bIFIT5}, \(2^{-1}\)-fold decrease for \textit{bIFIT2} and \textit{bIFIT3}, and \(2^{-1.5}\)-fold decrease for \textit{bMx1}. Infection with \(\Delta\)SH bRSV caused around \(2^{-1.5}\)-fold decrease of the levels of \textit{bIFIT1} and \textit{bMx1}; biologically significant decrease of \(2^{-2}\)-fold for \textit{bIFIT2} and \textit{bIFIT3}; and relulted in no alteration of \textit{bIFIT5} mRNA levels. Collectively, the presence of SH protein seems to be stimulating the \textit{bIFIT} and \textit{bMx1} induction. This is in contrast to what was observed in MDBK cell line, where there were no observed differeces in the induction potential of WT and \(\Delta\)SH bRSV, other than for \textit{bIFIT2} which was significantly upregulated in the absence of SH protein (Figure \ref{fig:MDBK responses to dSH}).

%%%%%%%%%%%%%%%%%%%%%%%%%%%%%%%%%%%%%%%%%%%%%%%%%%%%%%%%%%%%%%%%%%%%%%%%%%%%%%%%%%%%%%%%%%%%%%%%

\subsection{Bovine \textit{IFITs} Responses to hRSV} \label{subsec:Bovine IFITs Responses to hRSV}
Further, we wanted to investigate the potential cross-species protection between human RSV (hRSV) and bovine RSV (bRSV). As we have seen in Chapter \ref{ch:Assessment of Transcriptional Induction of Human IFITs in the Context of RSV} Section \ref{subsec:Human IFITs Responses to bRSV}, bRSV is capable of a \textit{hIFIT} induction which superseseed the levels of equivalent hRSV infection, suggesting human cells are responsive and cross-protected to both viruses. Although we so far observed minimal \textit{bIFIT} or \textit{bMx1} response to WT or mutant bRSV infection we wanted to assess the potential response to hRSV. We also observed in Chapter \ref{ch:Assessment of Transcriptional Induction of Human IFITs in the Context of RSV} Section \ref{subsec:Human IFITs Responses to Human RSV} that the purification methodology used to isolate hRSV had an effect on \textit{hIFIT} induction, with ultrapurified preapration causing higher magnitudes of induction. Taking all this into consediration, we infected MDBK and BT cells with crude extracted and ultracentrifugation-purified hRSV at MOI 1 for 24 hours. Cells were subsequently lysed, their mRNA extracted and converted into cDNA. Latly, the transcriopts were quantified using RT-qPCR.

\begin{figure}
    \centering
    \includegraphics[width=1\linewidth]{07. Chapter 2/Figs/02. Induction/07. mdbk_hrsv.pdf}
    \caption[MDBK \textit{bIFIT} Response to Crude-Extracted and Ultra-Purified hRSV Infection.]{\textbf{MDBK \textit{bIFIT} Response to Crude-Extracted and Ultra-Purified hRSV Infection.} (a) \textit{bIFIT1}, (b) \textit{bIFIT2}, (c) \textit{bIFIT3}, (d) \textit{bIFIT5}, (e) \textit{bMx1}, and (f) \textit{hRSV N} gene expression levels were assessed using quantitative real-time PCR (qPCR) in MDBK cell line following infection with crude-extraccted and ultra-purified hRSV at MOI 1 for 24 hours post-infection. Relative expression values are normalized to standardized mock-treated samples. Median values are represented by red circles. The black dotted line represents mock expression levels, while the red dotted lines indicate biologically significant induction thresholds. Numeric values indicate the p-values compared to mock-treated samples.}
    \label{fig:bIFIT responses to hRSV infection in MDBK}
\end{figure}

Figure \ref{fig:bIFIT responses to hRSV infection in MDBK} shows the responses of \textit{bIFITs} and \textit{bMx1} to hRSV in MDBK. Firstly, we can see that both viral preparations were able to sucessfuly replicate, as can be seen by the quantification of \textit{hRSV N} (Panel f; \(2^{5}\)-fold and \(2^{4.2}\)-fold increases respectively), although unexpeectedly there is a order of magnitude difference between the final fold changes. Fuirther, we can observe ultrapurified hRSV infection did not influence the relative levels of any of the genes tested. On the other hand, crude extracted hRSV infection causes a variety of effects. These include a small induction of \textit{bIFIT1} by 1.5-fold, and \textit{bIFIT2}, \textit{bIFIT3}, and \textit{bMx1} by 3-fold; as well as a significant dowregulation of \textit{bIFIT5} by \(2^{-2}\)-fold. \textit{bIFIT1}, \textit{bIFIT2}, and \textit{bIFIT3} datasets were of normal distribution with equal variances, while the rest exhibited normal distributions with unequal variances.

\begin{figure}
    \centering
    \includegraphics[width=1\linewidth]{07. Chapter 2/Figs/02. Induction/10. bt_hrsv.pdf}
    \caption[BT \textit{bIFIT} Response to Crude-Extracted and Ultra-Purified hRSV Infection.]{\textbf{BT \textit{bIFIT} Response to Crude-Extracted and Ultra-Purified hRSV Infection.} (a) \textit{bIFIT1}, (b) \textit{bIFIT2}, (c) \textit{bIFIT3}, (d) \textit{bIFIT5}, (e) \textit{bMx1}, and (f) \textit{hRSV N} gene expression levels were assessed using quantitative real-time PCR (qPCR) in BT cell line following infection with crude-extraccted and ultra-purified hRSV at MOI 1 for 24 hours post-infection. Relative expression values are normalized to standardized mock-treated samples. Median values are represented by red circles. The black dotted line represents mock expression levels, while the red dotted lines indicate biologically significant induction thresholds. Numeric values indicate the p-values compared to mock-treated samples.}
    \label{fig:Bt responses to hRSV}
\end{figure}

The results from using BT cells infected with hRSV can be seen in Figure \ref{fig:Bt responses to hRSV}. We observe that like in MDBK cells, hRSV is capable of infecting and sucesfully replicating in BT cells as well. We can also see the oposite effect on what we observed with MDBK above (Figure \ref{fig:bIFIT responses to hRSV infection in MDBK}), meaning the crude extracted viral preparation did not influence the expression of genes of interest, and ultra-purified hRSV caused significant effects. Namely, it caused a biologically significant downregulation of \textit{bIFIT1}, \textit{bIFIT2}, \textit{bIFIT3}, and \textit{bIFIT5} by \(2^{-2.5}\)-fold and induced \textit{bMx1} by 2-fold. \textit{bIFIT1} and \textit{bIFIT2} datasets were of normal distribution with equal variances, while the others were of normal distributions with unequal variances.

%%%%%%%%%%%%%%%%%%%%%%%%%%%%%%%%%%%%%%%%%%%%%%%%%%%%%%%%%%%%%%%%%%%%%%%%%%%%%%%%%%%%%%%%%%%%%%%%