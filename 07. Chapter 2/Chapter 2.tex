\chapter{Assessment of Transcriptional Induction of Bovine IFITs in the Context of RSV} \label{ch:Assessment of Transcriptional Induction of Bovine IFITs in the Context of RSV}
\section{Background and Aims} \label{sec:Background and Aims Chapter2}
From published literature, it is evident that induced human IFIT expression negatively impacts RSV replication. The results obtained in Chapter \ref{ch:Assessment of Transcriptional Induction of Human IFITs in the Context of RSV} confirm differential expression of human IFITs in response to both hRSV and bRSV infections. However, the investigation into bovine IFIT induction and its potential influence on bovine RSV has not been undertaken to date.

Our hypothesis posited that bovine IFITs would be induced by both human and bovine RSV, similar to the observation made for human IFITs in the previous chapter. Our objective was to systematically test this hypothesis, employing methodologies akin to those used in Chapter \ref{ch:Assessment of Transcriptional Induction of Human IFITs in the Context of RSV}. Initially, our efforts were directed towards preparing essential technological materials and information, such as qPCR primers for bIFIT detection. Presently, no commercially available primers exist for this purpose. Additionally, we aimed to assess bRSV growth in various model bovine cell lines employed in our study. Subsequently, we sought to confirm the capability of our model cell lines to induce IFITs following the treatment with known activators of the innate immune system, such as interferon alpha and LPS. Following this, our strategy involved evaluating IFIT induction during both human and bovine RSV infections, utilizing a range of viral concentrations and various end assay time points.

\section{Results} \label{sec:Results Chapter2}
\subsection{Technology Establishment} \label{subsec:Technology Establishment}
\subsubsection{Primer Design and Validation for Bovine \textit{IFIT} Quantification} \label{Primer Design and Validation for Bovine IFIT Quantification}
Due to the initial lack of commercially available primers for the detection of bovine \textit{IFIT} transcripts at the outset of the project, we devised a panel comprising three primer sets (PS) for each bovine \textit{IFIT} gene. Detailed information about this process is outlined in Section \ref{subsec:Primer Design and Assay Setup}. In a nutshell, we inputted the coding sequences into the PrimerQuest software (Integrated DNA Technologies) to identify the most suitable oligonucleotides. To evaluate the amplification efficiencies of each primer set, we employed \textit{IFIT} DNA clones from a bovine ISG library as standards (accessible through a collaboration with CVR Glasgow). The outcomes are depicted in Figure \ref{fig:Validation of custom-made bIFIT qPCR primers}. The graph demonstrates that primer sets 1 exhibited the most favourable amplification efficiencies. All primer sets, except for \textit{bIFIT3}, yielded nearly impeccable amplification efficiencies of around 100\%. Consequently, they were chosen for subsequent experiments. While \textit{bIFIT3} primer sets demonstrated similar outcomes in terms of standard curve slopes and amplification efficiencies, PS1 consistently outperformed the others in repeated testing rounds (data not presented). As a result, it was singled out for further experimentation.

\begin{figure}
    \centering
    \includegraphics[width=1\linewidth]{07. Chapter 2/Figs/01. Technologies/02. primer validation.pdf}
    \caption[Validation of Custom-Made \textit{bIFIT} qPCR Primers.]{\textbf{Validation of Custom-Made \textit{bIFIT} qPCR Primers.} The custom-designed primers were evaluated by creating a serial dilution of bovine \textit{IFIT}-containing plasmids, provided by the CVR Glasgow. The resulting standard curves are shown here. Primer set (PS) 1 (a), 2 (b), and 3 (c) are depicted for bovine \textit{IFIT1} (1.), \textit{IFIT2} (2.), \textit{IFIT3} (3.), and \textit{IFIT5} (4.), along with their calculated amplification efficiencies.}
    \label{fig:Validation of custom-made bIFIT qPCR primers}
\end{figure}

The PSs behaviour was monitored throughout the project, as fresh standard curves were created per experiment. Figure \ref{fig:The Performance of Custom Made Primer Sets Over Time} shows that the average data is consistent with what was observed in initial testing (Figure \ref{fig:Validation of custom-made bIFIT qPCR primers}), however, there were per experiment deviations in slope angles for each of the selected primer pairs. The underlying amplification efficiencies stayed consistent, as is highlighted by the averaged efficiencies displayed. The initial \textit{bIFIT3} PSs differential amplification slopes compared to the other \textit{bIFIT} PSs, as well as the variable nature of PSs performance throughout the project prohibits the usage of \(\Delta\)\(\Delta\)Ct methodologies for transcript quantification as the increase in cycle threshold would not be proportional to the decrease of transcript abundance between the \textit{bIFITs}, and thus a different methodology had to be adopted. This is described in detail in Section \ref{subsec:Data Processing}. In short, the copy numbers were deducted from standard curves and factorised by the relative abundance of bovine \textit{GAPDH}. This ensured the slope-independent establishment of relative expression values, mirroring and complementing data from \(\Delta\)\(\Delta\)Ct methodologies.

\begin{figure}
    \centering
    \includegraphics[width=1\linewidth]{07. Chapter 2/Figs/01. Technologies/03. standard curves behaviour.pdf}
    \caption[The Performance of Custom-Made Primer-Sets Over Time.]{\textbf{The Performance of Custom-Made Primer-Sets Over Time.} During the experiments with custom-made \textit{bIFIT} qPCR primers, standard curves had to be always constructed. Here, the underlying average amplification efficiencies and standard curves, along with the individual data, from all the experiments and coloured by the experiment are displayed.}
    \label{fig:The Performance of Custom Made Primer Sets Over Time}
\end{figure}

\subsubsection{Growth Curves of Bovine RSV in Bovine Cell Lines} \label{Growth Curves of Bovine RSV in Bovine Cell Lines}
Although my colleagues in the Viral Glycoproteins Group from the Pirbright Institute routinely perform hRSV infections of the MDBK cell line, the data in the BT cell line is currently lacking. Therefore we set up bRSV growth curves in both cell lines side by side to investigate which time points would be relevant for further infection experiments. Crude-extracted bRSV, isolated as described in Section \ref{subsec:Virus Propagation and Production} and quantified as described in Section \ref{subsec:Virus Quantification by TCID50 Assay} was used for the establishment of growth curves as described in Section \ref{subsec:Viral Growth Curves}. In brief, the MDBK and the BT cell line were seeded in 96-well plates and infected with WT bRSV at MOI 0.1. The supernatant and cellular fractions were collected at time intervals of 24, 48, 72, 96, 120, 144, and 168 hours post-infection, snap frozen in dry ice ethanol mixture, and later tittered as described in Section \ref{subsec:Virus Quantification by TCID50 Assay}. As can be observed in Figure \ref{fig:bRSV growth curves in MDBK and BT cell lines}, there is a differential infection profile between the cell lines. bRSV growth in the MDBK cell line was initially exponential, starting at \(10^{2.5}\) viral titre solely detected in the cytosolic fraction at 24 HPI, it increased to \(10^{4}\) and \(10^{3}\) for the cellular and supernatant fractions respectively at 48 hours. At 72 hours, the cellular viral titre reached \(10^{4.5}\), while the supernatant one further increased to \(10^{3.5}\). Afterwards, the cellular titre remained stable until 120 hours, after which it declined by more than an order of magnitude to \(10^{3}\) at 144 hours and then further increased again to \(10^{3.5}\) at 168 hours. The supernatant titre increased to \(10^{4.5}\) and \(10^{5.5}\) at 120 and 144 hours respectively, and afterwards it decreased to \(10^{4.5}\) again at 168 hours. The peak combined titre occurred at 144 hours but it was robust from 48 hours onwards. bRSV growth in BT cell line displayed virtually no titre from supernatant fraction at 24 and 48 hours. Regardless, the 24-hour time point was the peak value of the combined, as well as cellular titre with cellular titres of \(10^{4}\) at 24 hours and \(10^{3}\) at 48 hours. Cellular titres further declined to \(10^{2}\) at 72 hours, and since then they remained stable other than a small increase to \(10^{2.5}\) at 120 HPI. On the other hand, supernatant titre peaked at 72 hours at \(10^{4}\), slightly decreased to \(10^{3.5}\) at 120 hours and afterwards stayed stable until the end of the experiment. From this experiment, we can conclude that the viral replication kinetics differ substantially between the two cell lines and the data suggests that we can be certain of viral replication at 48 HPI in MDBK cell line and 24 HPI in the BT cell line.

\begin{figure}
    \centering
    \includegraphics[width=1\linewidth]{07. Chapter 2/Figs/01. Technologies/01. growth_curves.pdf}
    \caption[bRSV Growth Curves in MDBK and BT Cell Lines.]{\textbf{bRSV Growth Curves in MDBK and BT Cell Lines.} MDBK and BT cell lines were infected with 0.1 MOI wild-type bRSV and the cellular (top panel) and supernatant (bottom panel) fractions were extracted at time intervals of 24, 48, 72, 96, 120, 144, and 168 hours post-infection. These were subsequently quantified by TCID50 methodology. The total viral titre is shown in the middle panel.}
    \label{fig:bRSV growth curves in MDBK and BT cell lines}
\end{figure}

\subsection{Bovine \textit{IFIT} Responses to Activators of Innate Immune Response} \label{subsec:Bovine IFIT Responses to Activators of Innate Immune Response}
The Madin-Darby bovine kidney (MDBK) cell line, derived from bovine renal epithelium in 1958 (\cite{Madin1958EstablishedOrigin}), is a well-established model system used in bovine virology studies. We assayed the cells for the induction potential of bovine \textit{IFITs}, alongside bovine \textit{Mx1} using bovine interferon alpha and LPS (bovine interferon-gamma was not available commercially). Bovine \textit{Mx1} was included in the analyses as it is a ISG, widely reported in immunology and virology studies (ADD SOME REFERENCES HERE FOR CRYING OUT LOUD), and because we saw minimal bovine \textit{IFIT} responses throughout the study and wanted to ensure the cell lines used had the internal pathways for ISG induction correctly functioning. Figure \ref{fig:MDBK responses to bIFNa} shows the \textit{bIFIT} and \textit{bMx1} responses to the stimulation with bIFN\(\alpha\) at a concentration of 5 ng/mL (equivalent to 1,000 UI/mL of hIFN\(\alpha\)) for either 3 or 6 hours. Interestingly, we see a similar effect in induction amplitude but an opposing effect in time of stimulation to amplitude compared to hIFN\(\alpha\) induction in BEAS-2B cells (Figure \ref{fig:BEAS-2B responses to hIFNa}). DESCRIBE THE DATA ITSELF ....


all ifits:
    sig at 3
        1 highest, 2 3 the same, 5 lowest (like a549)
    higher 3 than 6
    1 sig at 6
    bmx1 sig at both
        3 double 1
        6 even higher (4x)
        shows that ifn stim worked and ifits are fast acting fast degrading
        also shows mdbk ifits are competent and ifn cascades work

normal unequal - bi1 bi2 bmx1

normal equal - bi3 bi5

mdbk bifna stim

bMx1 was induced by all bIFN alpha at different concentrations and timepoints, but at 5 ng/mL for 24h. This means that either that treatment failed or 24h post adding the bIFN treatment is too late to capture the ISG induction. All targets are induced by bIFN alpha treatment at 5 ng/mL for 3 hours. bIFIT1 also responds to low concentration (0.5 ng/mL) for 6h treatment, but not the other IFITs. No other concentration/time combination induced bIFITs. 

This all suggests that MDBK are responsive to bIFN alpha and capable of bIFIT induction, but the responses are week, especially compared to human cells.

I have a hypothesis that in bovine cells the IFITs are basally expressed to higher levels  (that would explain why we can detect them by IF in mock and infected cell although the qPCR data suggest there is no induction).

\begin{figure}
    \centering
    \includegraphics[width=1\linewidth]{07. Chapter 2/Figs/02. Induction/01. mdbk_treat_bifna.pdf}
    \caption[\textit{bIFIT} Gene Expression in MDBK Cells in Response to bIFN\(\alpha\) Stimulation.]{\textbf{\textit{bIFIT} Gene Expression in MDBK Cells in Response to bIFN\(\alpha\) Stimulation.} (a) \textit{bIFIT1}, (b) \textit{bIFIT2}, (c) \textit{bIFIT3}, (d) \textit{bIFIT5}, and (e) \textit{bMx1} gene expression levels were assessed using quantitative real-time PCR (qPCR) in MDBK cells following stimulation with bovine interferon alpha (IFN\(\alpha\)) at a concentration of 5 ng/mL for a treatment duration of either 3 or 6 hours. Relative expression values are normalized to standardized mock-treated samples. Median values are represented by red circles. The black dotted line represents mock expression levels, while the red dotted lines indicate biologically significant induction thresholds. Numeric values indicate the p-values compared to mock-treated samples.}
    \label{fig:MDBK responses to bIFNa}
\end{figure}

mdbk lps

Neither bMx1 nor bIFITs are induced by LPS in the concentration range tested. 

nothing responding to lps
0.5 2.5 1, 2.5 2, 2.5 3 - inhibition by Half
    while no response in bmx1 and 5
    not biologically significant probs

normal unequal - bi1 bi2 

normal equal - bi3 bi5 bmx1


\begin{figure}
    \centering
    \includegraphics[width=1\linewidth]{07. Chapter 2/Figs/02. Induction/02. mdbk_treat_lps.pdf}
    \caption[\textit{bIFIT} Gene Expression in MDBK Cells in Response to LPS Stimulation.]{\textbf{\textit{bIFIT} Gene Expression in MDBK Cells in Response to LPS Stimulation.} (a) \textit{bIFIT1}, (b) \textit{bIFIT2}, (c) \textit{bIFIT3}, (d) \textit{bIFIT5}, and (e) \textit{bMx1} gene expression levels were assessed using quantitative real-time PCR (qPCR) in MDBK cells following stimulation with bacterial LPS at a concentration of 0.5, 1, 2.5, 5, and 10 ng/mL for a treatment duration of 6 hours. Relative expression values are normalized to standardized mock-treated samples. Median values are represented by red circles. The black dotted line represents mock expression levels, while the red dotted lines indicate biologically significant induction thresholds. Numeric values indicate the p-values compared to mock-treated samples.}
    \label{fig:MDBK responses to LPS}
\end{figure}

bt intro

bt bifna

% bt paper
(\cite{McClurkin1974ComparisonVirus})

Validation in more physiologically relevant cell line. All genes but bIFIT2 respond to bIFN alpha 5 ng/mL for 3h; treatment for 24h cause no change in any of the genes; treatment for 6h downregulates IFITs but not bMx1. This shows that BT cells are responsive to IFN and have the capability to express bIFITs and bMx1.

2 not responsive
everything else sig induction at 3 and minimal induction for bmx1 and 5 at 24
shows bt being ifit competent, ifn cascades work, fast action that doesnt last a day


normal unequal - bi1 bi3 bi5 bmx1

normal euqal - bi2 


\begin{figure}
    \centering
    \includegraphics[width=1\linewidth]{07. Chapter 2/Figs/02. Induction/08. bt_bifna.pdf}
    \caption[\textit{bIFIT} Gene Expression in BT Cells in Response to bIFN\(\alpha\) Stimulation.]{\textbf{\textit{bIFIT} Gene Expression in BT Cells in Response to bIFN\(\alpha\) Stimulation.} (a) \textit{bIFIT1}, (b) \textit{bIFIT2}, (c) \textit{bIFIT3}, (d) \textit{bIFIT5}, and (e) \textit{bMx1} gene expression levels were assessed using quantitative real-time PCR (qPCR) in BT cells following stimulation with bovine interferon alpha (IFN\(\alpha\)) at a concentration of 5 ng/mL for a treatment duration of either 3 or 24 hours. Relative expression values are normalized to standardized mock-treated samples. Median values are represented by red circles. The black dotted line represents mock expression levels, while the red dotted lines indicate biologically significant induction thresholds. Numeric values indicate the p-values compared to mock-treated samples.}
    \label{fig:BT responses to bifna}
\end{figure}

\subsection{Bovine \textit{IFITs} Responses to bRSV} \label{subsec:Bovine IFITs Responses to bRSV}

intro about infections - methods and so


\begin{figure}
    \centering
    \includegraphics[width=1\linewidth]{07. Chapter 2/Figs/02. Induction/03. mdbk_brsv_timepoints.pdf}
    \caption[MDBK \textit{bIFIT} Response to bRSV Infection as a Function of Time and MOI.]{\textbf{MDBK \textit{bIFIT} Response to bRSV Infection as a Function of Time and MOI.} (a) \textit{bIFIT1}, (b) \textit{bIFIT2}, (c) \textit{bIFIT3}, (d) \textit{bIFIT5}, (e) \textit{bMx1} and (f) \textit{bRSV N} gene expression levels were assessed using quantitative real-time PCR (qPCR) in MDBK cell line following infection with bovine RSV at MOI of either 0.1, 1, or 2 for either 24 or 48 hours post-infection. Relative expression values are normalized to standardized mock-treated samples. Median values are represented by red circles. The black dotted line represents mock expression levels, while the red dotted lines indicate biologically significant induction thresholds. Numeric values indicate the p-values compared to mock-treated samples.}
    \label{fig:MDBK responses to bRSV timepoints}
\end{figure}

timepoints data

Low, mid, and high MOI (0.1, 1, 2) and two different time points (24 and 48 HPI) do not seem to influence the levels of any genes

no biologicaly sig response
very strong viral replication
1 48 0.1 1 half increase; 2 half decrease
2 48 2 half decrease
3 and 5 nothing
mx1 24 0.1 half increase; 48 2 half decrease
viruses replicating well but not stimulating; 48 should be quite high titre based on growth curves and still nothing

normal unequal - bi1 bi2  bi5 brsvn

normal euqal - bi3 bmx1

\begin{figure}
    \centering
    \includegraphics[width=1\linewidth]{07. Chapter 2/Figs/02. Induction/04. mdbk_brsv_uv_roxo.pdf}
    \caption[Impact of Ultra-Purification, UV-Inactivation, and INFR Inhibition on \textit{bIFIT} Induction in MDBK Cells Following bRSV Infection.]{\textbf{Impact of Ultra-Purification, UV-Inactivation, and INFR Inhibition on \textit{bIFIT} Induction in MDBK Cells Following bRSV Infection.} (a) \textit{bIFIT1}, (b) \textit{bIFIT2}, (c) \textit{bIFIT3}, (d) \textit{bIFIT5}, (e) \textit{bMx1}, and (f) \textit{bRSV N} gene expression levels were assessed using quantitative real-time PCR (qPCR) in MDBK cell line following infection with ultra-purified bRSV at MOI 1 for 24 hours. The cells were subjected to three different conditions: virus infection alone (top row), virus infection in the presence of 5 nM of ruxolitinib (interferon receptor inhibitor) throughout the infection (middle row), or UV-inactivated bRSV infection (bottom row). Relative expression values are normalized to standardized mock-treated samples. Median values are represented by red circles. The black dotted line represents mock expression levels, while the red dotted lines indicate biologically significant induction thresholds. Numeric values indicate the p-values compared to mock-treated samples.}
    \label{fig:The effect of ultra-purification, UV-inactivation and INFR inhibition on hIFIT induction following hRSV infection in MDBK}
\end{figure}


mdbk uv roxo puri

UV inactivated bRSV causes no change for bMx1 and bIFIT2 and seems to downregulate bIFIT1,3,5.

other than for bMx1 for wt ultracentrifugation purified bRSV 24 HPI MOI 1. I have one experiment where bRSV MOI 1 24 HPI downregulates all genes tested but it might be a technical error. 




normal unequal - bmx1 brsvn

normal euqal - bi1 bi2 bi3 bi5


\begin{figure}
    \centering
    \includegraphics[width=1\linewidth]{07. Chapter 2/Figs/02. Induction/05. mdbk_brsv_moi1_dsh.pdf}
    \caption[MDBK \textit{bIFIT} Response to WT and \(\Delta\)SH bRSV Infection.]{\textbf{MDBK \textit{bIFIT} Response to WT and \(\Delta\)SH bRSV Infection.} (a) \textit{bIFIT1}, (b) \textit{bIFIT2}, (c) \textit{bIFIT3}, (d) \textit{bIFIT5}, (e) \textit{bMx1}, and (f) \textit{bRSV N} gene expression levels were assessed using quantitative real-time PCR (qPCR) in MDBK cell line following infection with WT or \(\Delta\)SH bRSV at MOI 1 for 24 hours post-infection. Relative expression values are normalized to standardized mock-treated samples. Median values are represented by red circles. The black dotted line represents mock expression levels, while the red dotted lines indicate biologically significant induction thresholds. Numeric values indicate the p-values compared to mock-treated samples.}
    \label{fig:MDBK responses to dSH}
\end{figure}



mdbk dsh

wt bRSV along with MOI 1 dSH do not up or downregulate any genes tested in a biologically significant way.

normal unequal - bi1 bi2 bi3 bi5 brsvn

normal euqal - bmx1


\begin{figure}
    \centering
    \includegraphics[width=1\linewidth]{07. Chapter 2/Figs/02. Induction/09. bt_brsv.pdf}
    \caption[BT \textit{bIFIT} Response to WT and \(\Delta\)SH bRSV Infection.]{\textbf{BT \textit{bIFIT} Response to WT and \(\Delta\)SH bRSV Infection.} (a) \textit{bIFIT1}, (b) \textit{bIFIT2}, (c) \textit{bIFIT3}, (d) \textit{bIFIT5}, (e) \textit{bMx1}, and (f) \textit{bRSV N} gene expression levels were assessed using quantitative real-time PCR (qPCR) in BT cell line following infection with WT or \(\Delta\)SH bRSV at MOI 1 for 24 hours post-infection. Relative expression values are normalized to standardized mock-treated samples. Median values are represented by red circles. The black dotted line represents mock expression levels, while the red dotted lines indicate biologically significant induction thresholds. Numeric values indicate the p-values compared to mock-treated samples.}
    \label{fig:BT responses to bRSV}
\end{figure}

bt dsh and wt brsv

Infections with wt bRSV and dSH bRSV at the same MOI and HPI (1 and 24) cause slight downregulation in all genes tested.

normal unequal - bi1 brsvn

normal euqal - bi2 bi3 bi5 bmx1

\begin{figure}
    \centering
    \includegraphics[width=1\linewidth]{07. Chapter 2/Figs/02. Induction/06. mdbk_brsv_low_moi.pdf}
    \caption[MDBK \textit{bIFIT} Response to Low MOI bRSV Infections.]{\textbf{MDBK \textit{bIFIT} Response to Low MOI bRSV Infections.} (a) \textit{bIFIT1}, (b) \textit{bIFIT2}, (c) \textit{bIFIT3}, (d) \textit{bIFIT5}, (e) \textit{bMx1}, and (f) \textit{bRSV N} gene expression levels were assessed using quantitative real-time PCR (qPCR) in MDBK cell line following infection with WT or \(\Delta\)NS1, \(\Delta\)NS2, and \(\Delta\)NS1/2 bRSV at MOIs of 0.001 for 24 hours post-infection. Relative expression values are normalized to standardized mock-treated samples. Median values are represented by red circles. The black dotted line represents mock expression levels, while the red dotted lines indicate biologically significant induction thresholds. Numeric values indicate the p-values compared to mock-treated samples.}
    \label{fig:MDBK responses to low MOI mutant bRSV}
\end{figure}

low moi dnss mdbk

Very low MOI (0.001) wt bRSV along with MOI 1 dSH and very low MOI dNS1, dSN2 and dNS1/2 bRSV do not up or downregulate any genes tested in a biologically significant way.

normal unequal - bi5 bmx1 brsvn

normal euqal - bi1 bi2 bi3 

\subsection{Bovine \textit{IFITs} Responses to hRSV} \label{subsec:Bovine IFITs Responses to hRSV}

Intro into hrsv infections


\begin{figure}
    \centering
    \includegraphics[width=1\linewidth]{07. Chapter 2/Figs/02. Induction/07. mdbk_hrsv.pdf}
    \caption[MDBK \textit{bIFIT} Response to Crude-Extracted and Ultra-Purified hRSV Infection.]{\textbf{MDBK \textit{bIFIT} Response to Crude-Extracted and Ultra-Purified hRSV Infection.} (a) \textit{bIFIT1}, (b) \textit{bIFIT2}, (c) \textit{bIFIT3}, (d) \textit{bIFIT5}, (e) \textit{bMx1}, and (f) \textit{hRSV N} gene expression levels were assessed using quantitative real-time PCR (qPCR) in MDBK cell line following infection with crude-extraccted and ultra-purified hRSV at MOI 1 for 24 hours post-infection. Relative expression values are normalized to standardized mock-treated samples. Median values are represented by red circles. The black dotted line represents mock expression levels, while the red dotted lines indicate biologically significant induction thresholds. Numeric values indicate the p-values compared to mock-treated samples.}
    \label{fig:bIFIT responses to hRSV infection in MDBK}
\end{figure}

mdbk hrsv

Data show that ultracentrifugation purified hRSV causes no response in terms of bIFIT induction. Infection with normally purified virus does not cause induction either but hints at downregulation actually.

normal unequal - bi5 bmx1 brsvn

normal euqal - bi1 bi2 bi3 


\begin{figure}
    \centering
    \includegraphics[width=1\linewidth]{07. Chapter 2/Figs/02. Induction/10. bt_hrsv.pdf}
    \caption[BT \textit{bIFIT} Response to Crude-Extracted and Ultra-Purified hRSV Infection.]{\textbf{BT \textit{bIFIT} Response to Crude-Extracted and Ultra-Purified hRSV Infection.} (a) \textit{bIFIT1}, (b) \textit{bIFIT2}, (c) \textit{bIFIT3}, (d) \textit{bIFIT5}, (e) \textit{bMx1}, and (f) \textit{hRSV N} gene expression levels were assessed using quantitative real-time PCR (qPCR) in BT cell line following infection with crude-extraccted and ultra-purified hRSV at MOI 1 for 24 hours post-infection. Relative expression values are normalized to standardized mock-treated samples. Median values are represented by red circles. The black dotted line represents mock expression levels, while the red dotted lines indicate biologically significant induction thresholds. Numeric values indicate the p-values compared to mock-treated samples.}
    \label{fig:Bt responses to hRSV}
\end{figure}

bt hrsv

Ultracentrifugation purified hRSV causes downregulation in IFITs but not bMx1 (where it causes no change), while infection with normally extracted hRSV cause no change in none of the genes tested. 

normal unequal - bi3 bi5 bmx1 hrsvn

normal euqal - bi1 bi2



\section{Conclusions} \label{sec:Conclusions Chapter2}
Recap bovine

\cite{Johnston2019ExperimentalResponse.} bovine respiratory syncytial virus in dairy calves: bronchial lymph node - all ifits upregulated
\cite{Johnston2021MessengerCalves} Messenger RNA biomarkers of Bovine Respiratory Syncytial Virus infection in the whole blood of dairy calves - ifit2 and ifit5


\nomenclature[z-PS]{PS}{Primer Set}
\nomenclature[z-cDNA]{cDNA}{Complementary DNA}
\nomenclature[z-MDBK]{MDBK}{Madin-Darby Bovine Kidney}
\nomenclature[z-ISG]{ISG}{Interferon-Stimulated Gene}
\nomenclature[z-TLR4]{TLR4}{Toll-like Receptor 4}
\nomenclature[z-BVDV]{BVDV}{Bovine Viral Diarrhea Virus}
\nomenclature[z-HPI]{HPI}{Hours Post-Infection}
\nomenclature[z-hRSV]{hRSV}{Human RSV}
\nomenclature[z-bRSV]{bRSV}{Bovine RSV}

% Words in text: 3574
% Words in headers: 55