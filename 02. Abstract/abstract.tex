\chapter{Abstract}
The interferon-induced proteins with tetratricopeptide repeats (IFITs) are early-response antiviral proteins conserved across vertebrates. Following stimulation, they are some of the most highly induced genes, although the patterns of induction kinetics of individual IFIT proteins being species-, cell type-, tissue-, virus-, and inducer-specific. Most mammals, including Homo sapiens and Bos taurus, express four IFIT proteins—IFIT1, IFIT2, IFIT3, and IFIT5. These proteins exert their antiviral action through mechanisms well described in human systems, involving the potentiation of innate immune signalling cascades (IFIT1, IFIT3, and IFIT5), promotion of apoptosis (IFIT2), inhibition of cell cycle progression (IFIT3), and detection of non-self single-stranded RNA (IFIT1 and IFIT5). Human IFITs were shown to be the restrictors of several RNA viruses such as parainfluenza virus 3 and influenza A virus. Most importantly, IFITs, particularly IFIT1, IFIT2, and IFIT3, globally exhibit antiviral properties against respiratory syncytial virus (RSV), evidenced by the decreased viral mRNA production upon ectopic expression and an opposing effect upon IFIT gene silencing.
Human and bovine RSV stand as the predominant causes of lower respiratory tract infections, posing significant health risks to young calves, children under 5, the elderly, and immunocompromised individuals. RSV, an enveloped, single-stranded negative-sense RNA virus, is host-restricted in vivo. In infected cells, RSV forms membrane-less perinuclear cytoplasmic inclusion bodies (IBs), recognized as sites for viral RNA transcription and replication. These IBs also manipulate cellular components, either repurposing them for viral benefit (e.g., components of the eIF4F complex) or inhibiting their function (e.g., MAVS and MDA5).
Our study aimed to determine if the IFITs are induced during RSV infection and thus their reported ectopic inhibition of RSV is relevant in vivo. Additionally, we sought to unravel the nature of this inhibition, specifically by examining IFITs' interaction with RSV IBs. Lastly, we aimed to understand if this induction and subsequent inhibition are consistent between species by assessing bovine IFIT interaction with bovine RSV, as information on this aspect is lacking in the literature.
Our observations revealed the induction of human IFITs by both human and bovine RSV infections. Furthermore, we established that the induction by human RSV is dependent on viral replication and functional interferon signalling. Lastly, minimal induction of bovine IFITs was observed following infection with either bovine or human RSV.
Subsequently, our focus shifted to elucidating the mechanism of RSV fitness reduction by IFIT proteins. We assayed the interaction phenotypes of human and bovine IFITs during human and bovine RSV infection, revealing a phenotypically diverse set of interactions per IFIT. Importantly, these interactions were consistent between the species, ranging from intra-IB inclusion formation, colocalization with the IB boundary, diffusion throughout the cytoplasm and IB structure, to exclusion from these structures. Further analysis using pseudo-inclusion bodies (pIBs), IB-like structures that spontaneously emerge after the expression of RSV nucleoprotein and phosphoprotein, and overexpressed IFIT proteins during RSV infection showed consistent interaction of IFIT1, IFIT2, and IFIT5 with both pIBs and IBs. This suggests that the anti-RSV action of IFITs is mediated via interactions with these structures, potentially hindering RSV's ability to proceed with its physiological viral RNA transcription and replication.

