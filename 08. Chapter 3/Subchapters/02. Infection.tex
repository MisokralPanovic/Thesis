\subsection{Nascent Human and Bovine IFIT Localisation During RSV Infection} \label{subsec:Nascent Human and Bovine IFIT Localisation During RSV Infection}
\subsubsection{Phenotypic Diversity of Nascent IFIT1 Interaction with RSV Inclusion Bodies}
Initial experiment suggested hIFIT1 colocalises with the hRSV IBs in A549 cell line, however assessing all 281 observation we can see that hIFIT1 displays a range of phenotypic divesity with regards to interaction with these structures. As can be seen in Figure \ref{fig:Phenotypic Diversity of hIFIT1 Interactions with hRSV Inclusion Bodies in A549 Cell Line} panel (a), 30\% of interactions result in either full or partial exclusion from the structure or an even diffusion through the IB and the surroinding cytosol. Next most common phenotype is a inclusion inside the IB structure, occuring in circa 18\% of observation. Following phenotypes, orrucing in the same frequency of 9\%, are a phenotype of colocalisation with the edge of the inclusion body joined with exclusion from the area of the IB; and a edge exclusion phenotype, where IFIT1 was present equaly between the IB and the surrounding cytoplasm with the exeption of of the IB boundry, where its signal is decreased. Representative images of the phenotypes mentioed above can be seen in Figure \ref{fig:Representative Images of Phenotypic Diversity of hIFIT1 Interactions with hRSV Inclusion Bodies in A549 Cell Line}. We have also observed IFIT1 to dysplay edge exclusion with obvoius spots within the IB structure (Edge exclusion + IBAG) and also a colocalisation with phenotype but neither had a sufficiently high frequency of occurance where we can be certain that these are indeed relevant during infection and not just imaging artefacts. In Figure \ref{fig:Phenotypic Diversity of hIFIT1 Interactions with hRSV Inclusion Bodies in A549 Cell Line} panel (b), we can observe the measured areas of the IBs associated with phenotypes which had the frequency of accurance higher than 5\%. We can see that the two most common pheotypes, exclusion and diffusion, had median IB area sizes of 5 and 4.4 \(\mu m^2\), values either identical or very similar to the aggregated median IB area of all the observations in A549 cell line. Both phenotypes also encompass a range of IB sizes, from sub 1 \(\mu m^2\) until supra 20 \(\mu m^2\) IB areas. On the other hand, the other phenotyes observed are more prevalent in larger inclusion bodies. In more detail, inclusion phenotype was more prevalent in supra 3 \(\mu m^2\) inclusion bodies, with the median size of 8; colocalisation accompanied by exclusion phenotype median IB size was 12 \(\mu m^2\), although some IBs with sizes between 0.8 to 5 \(\mu m^2\) showed this phenotype as well; and the edge exlusion phenotype was in IBs with median size of 7 \(\mu m^2\), although these IB were clustered in two clusters with median values of approximately 5 and 10 \(\mu m^2\) respectively. Our data indicates that while in majrity of cases IFIT1 is either excluded or diffused throught the IB structures, regardless ofn the IB size and thus maturity, we can see a potential interaction with more mature IBs of sizes above 5 \(\mu m^2\), which in the literature, coinsides to the IB sizes which have IBAGs present (\cite{Rincheval2017FunctionalVirus}). 

\begin{figure}
    \begin{subfigure}{0.495\textwidth}
        \caption{}
        \includegraphics[width=1\linewidth]{08. Chapter 3/Figs/02. Infection/01. IFIT1/01. bar_i1_a549.pdf} 
    \end{subfigure}
    \begin{subfigure}{0.495\textwidth}
        \caption{}
        \includegraphics[width=1\linewidth]{08. Chapter 3/Figs/02. Infection/01. IFIT1/02. box_i1_a549.pdf}
    \end{subfigure}
    \caption[Phenotypic Diversity of hIFIT1 Interactions with hRSV Inclusion Bodies in A549 Cell Line.]{\textbf{Phenotypic Diversity of hIFIT1 Interactions with hRSV Inclusion Bodies in A549 Cell Line.} A549 cells were infected with human RSV at MOI 1 and fixed 24 HPI. Cells were double-labeled with with anti-RSV N and anti-IFIT1 antibodies and imaged on confocal microscope. Panel (a) shows percentual proportions of observed phenotypes between hRSV inclusion bodies and hIFIT1 (281 observations), with the red dotted line denoting the 5\% threshold, marking phenotypes considered relevant above this limit. Panel (b) shows the IB area in \(\mu m^2\) per observed relevant phenotype.}
    \label{fig:Phenotypic Diversity of hIFIT1 Interactions with hRSV Inclusion Bodies in A549 Cell Line}
\end{figure}

\begin{figure}
    \centering
    \includegraphics[width=1\linewidth]{08. Chapter 3/Figs/02. Infection/01. IFIT1/03. a549 i1.pdf}
    \caption[Representative Images of Phenotypic Diversity of hIFIT1 Interactions with hRSV Inclusion Bodies in A549 Cell Line.]{\textbf{Representative Images of Phenotypic Diversity of hIFIT1 Interactions with hRSV Inclusion Bodies in A549 Cell Line.} A549 cells were infected with hRSV at MOI 1 and fixed at 24 HPI. Cellular nuclei were stained with DAPI (yellow), and cells were double-labeled with anti-RSV N (cyan) and anti-IFIT1 (magenta) antibodies. This figure showcases representative examples of relevant phenotypes in the interaction between hIFIT1 and hRSV inclusion bodies. These phenotypes are presented in descending order based on their percentage proportions. The scale bar indicates 2 \(\mu m\).}
    \label{fig:Representative Images of Phenotypic Diversity of hIFIT1 Interactions with hRSV Inclusion Bodies in A549 Cell Line}
\end{figure}

We set to validate the previous data in the BEAS2B cell line. Figure \ref{fig:Phenotypic Diversity of hIFIT1 Interactions with hRSV Inclusion Bodies in BEAS2B Cell Line} shows the observed IFIT1/IB interaction phenotypes, their occurance (panel a) and the underlying IB sizes (panel b), while Figure \ref{fig:Representative Images of Phenotypic Diversity of hIFIT1 Interactions with hRSV Inclusion Bodies in BEAS2B Cell Line} shows the respresentative images of phenotypes with the occurance of above 5\%. The majority of observations showed IFIT1 to be either partially or fully excluded from the hRSV inclusion bodies (circa 85\% of observations), while 8\% of phenotypes were diffusion and 4\% displayed colocalisation cojoined with exclusion. The size range of IB from which IFIT1 was exclude mimic the aggregate distribution of all IBs detected within BEAS2B cells, having the equal median size value of 3 \(\mu m^2\) and spread from sub 1 \(\mu m^2\) IBs to supra 10 \(\mu m^2\) ones. The second most common phenotype and the only other that surpassed 5\% of total accurance was diffusion phenitipe, which was observed only in smaller IBs, with the median size of 0.5 \(\mu m^2\).

WRITE SOME CONCLUSION

\begin{figure}
    \begin{subfigure}{0.495\textwidth}
        \caption{}
        \includegraphics[width=1\linewidth]{08. Chapter 3/Figs/02. Infection/01. IFIT1/04. bar_i1_beas2b.pdf} 
    \end{subfigure}
    \begin{subfigure}{0.495\textwidth}
        \caption{}
        \includegraphics[width=1\linewidth]{08. Chapter 3/Figs/02. Infection/01. IFIT1/05. box_i1_beas2b.pdf}
    \end{subfigure}
    \caption[Phenotypic Diversity of hIFIT1 Interactions with hRSV Inclusion Bodies in BEAS2B Cell Line.]{\textbf{Phenotypic Diversity of hIFIT1 Interactions with hRSV Inclusion Bodies in BEAS2B Cell Line.} BEAS2B cells were infected with human RSV at MOI 1 and fixed 24 HPI. Cells were double-labeled with with anti-RSV N and anti-IFIT1 antibodies and imaged on confocal microscope. Panel (a) shows percentual proportions of observed phenotypes between hRSV inclusion bodies and hIFIT1 (281 observations), with the red dotted line denoting the 5\% threshold, marking phenotypes considered relevant above this limit. Panel (b) shows the IB area in \(\mu m^2\) per observed relevant phenotype.}
    \label{fig:Phenotypic Diversity of hIFIT1 Interactions with hRSV Inclusion Bodies in BEAS2B Cell Line}
\end{figure}

\begin{figure}
    \centering
    \includegraphics[width=1\linewidth]{08. Chapter 3/Figs/02. Infection/01. IFIT1/06. beas2b i1.pdf}
    \caption[Representative Images of Phenotypic Diversity of hIFIT1 Interactions with hRSV Inclusion Bodies in BEAS2B Cell Line.]{\textbf{Representative Images of Phenotypic Diversity of hIFIT1 Interactions with hRSV Inclusion Bodies in BEAS2B Cell Line.} BEAS2B cells were infected with hRSV at MOI 1 and fixed at 24 HPI. Cellular nuclei were stained with DAPI (yellow), and cells were double-labeled with anti-RSV N (cyan) and anti-IFIT1 (magenta) antibodies. This figure showcases representative examples of relevant phenotypes in the interaction between hIFIT1 and hRSV inclusion bodies. These phenotypes are presented in descending order based on their percentage proportions. The scale bar indicates 2 \(\mu m\).}
    \label{fig:Representative Images of Phenotypic Diversity of hIFIT1 Interactions with hRSV Inclusion Bodies in BEAS2B Cell Line}
\end{figure}

Lastly ...



66 excl, 23 incl, 5 diff

2, 1, 1.5

\begin{figure}
    \begin{subfigure}{0.495\textwidth}
        \caption{}
        \includegraphics[width=1\linewidth]{08. Chapter 3/Figs/02. Infection/01. IFIT1/07. bar_i1_mdbk.pdf} 
    \end{subfigure}
    \begin{subfigure}{0.495\textwidth}
        \caption{}
        \includegraphics[width=1\linewidth]{08. Chapter 3/Figs/02. Infection/01. IFIT1/08. box_i1_mdbk.pdf}
    \end{subfigure}
    \caption[Phenotypic Diversity of bIFIT1 Interactions with bRSV Inclusion Bodies in MDBK Cell Line.]{\textbf{Phenotypic Diversity of bIFIT1 Interactions with bRSV Inclusion Bodies in MDBK Cell Line.} MDBK cells were infected with bovine RSV at MOI 1 and fixed 24 HPI. Cells were double-labeled with with anti-RSV N and anti-IFIT1 antibodies and imaged on confocal microscope. Panel (a) shows percentual proportions of observed phenotypes between bRSV inclusion bodies and bIFIT1 (117 observations), with the red dotted line denoting the 5\% threshold, marking phenotypes considered relevant above this limit. Panel (b) shows the IB area in \(\mu m^2\) per observed relevant phenotype.}
    \label{fig:Phenotypic Diversity of bIFIT1 Interactions with bRSV Inclusion Bodies in MDBK Cell Line}
\end{figure}

\begin{figure}
    \centering
    \includegraphics[width=1\linewidth]{08. Chapter 3/Figs/02. Infection/01. IFIT1/09. mdbk i1.pdf}
    \caption[Representative Images of Phenotypic Diversity of bIFIT1 Interactions with bRSV Inclusion Bodies in MDBK Cell Line.]{\textbf{Representative Images of Phenotypic Diversity of bIFIT1 Interactions with bRSV Inclusion Bodies in MDBK Cell Line.} MDBK cells were infected with bRSV at MOI 1 and fixed at 24 HPI. Cellular nuclei were stained with DAPI (yellow), and cells were double-labeled with anti-RSV N (cyan) and anti-IFIT1 (magenta) antibodies. This figure showcases representative examples of relevant phenotypes in the interaction between bIFIT1 and bRSV inclusion bodies. These phenotypes are presented in descending order based on their percentage proportions. The scale bar indicates 2 \(\mu m\).}
    \label{fig:Representative Images of Phenotypic Diversity of bIFIT1 Interactions with bRSV Inclusion Bodies in MDBK Cell Line}
\end{figure}

\subsubsection{Phenotypic Diversity of Nascent IFIT2 Interaction with RSV Inclusion Bodies}

\lipsum[1-5]

\begin{figure}
    \begin{subfigure}{0.495\textwidth}
        \caption{}
        \includegraphics[width=1\linewidth]{08. Chapter 3/Figs/02. Infection/02. IFIT2/01. IFIT2A/01. bar_i2a_a549-n.pdf}
    \end{subfigure}
    \begin{subfigure}{0.495\textwidth}
        \caption{}
        \includegraphics[width=1\linewidth]{08. Chapter 3/Figs/02. Infection/02. IFIT2/01. IFIT2A/02. box_i2a_a549-n.pdf}
    \end{subfigure}
    \caption[Phenotypic Diversity of hIFIT2 Interactions with Nucleoprotein-Stained hRSV Inclusion Bodies, Detected by IFIT2A Antibody in A549 Cell Line.]{\textbf{Phenotypic Diversity of hIFIT2 Interactions with Nucleoprotein-Stained hRSV Inclusion Bodies, Detected by IFIT2A Antibody in A549 Cell Line.} A549 cells were infected with human RSV at MOI 1 and fixed 24 HPI. Cells were labeled with anti-RSV N and anti-IFIT2A antibodies and imaged on confocal microscope. Panel (a) shows percentual proportions of observed phenotypes between hRSV inclusion bodies and hIFIT2, detected by IFIT2A antibody (47 observations), with the red dotted line denoting the 5\% threshold, marking phenotypes considered relevant above this limit. Panel (b) shows the IB area in \(\mu m^2\) per observed relevant phenotype.}
    \label{fig:Phenotypic Diversity of hIFIT2 Interactions with Nucleoprotein-Stained hRSV Inclusion Bodies, Detected by IFIT2A Antibody in A549 Cell Line}
\end{figure}

\begin{figure}
    \centering
    \includegraphics[width=1\linewidth]{08. Chapter 3/Figs/02. Infection/02. IFIT2/01. IFIT2A/03. i2a a549 hrsv n.pdf}
    \caption[Representative Images of Phenotypic Diversity of hIFIT2 Interactions with Nucleoprotein-Stained hRSV Inclusion Bodies, Detected by IFIT2A Antibody in A549 Cell Line.]{\textbf{Representative Images of Phenotypic Diversity of hIFIT2 Interactions with Nucleoprotein-Stained hRSV Inclusion Bodies, Detected by IFIT2A Antibody in A549 Cell Line.} A549 cells were infected with hRSV at MOI 1 and fixed at 24 HPI. Cellular nuclei were stained with DAPI (yellow), and cells were double-labeled with anti-RSV N (cyan) and anti-IFIT2A (magenta) antibodies. This figure showcases representative examples of relevant phenotypes in the interaction between hIFIT2, detected by IFIT2A antibody, and hRSV inclusion bodies. These phenotypes are presented in descending order based on their percentage proportions. The scale bar indicates 2 \(\mu m\).}
    \label{fig:Representative Images of Phenotypic Diversity of hIFIT2 Interactions with Nucleoprotein-Stained hRSV Inclusion Bodies, Detected by IFIT2A Antibody in A549 Cell Line}
\end{figure}

60 30 8

3 12 2.6

Nascent human IFIT2 colocalises with the ring structure (outlined by RSV P staining) and to the inner edge of the IB.

\begin{figure}
    \begin{subfigure}{0.495\textwidth}
        \caption{}
        \includegraphics[width=1\linewidth]{08. Chapter 3/Figs/02. Infection/02. IFIT2/01. IFIT2A/04. bar_i2a_a549-p.pdf} 
    \end{subfigure}
    \begin{subfigure}{0.495\textwidth}
        \caption{}
        \includegraphics[width=1\linewidth]{08. Chapter 3/Figs/02. Infection/02. IFIT2/01. IFIT2A/05. box_i2a_a549-p.pdf}
    \end{subfigure}
    \caption[Phenotypic Diversity of hIFIT2 Interactions with Phosphoprotein-Stained hRSV Inclusion Bodies, Detected by IFIT2A Antibody in A549 Cell Line.]{\textbf{Phenotypic Diversity of hIFIT2 Interactions with Phosphoprotein-Stained hRSV Inclusion Bodies, Detected by IFIT2A Antibody in A549 Cell Line.} A549 cells were infected with human RSV at MOI 1 and fixed 24 HPI. Cells were labeled with anti-RSV P and anti-IFIT2A antibodies and imaged on confocal microscope. Panel (a) shows percentual proportions of observed phenotypes between hRSV inclusion bodies and hIFIT2, detected by IFIT2A antibody (48 observations), with the red dotted line denoting the 5\% threshold, marking phenotypes considered relevant above this limit. Panel (b) shows the IB area in \(\mu m^2\) per observed relevant phenotype.}
    \label{fig:Phenotypic Diversity of hIFIT2 Interactions with Phosphoprotein-Stained hRSV Inclusion Bodies, Detected by IFIT2A Antibody in A549 Cell Line}
\end{figure}

\begin{figure}
    \centering
    \includegraphics[width=1\linewidth]{08. Chapter 3/Figs/02. Infection/02. IFIT2/01. IFIT2A/06. i2a a549 hrsv p.pdf} 
    \caption[Representative Images of Phenotypic Diversity of hIFIT2 Interactions with Phosphoprotein-Stained hRSV Inclusion Bodies, Detected by IFIT2A Antibody in A549 Cell Line.]{\textbf{Representative Images of Phenotypic Diversity of hIFIT2 Interactions with Phosphoprotein-Stained hRSV Inclusion Bodies, Detected by IFIT2A Antibody in A549 Cell Line.} A549 cells were infected with hRSV at MOI 1 and fixed at 24 HPI. Cellular nuclei were stained with DAPI (yellow), and cells were double-labeled with anti-RSV P (cyan) and anti-IFIT2A (magenta) antibodies. This figure showcases representative examples of relevant phenotypes in the interaction between hIFIT2, detected by IFIT2A antibody, and hRSV inclusion bodies. These phenotypes are presented in descending order based on their percentage proportions. The scale bar indicates 2 \(\mu m\).}
    \label{fig:Representative Images of Phenotypic Diversity of hIFIT2 Interactions with Phosphoprotein-Stained hRSV Inclusion Bodies, Detected by IFIT2A Antibody in A549 Cell Line}
\end{figure}

82 18

2.5 7.3

With regards of colocalization with human RSV M2/1 protein, human IFIT2 seems to either form inclusion, which has a signal decrease towards the middle of the IB structure (top panel), or seems to strongly colocalise with the ring structure highlighted by M2/1 staining (bottom 2 panels; there also seems to be IFIT2 signal concentration on the inner edge of the IB structure).

\begin{figure}
    \begin{subfigure}{0.495\textwidth}
        \caption{}
        \includegraphics[width=1\linewidth]{08. Chapter 3/Figs/02. Infection/02. IFIT2/01. IFIT2A/07. bar_i2a_a549-m21.pdf} 
    \end{subfigure}
    \begin{subfigure}{0.495\textwidth}
        \caption{}
        \includegraphics[width=1\linewidth]{08. Chapter 3/Figs/02. Infection/02. IFIT2/01. IFIT2A/08. box_i2a_a549-m21.pdf}
    \end{subfigure}
    \caption[Phenotypic Diversity of hIFIT2 Interactions with M2/1-Stained hRSV Inclusion Bodies, Detected by IFIT2A Antibody in A549 Cell Line.]{\textbf{Phenotypic Diversity of hIFIT2 Interactions with M2/1-Stained hRSV Inclusion Bodies, Detected by IFIT2A Antibody in A549 Cell Line.} A549 cells were infected with human RSV at MOI 1 and fixed 24 HPI. Cells were labeled with anti-RSV M2/1 and anti-IFIT2A antibodies and imaged on confocal microscope. Panel (a) shows percentual proportions of observed phenotypes between hRSV inclusion bodies and hIFIT2, detected by IFIT2A antibody (69 observations), with the red dotted line denoting the 5\% threshold, marking phenotypes considered relevant above this limit. Panel (b) shows the IB area in \(\mu m^2\) per observed relevant phenotype.}
    \label{fig:Phenotypic Diversity of hIFIT2 Interactions with M2/1-Stained hRSV Inclusion Bodies, Detected by IFIT2A Antibody in A549 Cell Line}
\end{figure}

\begin{figure}
    \centering
    \includegraphics[width=1\linewidth]{08. Chapter 3/Figs/02. Infection/02. IFIT2/01. IFIT2A/09. i2a a549 hrsv m21.pdf} 
    \caption[Representative Images of Phenotypic Diversity of hIFIT2 Interactions with M2/1-Stained hRSV Inclusion Bodies, Detected by IFIT2A Antibody in A549 Cell Line.]{\textbf{Representative Images of Phenotypic Diversity of hIFIT2 Interactions with M2/1-Stained hRSV Inclusion Bodies, Detected by IFIT2A Antibody in A549 Cell Line.} A549 cells were infected with hRSV at MOI 1 and fixed at 24 HPI. Cellular nuclei were stained with DAPI (yellow), and cells were double-labeled with anti-RSV M2/1 (cyan) and anti-IFIT2A (magenta) antibodies. This figure showcases representative examples of relevant phenotypes in the interaction between hIFIT2, detected by IFIT2A antibody, and hRSV inclusion bodies. These phenotypes are presented in descending order based on their percentage proportions. The scale bar indicates 2 \(\mu m\).}
    \label{fig:Representative Images of Phenotypic Diversity of hIFIT2 Interactions with M2/1-Stained hRSV Inclusion Bodies, Detected by IFIT2A Antibody in A549 Cell Line}
\end{figure}

81 19

1 10

\begin{figure}
    \begin{subfigure}{0.495\textwidth}
        \caption{}
        \includegraphics[width=1\linewidth]{08. Chapter 3/Figs/02. Infection/02. IFIT2/02. IFIT2B/01. bar_i2b_a549-n.pdf}
    \end{subfigure}
    \begin{subfigure}{0.495\textwidth}
        \caption{}
        \includegraphics[width=1\linewidth]{08. Chapter 3/Figs/02. Infection/02. IFIT2/02. IFIT2B/02. box_i2b_a549-n.pdf}
    \end{subfigure}
    \caption[Phenotypic Diversity of hIFIT2 Interactions with Nucleoprotein-Stained hRSV Inclusion Bodies, Detected by IFIT2B Antibody in A549 Cell Line.]{\textbf{Phenotypic Diversity of hIFIT2 Interactions with Nucleoprotein-Stained hRSV Inclusion Bodies, Detected by IFIT2B Antibody in A549 Cell Line.} A549 cells were infected with human RSV at MOI 1 and fixed 24 HPI. Cells were labeled with anti-RSV N and anti-IFIT2B antibodies and imaged on confocal microscope. Panel (a) shows percentual proportions of observed phenotypes between hRSV inclusion bodies and hIFIT2, detected by IFIT2B antibody (56 observations), with the red dotted line denoting the 5\% threshold, marking phenotypes considered relevant above this limit. Panel (b) shows the IB area in \(\mu m^2\) per observed relevant phenotype.}
    \label{fig:Phenotypic Diversity of hIFIT2 Interactions with Nucleoprotein-Stained hRSV Inclusion Bodies, Detected by IFIT2B Antibody in A549 Cell Line}
\end{figure}

\begin{figure}
    \centering
    \includegraphics[width=1\linewidth]{08. Chapter 3/Figs/02. Infection/02. IFIT2/02. IFIT2B/03. i2b a549 hrsv n.pdf} 
    \caption[Representative Images of Phenotypic Diversity of hIFIT2 Interactions with Nucleoprotein-Stained hRSV Inclusion Bodies, Detected by IFIT2B Antibody in A549 Cell Line.]{\textbf{Representative Images of Phenotypic Diversity of hIFIT2 Interactions with Nucleoprotein-Stained hRSV Inclusion Bodies, Detected by IFIT2B Antibody in A549 Cell Line.} A549 cells were infected with hRSV at MOI 1 and fixed at 24 HPI. Cellular nuclei were stained with DAPI (yellow), and cells were double-labeled with anti-RSV N (cyan) and anti-IFIT2B (magenta) antibodies. This figure showcases representative examples of relevant phenotypes in the interaction between hIFIT2, detected by IFIT2B antibody, and hRSV inclusion bodies. These phenotypes are presented in descending order based on their percentage proportions. The scale bar indicates 2 \(\mu m\).}
    \label{fig:Representative Images of Phenotypic Diversity of hIFIT2 Interactions with Nucleoprotein-Stained hRSV Inclusion Bodies, Detected by IFIT2B Antibody in A549 Cell Line}
\end{figure}

96

2.1

Endogenous human IFIT2 is either partially excluded (top panel; decrease of intra IB signal compared to cytoplasmic signal) or completely excluded (bottom panel) from the human IB structure.

\begin{figure}
    \begin{subfigure}{0.495\textwidth}
        \caption{}
        \includegraphics[width=1\linewidth]{08. Chapter 3/Figs/02. Infection/02. IFIT2/02. IFIT2B/04. bar_i2b_a549-p.pdf} 
    \end{subfigure}
    \begin{subfigure}{0.495\textwidth}
        \caption{}
        \includegraphics[width=1\linewidth]{08. Chapter 3/Figs/02. Infection/02. IFIT2/02. IFIT2B/05. box_i2b_a549-p.pdf}
    \end{subfigure}
    \caption[Phenotypic Diversity of hIFIT2 Interactions with Phosphoprotein-Stained hRSV Inclusion Bodies, Detected by IFIT2B Antibody in A549 Cell Line.]{\textbf{Phenotypic Diversity of hIFIT2 Interactions with Phosphoprotein-Stained hRSV Inclusion Bodies, Detected by IFIT2B Antibody in A549 Cell Line.} A549 cells were infected with human RSV at MOI 1 and fixed 24 HPI. Cells were labeled with anti-RSV P and anti-IFIT2B antibodies and imaged on confocal microscope. Panel (a) shows percentual proportions of observed phenotypes between hRSV inclusion bodies and hIFIT2, detected by IFIT2B antibody (27 observations), with the red dotted line denoting the 5\% threshold, marking phenotypes considered relevant above this limit. Panel (b) shows the IB area in \(\mu m^2\) per observed relevant phenotype.}
    \label{fig:Phenotypic Diversity of hIFIT2 Interactions with Phosphoprotein-Stained hRSV Inclusion Bodies, Detected by IFIT2B Antibody in A549 Cell Line}
\end{figure}

\begin{figure}
    \centering
    \includegraphics[width=1\linewidth]{08. Chapter 3/Figs/02. Infection/02. IFIT2/02. IFIT2B/06. i2b a549 hrsv p.pdf} 
    \caption[Representative Images of Phenotypic Diversity of hIFIT2 Interactions with Phosphoprotein-Stained hRSV Inclusion Bodies, Detected by IFIT2B Antibody in A549 Cell Line.]{\textbf{Representative Images of Phenotypic Diversity of hIFIT2 Interactions with Phosphoprotein-Stained hRSV Inclusion Bodies, Detected by IFIT2B Antibody in A549 Cell Line.} A549 cells were infected with hRSV at MOI 1 and fixed at 24 HPI. Cellular nuclei were stained with DAPI (yellow), and cells were double-labeled with anti-RSV P (cyan) and anti-IFIT2B (magenta) antibodies. This figure showcases representative examples of relevant phenotypes in the interaction between hIFIT2, detected by IFIT2B antibody, and hRSV inclusion bodies. These phenotypes are presented in descending order based on their percentage proportions. The scale bar indicates 2 \(\mu m\).}
    \label{fig:Representative Images of Phenotypic Diversity of hIFIT2 Interactions with Phosphoprotein-Stained hRSV Inclusion Bodies, Detected by IFIT2B Antibody in A549 Cell Line}
\end{figure}

78 22

4.3 1.3

We observe similar pattern of staining to what was observed with N stained human IBs. IFIT2 signal is either partially or totally excluded from the IB structure.

\begin{figure}
    \begin{subfigure}{0.495\textwidth}
        \caption{}
        \includegraphics[width=1\linewidth]{08. Chapter 3/Figs/02. Infection/02. IFIT2/02. IFIT2B/07. bar_i2b_a549-m21.pdf} 
    \end{subfigure}
    \begin{subfigure}{0.495\textwidth}
        \caption{}
        \includegraphics[width=1\linewidth]{08. Chapter 3/Figs/02. Infection/02. IFIT2/02. IFIT2B/08. box_i2b_a549-m21.pdf}
    \end{subfigure}
    \caption[Phenotypic Diversity of hIFIT2 Interactions with M2/1-Stained hRSV Inclusion Bodies, Detected by IFIT2B Antibody in A549 Cell Line.]{\textbf{Phenotypic Diversity of hIFIT2 Interactions with M2/1-Stained hRSV Inclusion Bodies, Detected by IFIT2B Antibody in A549 Cell Line.} A549 cells were infected with human RSV at MOI 1 and fixed 24 HPI. Cells were labeled with anti-RSV M2/1 and anti-IFIT2B antibodies and imaged on confocal microscope. Panel (a) shows percentual proportions of observed phenotypes between hRSV inclusion bodies and hIFIT2, detected by IFIT2B antibody (31 observations), with the red dotted line denoting the 5\% threshold, marking phenotypes considered relevant above this limit. Panel (b) shows the IB area in \(\mu m^2\) per observed relevant phenotype.}
    \label{fig:Phenotypic Diversity of hIFIT2 Interactions with M2/1-Stained hRSV Inclusion Bodies, Detected by IFIT2B Antibody in A549 Cell Line}
\end{figure}

\begin{figure}
    \centering
    \includegraphics[width=1\linewidth]{08. Chapter 3/Figs/02. Infection/02. IFIT2/02. IFIT2B/09. i2b a549 hrsv m21.pdf} 
    \caption[Representative Images of Phenotypic Diversity of hIFIT2 Interactions with M2/1-Stained hRSV Inclusion Bodies, Detected by IFIT2B Antibody in A549 Cell Line.]{\textbf{Representative Images of Phenotypic Diversity of hIFIT2 Interactions with M2/1-Stained hRSV Inclusion Bodies, Detected by IFIT2B Antibody in A549 Cell Line.} A549 cells were infected with hRSV at MOI 1 and fixed at 24 HPI. Cellular nuclei were stained with DAPI (yellow), and cells were double-labeled with anti-RSV M2/1 (cyan) and anti-IFIT2B (magenta) antibodies. This figure showcases representative examples of relevant phenotypes in the interaction between hIFIT2, detected by IFIT2B antibody, and hRSV inclusion bodies. These phenotypes are presented in descending order based on their percentage proportions. The scale bar indicates 2 \(\mu m\).}
    \label{fig:Representative Images of Phenotypic Diversity of hIFIT2 Interactions with M2/1-Stained hRSV Inclusion Bodies, Detected by IFIT2B Antibody in A549 Cell Line}
\end{figure}

71 21 6

2.1 8 15

Endogenous human IFIT2 seems to be excluded from hRSV IBs.

\begin{figure}
    \begin{subfigure}{0.495\textwidth}
        \caption{}
        \includegraphics[width=1\linewidth]{08. Chapter 3/Figs/02. Infection/02. IFIT2/01. IFIT2A/10. bar_i2a_beas2b.pdf} 
    \end{subfigure}
    \begin{subfigure}{0.495\textwidth}
        \caption{}
        \includegraphics[width=1\linewidth]{08. Chapter 3/Figs/02. Infection/02. IFIT2/01. IFIT2A/11. box_i2a_beas2b.pdf}
    \end{subfigure}
    \caption[Phenotypic Diversity of hIFIT2 Interactions with hRSV Inclusion Bodies, Detected by IFIT2A Antibody in BEAS2B Cell Line.]{\textbf{Phenotypic Diversity of hIFIT2 Interactions with hRSV Inclusion Bodies, Detected by IFIT2A Antibody in BEAS2B Cell Line.} BEAS2B cells were infected with human RSV at MOI 1 and fixed 24 HPI. Cells were labeled with anti-RSV N and anti-IFIT2A antibodies and imaged on confocal microscope. Panel (a) shows percentual proportions of observed phenotypes between hRSV inclusion bodies and hIFIT2, detected by IFIT2A antibody (99 observations), with the red dotted line denoting the 5\% threshold, marking phenotypes considered relevant above this limit. Panel (b) shows the IB area in \(\mu m^2\) per observed relevant phenotype.}
    \label{fig:Phenotypic Diversity of hIFIT2 Interactions with hRSV Inclusion Bodies, Detected by IFIT2A Antibody in BEAS2B Cell Line}
\end{figure}

\begin{figure}
    \centering
    \includegraphics[width=1\linewidth]{08. Chapter 3/Figs/02. Infection/02. IFIT2/01. IFIT2A/12. i2a beas2b.pdf} 
    \caption[Representative Images of Phenotypic Diversity of hIFIT2 Interactions with hRSV Inclusion Bodies, Detected by IFIT2A Antibody in BEAS2B Cell Line.]{\textbf{Representative Images of Phenotypic Diversity of hIFIT2 Interactions with hRSV Inclusion Bodies, Detected by IFIT2A Antibody in BEAS2B Cell Line.} BEAS2B cells were infected with hRSV at MOI 1 and fixed at 24 HPI. Cellular nuclei were stained with DAPI (yellow), and cells were double-labeled with anti-RSV N (cyan) and anti-IFIT2A (magenta) antibodies. This figure showcases representative examples of relevant phenotypes in the interaction between hIFIT2, detected by IFIT2A antibody, and hRSV inclusion bodies. These phenotypes are presented in descending order based on their percentage proportions. The scale bar indicates 2 \(\mu m\).}
    \label{fig:Representative Images of Phenotypic Diversity of hIFIT2 Interactions with hRSV Inclusion Bodies, Detected by IFIT2A Antibody in BEAS2B Cell Line}
\end{figure}

58 35 5

3 1.2 1.2


Nascent bovine IFIT2 colocalization with regards of N stained bRSV IBs seems to strongly associate with the ring of the structure.

\begin{figure}
    \begin{subfigure}{0.495\textwidth}
        \caption{}
        \includegraphics[width=1\linewidth]{08. Chapter 3/Figs/02. Infection/02. IFIT2/01. IFIT2A/13. bar_i2a_mdbk.pdf} 
    \end{subfigure}
    \begin{subfigure}{0.495\textwidth}
        \caption{}
        \includegraphics[width=1\linewidth]{08. Chapter 3/Figs/02. Infection/02. IFIT2/01. IFIT2A/14. box_i2a_mdbk.pdf}
    \end{subfigure}
    \caption[Phenotypic Diversity of hIFIT2 Interactions with bRSV Inclusion Bodies, Detected by IFIT2A Antibody in MDBK Cell Line.]{\textbf{Phenotypic Diversity of hIFIT2 Interactions with bRSV Inclusion Bodies, Detected by IFIT2A Antibody in MDBK Cell Line.} MDBK cells were infected with bovine RSV at MOI 1 and fixed 24 HPI. Cells were labeled with anti-RSV N and anti-IFIT2A antibodies and imaged on confocal microscope. Panel (a) shows percentual proportions of observed phenotypes between bRSV inclusion bodies and bIFIT2, detected by IFIT2A antibody (162 observations), with the red dotted line denoting the 5\% threshold, marking phenotypes considered relevant above this limit. Panel (b) shows the IB area in \(\mu m^2\) per observed relevant phenotype.}
    \label{fig:Phenotypic Diversity of hIFIT2 Interactions with bRSV Inclusion Bodies, Detected by IFIT2A Antibody in MDBK Cell Line}
\end{figure}

\begin{figure}
    \centering
    \includegraphics[width=1\linewidth]{08. Chapter 3/Figs/02. Infection/02. IFIT2/01. IFIT2A/15. i2a mdbk brsv.pdf} 
    \caption[Representative Images of Phenotypic Diversity of hIFIT2 Interactions with bRSV Inclusion Bodies, Detected by IFIT2A Antibody in MDBK Cell Line.]{\textbf{Representative Images of Phenotypic Diversity of hIFIT2 Interactions with bRSV Inclusion Bodies, Detected by IFIT2A Antibody in MDBK Cell Line.} MDBK cells were infected with bRSV at MOI 1 and fixed at 24 HPI. Cellular nuclei were stained with DAPI (yellow), and cells were double-labeled with anti-RSV N (cyan) and anti-IFIT2A (magenta) antibodies. This figure showcases representative examples of relevant phenotypes in the interaction between bIFIT2, detected by IFIT2A antibody, and bRSV inclusion bodies. These phenotypes are presented in descending order based on their percentage proportions. The scale bar indicates 2 \(\mu m\).}
    \label{fig:Representative Images of Phenotypic Diversity of hIFIT2 Interactions with bRSV Inclusion Bodies, Detected by IFIT2A Antibody in MDBK Cell Line}
\end{figure}

64 36

1 9

\begin{figure}
    \begin{subfigure}{0.495\textwidth}
        \caption{}
        \includegraphics[width=1\linewidth]{08. Chapter 3/Figs/02. Infection/02. IFIT2/02. IFIT2B/10. bar_i2b_mdbk.pdf} 
    \end{subfigure}
    \begin{subfigure}{0.495\textwidth}
        \caption{}
        \includegraphics[width=1\linewidth]{08. Chapter 3/Figs/02. Infection/02. IFIT2/02. IFIT2B/11. box_i2b_mdbk.pdf}
    \end{subfigure}
    \caption[Phenotypic Diversity of hIFIT2 Interactions with bRSV Inclusion Bodies, Detected by IFIT2B Antibody in MDBK Cell Line.]{\textbf{Phenotypic Diversity of hIFIT2 Interactions with bRSV Inclusion Bodies, Detected by IFIT2B Antibody in MDBK Cell Line.} MDBK cells were infected with bovine RSV at MOI 1 and fixed 24 HPI. Cells were labeled with anti-RSV N and anti-IFIT2B antibodies and imaged on confocal microscope. Panel (a) shows percentual proportions of observed phenotypes between bRSV inclusion bodies and bIFIT2, detected by IFIT2B antibody (66 observations), with the red dotted line denoting the 5\% threshold, marking phenotypes considered relevant above this limit. Panel (b) shows the IB area in \(\mu m^2\) per observed relevant phenotype.}
    \label{fig:Phenotypic Diversity of hIFIT2 Interactions with bRSV Inclusion Bodies, Detected by IFIT2B Antibody in MDBK Cell Line}
\end{figure}

\begin{figure}
    \centering
    \includegraphics[width=1\linewidth]{08. Chapter 3/Figs/02. Infection/02. IFIT2/02. IFIT2B/12. i2b mdbk brsv.pdf} 
    \caption[Representative Images of Phenotypic Diversity of hIFIT2 Interactions with bRSV Inclusion Bodies, Detected by IFIT2B Antibody in MDBK Cell Line.]{\textbf{Representative Images of Phenotypic Diversity of hIFIT2 Interactions with bRSV Inclusion Bodies, Detected by IFIT2B Antibody in MDBK Cell Line.} MDBK cells were infected with bRSV at MOI 1 and fixed at 24 HPI. Cellular nuclei were stained with DAPI (yellow), and cells were double-labeled with anti-RSV N (cyan) and anti-IFIT2B (magenta) antibodies. This figure showcases representative examples of relevant phenotypes in the interaction between bIFIT2, detected by IFIT2B antibody, and bRSV inclusion bodies. These phenotypes are presented in descending order based on their percentage proportions. The scale bar indicates 2 \(\mu m\).}
    \label{fig:Representative Images of Phenotypic Diversity of hIFIT2 Interactions with bRSV Inclusion Bodies, Detected by IFIT2B Antibody in MDBK Cell Line}
\end{figure}

71 21 6

2 8 17

\subsubsection{Phenotypic Diversity of Nascent IFIT3 Interaction with RSV Inclusion Bodies}
Nascent human IFIT3 seems to have mainly diffused phenotype (top and bottom panel) with occasional exclusion without any marked IFIT3 concentration adjacent to the IB structure (middle panel).

53, 17, 16, 10

4.5, 12, 5, 1.9

\lipsum[1-5]

\begin{figure}
    \begin{subfigure}{0.495\textwidth}
        \caption{}
        \includegraphics[width=1\linewidth]{08. Chapter 3/Figs/02. Infection/03. IFIT3/01. bar_i3_a549.pdf} 
    \end{subfigure}
    \begin{subfigure}{0.495\textwidth}
        \caption{}
        \includegraphics[width=1\linewidth]{08. Chapter 3/Figs/02. Infection/03. IFIT3/02. box_i3_a549.pdf}
    \end{subfigure}
    \caption[Phenotypic Diversity of hIFIT3 Interactions with hRSV Inclusion Bodies in A549 Cell Line.]{\textbf{Phenotypic Diversity of hIFIT3 Interactions with hRSV Inclusion Bodies in A549 Cell Line.} A549 cells were infected with human RSV at MOI 1 and fixed 24 HPI. Cells were double-labeled with with anti-RSV N and anti-IFIT3 antibodies and imaged on confocal microscope. Panel (a) shows percentual proportions of observed phenotypes between hRSV inclusion bodies and hIFIT3 (80 observations), with the red dotted line denoting the 5\% threshold, marking phenotypes considered relevant above this limit. Panel (b) shows the IB area in \(\mu m^2\) per observed relevant phenotype.}
    \label{fig:Phenotypic Diversity of hIFIT3 Interactions with hRSV Inclusion Bodies in A549 Cell Line}
\end{figure}

\begin{figure}
    \centering
    \includegraphics[width=1\linewidth]{08. Chapter 3/Figs/02. Infection/03. IFIT3/03. a549 i3.pdf}
    \caption[Representative Images of Phenotypic Diversity of hIFIT3 Interactions with hRSV Inclusion Bodies in A549 Cell Line.]{\textbf{Representative Images of Phenotypic Diversity of hIFIT3 Interactions with hRSV Inclusion Bodies in A549 Cell Line.} A549 cells were infected with hRSV at MOI 1 and fixed at 24 HPI. Cellular nuclei were stained with DAPI (yellow), and cells were double-labeled with anti-RSV N (cyan) and anti-IFIT3 (magenta) antibodies. This figure showcases representative examples of relevant phenotypes in the interaction between hIFIT3 and hRSV inclusion bodies. These phenotypes are presented in descending order based on their percentage proportions. The scale bar indicates 2 \(\mu m\).}
    \label{fig:Representative Images of Phenotypic Diversity of hIFIT3 Interactions with hRSV Inclusion Bodies in A549 Cell Line}
\end{figure}

62, 19, 12

5.5, 3, 10, 2.3

\begin{figure}
    \begin{subfigure}{0.495\textwidth}
        \caption{}
        \includegraphics[width=1\linewidth]{08. Chapter 3/Figs/02. Infection/03. IFIT3/04. bar_i3_beas2b.pdf} 
    \end{subfigure}
    \begin{subfigure}{0.495\textwidth}
        \caption{}        
        \includegraphics[width=1\linewidth]{08. Chapter 3/Figs/02. Infection/03. IFIT3/05. box_i3_beas2b.pdf}
    \end{subfigure}
    \caption[Phenotypic Diversity of hIFIT3 Interactions with hRSV Inclusion Bodies in BEAS2B Cell Line.]{\textbf{Phenotypic Diversity of hIFIT3 Interactions with hRSV Inclusion Bodies in BEAS2B Cell Line.} BEAS2B cells were infected with human RSV at MOI 1 and fixed 24 HPI. Cells were labeled with anti-RSV N and anti-IFIT3 antibodies and imaged on confocal microscope. Panel (a) shows percentual proportions of observed phenotypes between hRSV inclusion bodies and hIFIT3 (16 observations), with the red dotted line denoting the 5\% threshold, marking phenotypes considered relevant above this limit. Panel (b) shows the IB area in \(\mu m^2\) per observed relevant phenotype.}
    \label{fig:Phenotypic Diversity of hIFIT3 Interactions with hRSV Inclusion Bodies in BEAS2B Cell Line}
\end{figure}

\begin{figure}
    \centering
    \includegraphics[width=1\linewidth]{08. Chapter 3/Figs/02. Infection/03. IFIT3/06. beas2b i3.pdf}
    \caption[Representative Images of Phenotypic Diversity of hIFIT3 Interactions with hRSV Inclusion Bodies in BEAS2B Cell Line]{\textbf{Representative Images of Phenotypic Diversity of hIFIT3 Interactions with hRSV Inclusion Bodies in BEAS2B Cell Line.} BEAS2B cells were infected with hRSV at MOI 1 and fixed at 24 HPI. Cellular nuclei were stained with DAPI (yellow), and cells were double-labeled with anti-RSV N (cyan) and anti-IFIT3 (magenta) antibodies. This figure showcases representative examples of relevant phenotypes in the interaction between hIFIT3 and hRSV inclusion bodies. These phenotypes are presented in descending order based on their percentage proportions. The scale bar indicates 2 \(\mu m\).}
    \label{fig:Representative Images of Phenotypic Diversity of hIFIT3 Interactions with hRSV Inclusion Bodies in BEAS2B Cell Line}
\end{figure}

43, 34, 11, 8

1.1, 3.3, 1, 11

\begin{figure}
    \begin{subfigure}{0.495\textwidth}
        \caption{}
        \includegraphics[width=1\linewidth]{08. Chapter 3/Figs/02. Infection/03. IFIT3/07. bar_i3_mdbk.pdf} 
    \end{subfigure}
    \begin{subfigure}{0.495\textwidth}
        \caption{}
        \includegraphics[width=1\linewidth]{08. Chapter 3/Figs/02. Infection/03. IFIT3/08. box_i3_mdbk.pdf}
    \end{subfigure}
    \caption[Phenotypic Diversity of bIFIT3 Interactions with bRSV Inclusion Bodies in MDBK Cell Line.]{\textbf{Phenotypic Diversity of bIFIT3 Interactions with bRSV Inclusion Bodies in MDBK Cell Line.} MDBK cells were infected with bovine RSV at MOI 1 and fixed 24 HPI. Cells were labeled with anti-RSV N and anti-IFIT3 antibodies and imaged on confocal microscope. Panel (a) shows percentual proportions of observed phenotypes between bRSV inclusion bodies and bIFIT3 (214 observations), with the red dotted line denoting the 5\% threshold, marking phenotypes considered relevant above this limit. Panel (b) shows the IB area in \(\mu m^2\) per observed relevant phenotype.}
    \label{fig:Phenotypic Diversity of bIFIT3 Interactions with bRSV Inclusion Bodies in MDBK Cell Line}
\end{figure}

\begin{figure}
    \centering
    \includegraphics[width=1\linewidth]{08. Chapter 3/Figs/02. Infection/03. IFIT3/09. mdbk i3.pdf}
    \caption[Representative Images of Phenotypic Diversity of bIFIT3 Interactions with bRSV Inclusion Bodies in MDBK Cell Line.]{\textbf{Representative Images of Phenotypic Diversity of bIFIT3 Interactions with bRSV Inclusion Bodies in MDBK Cell Line.} MDBK cells were infected with bRSV at MOI 1 and fixed at 24 HPI. Cellular nuclei were stained with DAPI (yellow), and cells were double-labeled with anti-RSV N (cyan) and anti-IFIT3 (magenta) antibodies. This figure showcases representative examples of relevant phenotypes in the interaction between bIFIT3 and bRSV inclusion bodies. These phenotypes are presented in descending order based on their percentage proportions. The scale bar indicates 2 \(\mu m\).}
    \label{fig:Representative Images of Phenotypic Diversity of bIFIT3 Interactions with bRSV Inclusion Bodies in MDBK Cell Line}
\end{figure}

\subsubsection{Phenotypic Diversity of Nascent IFIT5 Interaction with RSV Inclusion Bodies}
hIFIT5 seems to be excluded from hRSV IBs. There is a hint of accumulation of IFIT5 on the outside of IB (bottom panel; no z stacks to confirm this). 

58, 17, 16, 5

6.3, 4, 8, 13

\begin{figure}
    \begin{subfigure}{0.495\textwidth}
        \caption{}
        \includegraphics[width=1\linewidth]{08. Chapter 3/Figs/02. Infection/04. IFIT5/01. bar_i5_a549.pdf} 
    \end{subfigure}
    \begin{subfigure}{0.495\textwidth}
        \caption{}
        \includegraphics[width=1\linewidth]{08. Chapter 3/Figs/02. Infection/04. IFIT5/02. box_i5_a549.pdf}
    \end{subfigure}
    \caption[Phenotypic Diversity of hIFIT5 Interactions with hRSV Inclusion Bodies in A549 Cell Line.]{\textbf{Phenotypic Diversity of hIFIT5 Interactions with hRSV Inclusion Bodies in A549 Cell Line.} A549 cells were infected with human RSV at MOI 1 and fixed 24 HPI. Cells were labeled with anti-RSV N and anti-IFIT5 antibodies and imaged on confocal microscope. Panel (a) shows percentual proportions of observed phenotypes between hRSV inclusion bodies and hIFIT5 (77 observations), with the red dotted line denoting the 5\% threshold, marking phenotypes considered relevant above this limit. Panel (b) shows the IB area in \(\mu m^2\) per observed relevant phenotype.}
    \label{fig:Phenotypic Diversity of hIFIT5 Interactions with hRSV Inclusion Bodies in A549 Cell Line}
\end{figure}

\begin{figure}
    \centering
    \includegraphics[width=1\linewidth]{08. Chapter 3/Figs/02. Infection/04. IFIT5/03. a549 i5.pdf}
    \caption[Representative Images of Phenotypic Diversity of hIFIT5 Interactions with hRSV Inclusion Bodies in A549 Cell Line.]{\textbf{Representative Images of Phenotypic Diversity of hIFIT5 Interactions with hRSV Inclusion Bodies in A549 Cell Line.} A549 cells were infected with hRSV at MOI 1 and fixed at 24 HPI. Cellular nuclei were stained with DAPI (yellow), and cells were double-labeled with anti-RSV N (cyan) and anti-IFIT5 (magenta) antibodies. This figure showcases representative examples of relevant phenotypes in the interaction between hIFIT5 and hRSV inclusion bodies. These phenotypes are presented in descending order based on their percentage proportions. The scale bar indicates 2 \(\mu m\).}
    \label{fig:Representative Images of Phenotypic Diversity of hIFIT5 Interactions with hRSV Inclusion Bodies in A549 Cell Line}
\end{figure}

62, 18, 18

2, 2.3, 3

\begin{figure}
    \begin{subfigure}{0.495\textwidth}
        \caption{}
        \includegraphics[width=1\linewidth]{08. Chapter 3/Figs/02. Infection/04. IFIT5/04. bar_i5_beas2b.pdf}
    \end{subfigure}
    \begin{subfigure}{0.495\textwidth}
        \caption{}
        \includegraphics[width=1\linewidth]{08. Chapter 3/Figs/02. Infection/04. IFIT5/05. box_i5_beas2b.pdf}
    \end{subfigure}
    \caption[Phenotypic Diversity of hIFIT5 Interactions with hRSV Inclusion Bodies in BEAS2B Cell Line.]{\textbf{Phenotypic Diversity of hIFIT5 Interactions with hRSV Inclusion Bodies in BEAS2B Cell Line.} BEAS2B cells were infected with human RSV at MOI 1 and fixed 24 HPI. Cells were labeled with anti-RSV N and anti-IFIT5 antibodies and imaged on confocal microscope. Panel (a) shows percentual proportions of observed phenotypes between hRSV inclusion bodies and hIFIT5 (21 observations), with the red dotted line denoting the 5\% threshold, marking phenotypes considered relevant above this limit. Panel (b) shows the IB area in \(\mu m^2\) per observed relevant phenotype.}
    \label{fig:Phenotypic Diversity of hIFIT5 Interactions with hRSV Inclusion Bodies in BEAS2B Cell Line}
\end{figure}

\begin{figure}
    \centering
    \includegraphics[width=1\linewidth]{08. Chapter 3/Figs/02. Infection/04. IFIT5/06. beas2b i5.pdf}
    \caption[Representative Images of Phenotypic Diversity of hIFIT5 Interactions with hRSV Inclusion Bodies in BEAS2B Cell Line.]{\textbf{Representative Images of Phenotypic Diversity of hIFIT5 Interactions with hRSV Inclusion Bodies in BEAS2B Cell Line.} BEAS2B cells were infected with hRSV at MOI 1 and fixed at 24 HPI. Cellular nuclei were stained with DAPI (yellow), and cells were double-labeled with anti-RSV N (cyan) and anti-IFIT5 (magenta) antibodies. This figure showcases representative examples of relevant phenotypes in the interaction between hIFIT5 and hRSV inclusion bodies. These phenotypes are presented in descending order based on their percentage proportions. The scale bar indicates 2 \(\mu m\).}
    \label{fig:Representative Images of Phenotypic Diversity of hIFIT5 Interactions with hRSV Inclusion Bodies in BEAS2B Cell Line}
\end{figure}

51, 29, 10, 7

1, 10, 3, 0.9

\begin{figure}
    \begin{subfigure}{0.495\textwidth}
        \caption{}
        \includegraphics[width=1\linewidth]{08. Chapter 3/Figs/02. Infection/04. IFIT5/07. bar_i5_mdbk.pdf} 
    \end{subfigure}
    \begin{subfigure}{0.495\textwidth}
        \caption{}
        \includegraphics[width=1\linewidth]{08. Chapter 3/Figs/02. Infection/04. IFIT5/08. box_i5_mdbk.pdf}
    \end{subfigure}
    \caption[Phenotypic Diversity of bIFIT5 Interactions with bRSV Inclusion Bodies in MDBK Cell Line.]{\textbf{Phenotypic Diversity of bIFIT5 Interactions with bRSV Inclusion Bodies in MDBK Cell Line.} MDBK cells were infected with bovine RSV at MOI 1 and fixed 24 HPI. Cells were labeled with anti-RSV N and anti-IFIT5 antibodies and imaged on confocal microscope. Panel (a) shows percentual proportions of observed phenotypes between bRSV inclusion bodies and bIFIT5 (61 observations), with the red dotted line denoting the 5\% threshold, marking phenotypes considered relevant above this limit. Panel (b) shows the IB area in \(\mu m^2\) per observed relevant phenotype.}
    \label{fig:Phenotypic Diversity of bIFIT5 Interactions with bRSV Inclusion Bodies in MDBK Cell Line}
\end{figure}

\begin{figure}
    \centering
    \includegraphics[width=1\linewidth]{08. Chapter 3/Figs/02. Infection/04. IFIT5/09. mdbk i5.pdf}
    \caption[Representative Images of Phenotypic Diversity of bIFIT5 Interactions with bRSV Inclusion Bodies in MDBK Cell Line.]{\textbf{Representative Images of Phenotypic Diversity of bIFIT5 Interactions with bRSV Inclusion Bodies in MDBK Cell Line.} MDBK cells were infected with bRSV at MOI 1 and fixed at 24 HPI. Cellular nuclei were stained with DAPI (yellow), and cells were double-labeled with anti-RSV N (cyan) and anti-IFIT5 (magenta) antibodies. This figure showcases representative examples of relevant phenotypes in the interaction between bIFIT5 and bRSV inclusion bodies. These phenotypes are presented in descending order based on their percentage proportions. The scale bar indicates 2 \(\mu m\).}
    \label{fig:Representative Images of Phenotypic Diversity of bIFIT5 Interactions with bRSV Inclusion Bodies in MDBK Cell Line}
\end{figure}

% summary
In the context of infection, endogenous human IFIT1 concentrates within the human RSV IB structure; colocalises to the edge of the IB; is diffused through the structure and cytoplasm equally; or is excluded from the structure. This suggests that the interaction between human IFIT1 and hRSV IB is dynamic and depends on factors that we do not understand yet. In the case of endogenous bovine IFIT1 in the context of bRSV IBs, IFIT1 is either excluded from the structure; excluded from the IB inner edge but concentrated inside; or excluded from the centre of IB structure but concentrated on the inner edge of the structure. 

Nascent human IFIT3 during hRSV infection is either excluded from IB structure or is diffused through the structure. Occasionally it colocalises to the IB ring. Nascent bIFIT3 during bRSV infection either siphons inside IBs and shows sub-IB granules or is excluded from the IB boundary with slightly decreased signal inside of the IB.

In human cells during hRSV infection IFIT5 is mainly excluded from the IBs but seems to concentrate on their edge. Once we saw colocalization with the IB ring and a concentration of IFIT5 inside it. In bovine cells IFIT5 is always excluded from the IB boundary and the signal inside is either slightly decreased or equal compared to cytoplasmic IFIT5. 
