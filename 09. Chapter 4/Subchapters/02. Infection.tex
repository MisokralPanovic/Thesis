\subsection{Nascent IFIT1, IFIT3, and IFIT5 Localisation During RSV Infection} \label{subsec:Nascent IFIT1, IFIT3, and IFIT5 Localisation During RSV Infection}
\subsubsection{Phenotypic Diversity of Nascent Human and Bovine IFIT1 Interaction with RSV Inclusion Bodies}
Nascent human IFIT1 shows several distinct phenotypes with respect to the hRSV IB interaction. IFIT1 is either concentrated inside the structure (top panel), concentrated on the edge of IB ring (2nd and 3rd panels)  , excluded from the IB structure (4th panel),  or is diffused evenly between the cytoplasm and IB structure (bottom panel).

\begin{figure}
    \begin{subfigure}{0.495\textwidth}
        \caption{}
        \includegraphics[width=1\linewidth]{09. Chapter 4/Figs/02. Infection/01. IFIT1/01. bar_i1_a549.pdf} 
    \end{subfigure}
    \begin{subfigure}{0.495\textwidth}
        \caption{}
        \includegraphics[width=1\linewidth]{09. Chapter 4/Figs/02. Infection/01. IFIT1/02. box_i1_a549.pdf}
    \end{subfigure}
    \caption[Phenotypic Diversity of hIFIT1 Interactions with hRSV Inclusion Bodies in A549 Cell Line.]{\textbf{Phenotypic Diversity of hIFIT1 Interactions with hRSV Inclusion Bodies in A549 Cell Line.} A549 cells were infected with human RSV at MOI 1 and fixed 24 HPI. Cells were double-labeled with with anti-RSV N and anti-IFIT1 antibodies and imaged on confocal microscope. Panel (a) shows percentual proportions of observed phenotypes between hRSV inclusion bodies and hIFIT1 (281 observations), with the red dotted line denoting the 5\% threshold, marking phenotypes considered relevant above this limit. Panel (b) shows the IB area in (\(\mu m^2\)) per observed relevant phenotype.}
    \label{fig:Phenotypic Diversity of hIFIT1 Interactions with hRSV Inclusion Bodies in A549 Cell Line}
\end{figure}

\begin{figure}
    \centering
    \includegraphics[width=1\linewidth]{09. Chapter 4/Figs/02. Infection/01. IFIT1/03. a549 i1.pdf}
    \caption[Representative Images of Phenotypic Diversity of hIFIT1 Interactions with hRSV Inclusion Bodies in A549 Cell Line.]{\textbf{Representative Images of Phenotypic Diversity of hIFIT1 Interactions with hRSV Inclusion Bodies in A549 Cell Line.} A549 cells were infected with hRSV at MOI 1 and fixed at 24 HPI. Cellular nuclei were stained with DAPI (yellow), and cells were double-labeled with anti-RSV N (cyan) and anti-IFIT1 (magenta) antibodies. This figure showcases representative examples of relevant phenotypes in the interaction between hIFIT1 and hRSV inclusion bodies. These phenotypes are presented in descending order based on their percentage proportions. The scale bar indicates 2 (\(\mu m\)).}
    \label{fig:Representative Images of Phenotypic Diversity of hIFIT1 Interactions with hRSV Inclusion Bodies in A549 Cell Line}
\end{figure}

\begin{figure}
    \begin{subfigure}{0.495\textwidth}
        \caption{}
        \includegraphics[width=1\linewidth]{09. Chapter 4/Figs/02. Infection/01. IFIT1/04. bar_i1_beas2b.pdf} 
    \end{subfigure}
    \begin{subfigure}{0.495\textwidth}
        \caption{}
        \includegraphics[width=1\linewidth]{09. Chapter 4/Figs/02. Infection/01. IFIT1/05. box_i1_beas2b.pdf}
    \end{subfigure}
    \caption[Phenotypic Diversity of hIFIT1 Interactions with hRSV Inclusion Bodies in BEAS2B Cell Line.]{\textbf{Phenotypic Diversity of hIFIT1 Interactions with hRSV Inclusion Bodies in BEAS2B Cell Line.} BEAS2B cells were infected with human RSV at MOI 1 and fixed 24 HPI. Cells were double-labeled with with anti-RSV N and anti-IFIT1 antibodies and imaged on confocal microscope. Panel (a) shows percentual proportions of observed phenotypes between hRSV inclusion bodies and hIFIT1 (281 observations), with the red dotted line denoting the 5\% threshold, marking phenotypes considered relevant above this limit. Panel (b) shows the IB area in (\(\mu m^2\)) per observed relevant phenotype.}
    \label{fig:Phenotypic Diversity of hIFIT1 Interactions with hRSV Inclusion Bodies in BEAS2B Cell Line}
\end{figure}

\begin{figure}
    \centering
    \includegraphics[width=1\linewidth]{09. Chapter 4/Figs/02. Infection/01. IFIT1/06. beas2b i1.pdf}
    \caption[Representative Images of Phenotypic Diversity of hIFIT1 Interactions with hRSV Inclusion Bodies in BEAS2B Cell Line.]{\textbf{Representative Images of Phenotypic Diversity of hIFIT1 Interactions with hRSV Inclusion Bodies in BEAS2B Cell Line.} BEAS2B cells were infected with hRSV at MOI 1 and fixed at 24 HPI. Cellular nuclei were stained with DAPI (yellow), and cells were double-labeled with anti-RSV N (cyan) and anti-IFIT1 (magenta) antibodies. This figure showcases representative examples of relevant phenotypes in the interaction between hIFIT1 and hRSV inclusion bodies. These phenotypes are presented in descending order based on their percentage proportions. The scale bar indicates 2 (\(\mu m\)).}
    \label{fig:Representative Images of Phenotypic Diversity of hIFIT1 Interactions with hRSV Inclusion Bodies in BEAS2B Cell Line}
\end{figure}

Nascent bovine IFIT1 in the context of bRSV infection has been observed to localise with the respect of IB in three distinct spaces. We observed it either concentrated inside the central point of the IB structure, while having reduced signal on the inner IB edge, compared to the cytoplasm (top and bottom panels), being excluded from the IB structure (3rd panel), or colocalising on the inner edge of the IB structure while having reduced signal in the middle of the structure compared to cytoplasm, or the edge staining (2nd panel).

\begin{figure}
    \begin{subfigure}{0.495\textwidth}
        \caption{}
        \includegraphics[width=1\linewidth]{09. Chapter 4/Figs/02. Infection/01. IFIT1/07. bar_i1_mdbk.pdf} 
    \end{subfigure}
    \begin{subfigure}{0.495\textwidth}
        \caption{}
        \includegraphics[width=1\linewidth]{09. Chapter 4/Figs/02. Infection/01. IFIT1/08. box_i1_mdbk.pdf}
    \end{subfigure}
    \caption[Phenotypic Diversity of bIFIT1 Interactions with bRSV Inclusion Bodies in MDBK Cell Line.]{\textbf{Phenotypic Diversity of bIFIT1 Interactions with bRSV Inclusion Bodies in MDBK Cell Line.} MDBK cells were infected with bovine RSV at MOI 1 and fixed 24 HPI. Cells were double-labeled with with anti-RSV N and anti-IFIT1 antibodies and imaged on confocal microscope. Panel (a) shows percentual proportions of observed phenotypes between bRSV inclusion bodies and bIFIT1 (117 observations), with the red dotted line denoting the 5\% threshold, marking phenotypes considered relevant above this limit. Panel (b) shows the IB area in (\(\mu m^2\)) per observed relevant phenotype.}
    \label{fig:Phenotypic Diversity of bIFIT1 Interactions with bRSV Inclusion Bodies in MDBK Cell Line}
\end{figure}

\begin{figure}
    \centering
    \includegraphics[width=1\linewidth]{09. Chapter 4/Figs/02. Infection/01. IFIT1/09. mdbk i1.pdf}
    \caption[Representative Images of Phenotypic Diversity of bIFIT1 Interactions with bRSV Inclusion Bodies in MDBK Cell Line.]{\textbf{Representative Images of Phenotypic Diversity of bIFIT1 Interactions with bRSV Inclusion Bodies in MDBK Cell Line.} MDBK cells were infected with bRSV at MOI 1 and fixed at 24 HPI. Cellular nuclei were stained with DAPI (yellow), and cells were double-labeled with anti-RSV N (cyan) and anti-IFIT1 (magenta) antibodies. This figure showcases representative examples of relevant phenotypes in the interaction between bIFIT1 and bRSV inclusion bodies. These phenotypes are presented in descending order based on their percentage proportions. The scale bar indicates 2 (\(\mu m\)).}
    \label{fig:Representative Images of Phenotypic Diversity of bIFIT1 Interactions with bRSV Inclusion Bodies in MDBK Cell Line}
\end{figure}

\subsubsection{Phenotypic Diversity of Nascent Human and Bovine IFIT3 Interaction with RSV Inclusion Bodies}
Nascent human IFIT3 seems to have mainly diffused phenotype (top and bottom panel) with occasional exclusion without any marked IFIT3 concentration adjacent to the IB structure (middle panel).

\begin{figure}
    \begin{subfigure}{0.495\textwidth}
        \caption{}
        \includegraphics[width=1\linewidth]{09. Chapter 4/Figs/02. Infection/02. IFIT3/01. bar_i3_a549.pdf} 
    \end{subfigure}
    \begin{subfigure}{0.495\textwidth}
        \caption{}
        \includegraphics[width=1\linewidth]{09. Chapter 4/Figs/02. Infection/02. IFIT3/02. box_i3_a549.pdf}
    \end{subfigure}
    \caption[Phenotypic Diversity of hIFIT3 Interactions with hRSV Inclusion Bodies in A549 Cell Line.]{\textbf{Phenotypic Diversity of hIFIT3 Interactions with hRSV Inclusion Bodies in A549 Cell Line.} A549 cells were infected with human RSV at MOI 1 and fixed 24 HPI. Cells were double-labeled with with anti-RSV N and anti-IFIT3 antibodies and imaged on confocal microscope. Panel (a) shows percentual proportions of observed phenotypes between hRSV inclusion bodies and hIFIT3 (80 observations), with the red dotted line denoting the 5\% threshold, marking phenotypes considered relevant above this limit. Panel (b) shows the IB area in (\(\mu m^2\)) per observed relevant phenotype.}
    \label{fig:Phenotypic Diversity of hIFIT3 Interactions with hRSV Inclusion Bodies in A549 Cell Line}
\end{figure}

\begin{figure}
    \centering
    \includegraphics[width=1\linewidth]{09. Chapter 4/Figs/02. Infection/02. IFIT3/03. a549 i3.pdf}
    \caption[Representative Images of Phenotypic Diversity of hIFIT3 Interactions with hRSV Inclusion Bodies in A549 Cell Line.]{\textbf{Representative Images of Phenotypic Diversity of hIFIT3 Interactions with hRSV Inclusion Bodies in A549 Cell Line.} A549 cells were infected with hRSV at MOI 1 and fixed at 24 HPI. Cellular nuclei were stained with DAPI (yellow), and cells were double-labeled with anti-RSV N (cyan) and anti-IFIT3 (magenta) antibodies. This figure showcases representative examples of relevant phenotypes in the interaction between hIFIT3 and hRSV inclusion bodies. These phenotypes are presented in descending order based on their percentage proportions. The scale bar indicates 2 (\(\mu m\)).}
    \label{fig:Representative Images of Phenotypic Diversity of hIFIT3 Interactions with hRSV Inclusion Bodies in A549 Cell Line}
\end{figure}

\begin{figure}
    \begin{subfigure}{0.495\textwidth}
        \caption{}
        \includegraphics[width=1\linewidth]{09. Chapter 4/Figs/02. Infection/02. IFIT3/04. bar_i3_beas2b.pdf} 
    \end{subfigure}
    \begin{subfigure}{0.495\textwidth}
        \caption{}        
        \includegraphics[width=1\linewidth]{09. Chapter 4/Figs/02. Infection/02. IFIT3/05. box_i3_beas2b.pdf}
    \end{subfigure}
    \caption[Phenotypic Diversity of hIFIT3 Interactions with hRSV Inclusion Bodies in BEAS2B Cell Line.]{\textbf{Phenotypic Diversity of hIFIT3 Interactions with hRSV Inclusion Bodies in BEAS2B Cell Line.} BEAS2B cells were infected with human RSV at MOI 1 and fixed 24 HPI. Cells were labeled with anti-RSV N and anti-IFIT3 antibodies and imaged on confocal microscope. Panel (a) shows percentual proportions of observed phenotypes between hRSV inclusion bodies and hIFIT3 (16 observations), with the red dotted line denoting the 5\% threshold, marking phenotypes considered relevant above this limit. Panel (b) shows the IB area in (\(\mu m^2\)) per observed relevant phenotype.}
    \label{fig:Phenotypic Diversity of hIFIT3 Interactions with hRSV Inclusion Bodies in BEAS2B Cell Line}
\end{figure}

\begin{figure}
    \centering
    \includegraphics[width=1\linewidth]{09. Chapter 4/Figs/02. Infection/02. IFIT3/06. beas2b i3.pdf}
    \caption[Representative Images of Phenotypic Diversity of hIFIT3 Interactions with hRSV Inclusion Bodies in BEAS2B Cell Line]{\textbf{Representative Images of Phenotypic Diversity of hIFIT3 Interactions with hRSV Inclusion Bodies in BEAS2B Cell Line.} BEAS2B cells were infected with hRSV at MOI 1 and fixed at 24 HPI. Cellular nuclei were stained with DAPI (yellow), and cells were double-labeled with anti-RSV N (cyan) and anti-IFIT3 (magenta) antibodies. This figure showcases representative examples of relevant phenotypes in the interaction between hIFIT3 and hRSV inclusion bodies. These phenotypes are presented in descending order based on their percentage proportions. The scale bar indicates 2 (\(\mu m\)).}
    \label{fig:Representative Images of Phenotypic Diversity of hIFIT3 Interactions with hRSV Inclusion Bodies in BEAS2B Cell Line}
\end{figure}

In this experiment the nascent bovine IFIT3 was consistently concentrated inside IBs. In some IBs the IFIT3 signal showed signs of sub-concentrations within the inclusions (bottom panel; highlighted with arrows), resembling IBAGs (inclusion body associated granules).

Subsequent experiments did not recapitulate the IFIT3 inclusions. It however shows signal which is almost identical to what is observed with IFIT5 in MDBKs during bRSV infection i.e., decrease, but not complete abolishment of IFIT3 signal inside the IB structure (top panel), and IB boundary exclusion while maintaining similar levels of signal intensity between cytoplasmic and intra IB stain.

\begin{figure}
    \begin{subfigure}{0.495\textwidth}
        \caption{}
        \includegraphics[width=1\linewidth]{09. Chapter 4/Figs/02. Infection/02. IFIT3/07. bar_i3_mdbk.pdf} 
    \end{subfigure}
    \begin{subfigure}{0.495\textwidth}
        \caption{}
        \includegraphics[width=1\linewidth]{09. Chapter 4/Figs/02. Infection/02. IFIT3/08. box_i3_mdbk.pdf}
    \end{subfigure}
    \caption[Phenotypic Diversity of bIFIT3 Interactions with bRSV Inclusion Bodies in MDBK Cell Line.]{\textbf{Phenotypic Diversity of bIFIT3 Interactions with bRSV Inclusion Bodies in MDBK Cell Line.} MDBK cells were infected with bovine RSV at MOI 1 and fixed 24 HPI. Cells were labeled with anti-RSV N and anti-IFIT3 antibodies and imaged on confocal microscope. Panel (a) shows percentual proportions of observed phenotypes between bRSV inclusion bodies and bIFIT3 (214 observations), with the red dotted line denoting the 5\% threshold, marking phenotypes considered relevant above this limit. Panel (b) shows the IB area in (\(\mu m^2\)) per observed relevant phenotype.}
    \label{fig:Phenotypic Diversity of bIFIT3 Interactions with bRSV Inclusion Bodies in MDBK Cell Line}
\end{figure}

\begin{figure}
    \centering
    \includegraphics[width=1\linewidth]{09. Chapter 4/Figs/02. Infection/02. IFIT3/09. mdbk i3.pdf}
    \caption[Representative Images of Phenotypic Diversity of bIFIT3 Interactions with bRSV Inclusion Bodies in MDBK Cell Line.]{\textbf{Representative Images of Phenotypic Diversity of bIFIT3 Interactions with bRSV Inclusion Bodies in MDBK Cell Line.} BEAS2B cells were infected with bRSV at MOI 1 and fixed at 24 HPI. Cellular nuclei were stained with DAPI (yellow), and cells were double-labeled with anti-RSV N (cyan) and anti-IFIT3 (magenta) antibodies. This figure showcases representative examples of relevant phenotypes in the interaction between bIFIT3 and bRSV inclusion bodies. These phenotypes are presented in descending order based on their percentage proportions. The scale bar indicates 2 (\(\mu m\)).}
    \label{fig:Representative Images of Phenotypic Diversity of bIFIT3 Interactions with bRSV Inclusion Bodies in MDBK Cell Line}
\end{figure}

\subsubsection{Phenotypic Diversity of Nascent Human and Bovine IFIT5 Interaction with RSV Inclusion Bodies}
hIFIT5 seems to be excluded from hRSV IBs. There is a hint of accumulation of IFIT5 on the outside of IB (bottom panel; no z stacks to confirm this). 

\begin{figure}
    \begin{subfigure}{0.495\textwidth}
        \caption{}
        \includegraphics[width=1\linewidth]{09. Chapter 4/Figs/02. Infection/03. IFIT5/01. bar_i5_a549.pdf} 
    \end{subfigure}
    \begin{subfigure}{0.495\textwidth}
        \caption{}
        \includegraphics[width=1\linewidth]{09. Chapter 4/Figs/02. Infection/03. IFIT5/02. box_i5_a549.pdf}
    \end{subfigure}
    \caption[Phenotypic Diversity of hIFIT5 Interactions with hRSV Inclusion Bodies in A549 Cell Line.]{\textbf{Phenotypic Diversity of hIFIT5 Interactions with hRSV Inclusion Bodies in A549 Cell Line.} A549 cells were infected with human RSV at MOI 1 and fixed 24 HPI. Cells were labeled with anti-RSV N and anti-IFIT5 antibodies and imaged on confocal microscope. Panel (a) shows percentual proportions of observed phenotypes between hRSV inclusion bodies and hIFIT5 (77 observations), with the red dotted line denoting the 5\% threshold, marking phenotypes considered relevant above this limit. Panel (b) shows the IB area in (\(\mu m^2\)) per observed relevant phenotype.}
    \label{fig:Phenotypic Diversity of hIFIT5 Interactions with hRSV Inclusion Bodies in A549 Cell Line}
\end{figure}

\begin{figure}
    \centering
    \includegraphics[width=1\linewidth]{09. Chapter 4/Figs/02. Infection/03. IFIT5/03. a549 i5.pdf}
    \caption[Representative Images of Phenotypic Diversity of hIFIT5 Interactions with hRSV Inclusion Bodies in A549 Cell Line.]{\textbf{Representative Images of Phenotypic Diversity of hIFIT5 Interactions with hRSV Inclusion Bodies in A549 Cell Line.} A549 cells were infected with hRSV at MOI 1 and fixed at 24 HPI. Cellular nuclei were stained with DAPI (yellow), and cells were double-labeled with anti-RSV N (cyan) and anti-IFIT5 (magenta) antibodies. This figure showcases representative examples of relevant phenotypes in the interaction between hIFIT5 and hRSV inclusion bodies. These phenotypes are presented in descending order based on their percentage proportions. The scale bar indicates 2 (\(\mu m\)).}
    \label{fig:Representative Images of Phenotypic Diversity of hIFIT5 Interactions with hRSV Inclusion Bodies in A549 Cell Line}
\end{figure}

\begin{figure}
    \begin{subfigure}{0.495\textwidth}
        \caption{}
        \includegraphics[width=1\linewidth]{09. Chapter 4/Figs/02. Infection/03. IFIT5/04. bar_i5_beas2b.pdf}
    \end{subfigure}
    \begin{subfigure}{0.495\textwidth}
        \caption{}
        \includegraphics[width=1\linewidth]{09. Chapter 4/Figs/02. Infection/03. IFIT5/05. box_i5_beas2b.pdf}
    \end{subfigure}
    \caption[Phenotypic Diversity of hIFIT5 Interactions with hRSV Inclusion Bodies in BEAS2B Cell Line.]{\textbf{Phenotypic Diversity of hIFIT5 Interactions with hRSV Inclusion Bodies in BEAS2B Cell Line.} BEAS2B cells were infected with human RSV at MOI 1 and fixed 24 HPI. Cells were labeled with anti-RSV N and anti-IFIT5 antibodies and imaged on confocal microscope. Panel (a) shows percentual proportions of observed phenotypes between hRSV inclusion bodies and hIFIT5 (21 observations), with the red dotted line denoting the 5\% threshold, marking phenotypes considered relevant above this limit. Panel (b) shows the IB area in (\(\mu m^2\)) per observed relevant phenotype.}
    \label{fig:Phenotypic Diversity of hIFIT5 Interactions with hRSV Inclusion Bodies in BEAS2B Cell Line}
\end{figure}

\begin{figure}
    \centering
    \includegraphics[width=1\linewidth]{09. Chapter 4/Figs/02. Infection/03. IFIT5/06. beas2b i5.pdf}
    \caption[Representative Images of Phenotypic Diversity of hIFIT5 Interactions with hRSV Inclusion Bodies in BEAS2B Cell Line.]{\textbf{Representative Images of Phenotypic Diversity of hIFIT5 Interactions with hRSV Inclusion Bodies in BEAS2B Cell Line.} BEAS2B cells were infected with hRSV at MOI 1 and fixed at 24 HPI. Cellular nuclei were stained with DAPI (yellow), and cells were double-labeled with anti-RSV N (cyan) and anti-IFIT5 (magenta) antibodies. This figure showcases representative examples of relevant phenotypes in the interaction between hIFIT5 and hRSV inclusion bodies. These phenotypes are presented in descending order based on their percentage proportions. The scale bar indicates 2 (\(\mu m\)).}
    \label{fig:Representative Images of Phenotypic Diversity of hIFIT5 Interactions with hRSV Inclusion Bodies in BEAS2B Cell Line}
\end{figure}

The distribution of bIFIT5 is equal between cytoplasmic stain and inside of the IB (with a hint of concentrations/substructures in top and bottom panel; no z stack data available). All 3 panels show a depression of IFIT5 signal at the side of IB boundary (highlighted by arrows).

\begin{figure}
    \begin{subfigure}{0.495\textwidth}
        \caption{}
        \includegraphics[width=1\linewidth]{09. Chapter 4/Figs/02. Infection/03. IFIT5/07. bar_i5_mdbk.pdf} 
    \end{subfigure}
    \begin{subfigure}{0.495\textwidth}
        \caption{}
        \includegraphics[width=1\linewidth]{09. Chapter 4/Figs/02. Infection/03. IFIT5/08. box_i5_mdbk.pdf}
    \end{subfigure}
    \caption[Phenotypic Diversity of bIFIT5 Interactions with bRSV Inclusion Bodies in MDBK Cell Line.]{\textbf{Phenotypic Diversity of bIFIT5 Interactions with bRSV Inclusion Bodies in MDBK Cell Line.} MDBK cells were infected with bovine RSV at MOI 1 and fixed 24 HPI. Cells were labeled with anti-RSV N and anti-IFIT5 antibodies and imaged on confocal microscope. Panel (a) shows percentual proportions of observed phenotypes between bRSV inclusion bodies and bIFIT5 (61 observations), with the red dotted line denoting the 5\% threshold, marking phenotypes considered relevant above this limit. Panel (b) shows the IB area in (\(\mu m^2\)) per observed relevant phenotype.}
    \label{fig:Phenotypic Diversity of bIFIT5 Interactions with bRSV Inclusion Bodies in MDBK Cell Line}
\end{figure}

\begin{figure}
    \centering
    \includegraphics[width=1\linewidth]{09. Chapter 4/Figs/02. Infection/03. IFIT5/09. mdbk i5.pdf}
    \caption[Representative Images of Phenotypic Diversity of bIFIT5 Interactions with bRSV Inclusion Bodies in MDBK Cell Line.]{\textbf{Representative Images of Phenotypic Diversity of bIFIT5 Interactions with bRSV Inclusion Bodies in MDBK Cell Line.} MDBK cells were infected with bRSV at MOI 1 and fixed at 24 HPI. Cellular nuclei were stained with DAPI (yellow), and cells were double-labeled with anti-RSV N (cyan) and anti-IFIT5 (magenta) antibodies. This figure showcases representative examples of relevant phenotypes in the interaction between bIFIT5 and bRSV inclusion bodies. These phenotypes are presented in descending order based on their percentage proportions. The scale bar indicates 2 (\(\mu m\)).}
    \label{fig:Representative Images of Phenotypic Diversity of bIFIT5 Interactions with bRSV Inclusion Bodies in MDBK Cell Line}
\end{figure}

\subsubsection{Summary} \label{Summary-infection}
In the context of infection, endogenous human IFIT1 concentrates within the human RSV IB structure; colocalises to the edge of the IB; is diffused through the structure and cytoplasm equally; or is excluded from the structure. This suggests that the interaction between human IFIT1 and hRSV IB is dynamic and depends on factors that we do not understand yet. In the case of endogenous bovine IFIT1 in the context of bRSV IBs, IFIT1 is either excluded from the structure; excluded from the IB inner edge but concentrated inside; or excluded from the centre of IB structure but concentrated on the inner edge of the structure. 

Nascent human IFIT3 during hRSV infection is either excluded from IB structure or is diffused through the structure. Occasionally it colocalises to the IB ring. Nascent bIFIT3 during bRSV infection either siphons inside IBs and shows sub-IB granules or is excluded from the IB boundary with slightly decreased signal inside of the IB.

In human cells during hRSV infection IFIT5 is mainly excluded from the IBs but seems to concentrate on their edge. Once we saw colocalization with the IB ring and a concentration of IFIT5 inside it. In bovine cells IFIT5 is always excluded from the IB boundary and the signal inside is either slightly decreased or equal compared to cytoplasmic IFIT5. 
