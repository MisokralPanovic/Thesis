\subsection{Nascent IFIT1, IFIT3, and IFIT5 Localisation During Infection} \label{subsec:Nascent IFIT1, IFIT3, and IFIT5 Localisation During Infection}
\subsubsection{Infection IFIT1}
%a549 hrsv
Detecting magenta: endogenous human IFIT1 \newline
Detecting cyan: human IB \newline
Cell Line: A549 \newline
Treatment: hRSV \newline

Nascent human IFIT1 shows several distinct phenotypes with respect to the hRSV IB interaction. IFIT1 is either concentrated inside the structure (top panel), concentrated on the edge of IB ring (2nd and 3rd panels)  , excluded from the IB structure (4th panel),  or is diffused evenly between the cytoplasm and IB structure (bottom panel).


\begin{figure}
    \begin{subfigure}{0.5\textwidth}
        \includegraphics[width=1\linewidth]{09. Chapter 4/Figs/02. Infection/01. IFIT1/01. bar_i1_a549.pdf} 
        \caption[]{Occurance of Phenotypes}
    \end{subfigure}
    \begin{subfigure}{0.5\textwidth}
        \includegraphics[width=1\linewidth]{09. Chapter 4/Figs/02. Infection/01. IFIT1/02. box_i1_a549.pdf}
        \caption[]{IB Area per Phenotype}
    \end{subfigure}
    \caption[i1 a549 hrsv plots]{i1 a549 hrsv plots}
    \label{fig:i1 a549 hrsv plots}
\end{figure}


\begin{figure}
    \centering
    \includegraphics[width=1\linewidth]{09. Chapter 4/Figs/02. Infection/01. IFIT1/03. a549 i1.pdf}
    \caption[i1 a549 hrsv]{i1 a549 hrsv}
    \label{fig:i1 a549 hrsv}
\end{figure}


%beas2b hrsv
Detecting magenta: endogenous human IFIT1 \newline
Detecting cyan: human IB \newline
Cell Line: BEAS2B \newline
Treatment: hRSV \newline

\begin{figure}
    \begin{subfigure}{0.5\textwidth}
        \caption{}
        \includegraphics[width=1\linewidth]{09. Chapter 4/Figs/02. Infection/01. IFIT1/04. bar_i1_beas2b.pdf} 
    \end{subfigure}
    \begin{subfigure}{0.5\textwidth}
        \caption{}
        \includegraphics[width=1\linewidth]{09. Chapter 4/Figs/02. Infection/01. IFIT1/05. box_i1_beas2b.pdf}
    \end{subfigure}
    \begin{subfigure}{1\textwidth}
        \centering
        \caption{}
        \includegraphics[width=1\linewidth]{09. Chapter 4/Figs/02. Infection/01. IFIT1/06. beas2b i1.pdf}
    \end{subfigure}
    \caption[i1 beas2b hrsv]{i1 beas2b hrsv}
    \label{fig:i1 beas2b hrsv}
\end{figure}

%bIFIT1 Localisation During h/bRSV Infection
%mdbk brsv
Detecting magenta: endogenous bovine IFIT1 \newline
Detecting cyan: bovine IB \newline
Cell Line: MDBK \newline
Treatment: bRSV dSH + bIFNa \newline

Nascent bovine IFIT1 in the context of bRSV infection has been observed to localise with the respect of IB in three distinct spaces. We observed it either concentrated inside the central point of the IB structure, while having reduced signal on the inner IB edge, compared to the cytoplasm (top and bottom panels), being excluded from the IB structure (3rd panel), or colocalising on the inner edge of the IB structure while having reduced signal in the middle of the structure compared to cytoplasm, or the edge staining (2nd panel).

\begin{figure}
    \begin{subfigure}{0.5\textwidth}
        \includegraphics[width=1\linewidth]{09. Chapter 4/Figs/02. Infection/01. IFIT1/07. bar_i1_mdbk.pdf} 
        \caption[]{Occurance of Phenotypes}
    \end{subfigure}
    \begin{subfigure}{0.5\textwidth}
        \includegraphics[width=1\linewidth]{09. Chapter 4/Figs/02. Infection/01. IFIT1/08. box_i1_mdbk.pdf}
        \caption[]{IB Area per Phenotype}
    \end{subfigure}
    \caption[i1 mdbk brsv plots]{i1 mdbk brsv plots}
    \label{fig:i1 mdbk brsv plots}

    \begin{subfigure}{1\textwidth}
        \centering
        \includegraphics[width=1\linewidth]{09. Chapter 4/Figs/02. Infection/01. IFIT1/09. mdbk i1.pdf}
    \end{subfigure}
    \caption[i1 mdbk brsv]{i1 mdbk brsv}
    \label{fig:i1 mdbk brsv}
\end{figure}


\subsubsection{Infection IFIT3}
%a549 hrsv
Detecting magenta: endogenous human IFIT3 \newline
Detecting cyan: human IB \newline
Cell Line: A549 \newline
Treatment: hRSV \newline

Nascent human IFIT3 seems to have mainly diffused phenotype (top and bottom panel) with occasional exclusion without any marked IFIT3 concentration adjacent to the IB structure (middle panel).

\begin{figure}
    \begin{subfigure}{0.5\textwidth}
        \includegraphics[width=1\linewidth]{09. Chapter 4/Figs/02. Infection/02. IFIT3/01. bar_i3_a549.pdf} 
        \caption[]{Occurance of Phenotypes}
    \end{subfigure}
    \begin{subfigure}{0.5\textwidth}
        \includegraphics[width=1\linewidth]{09. Chapter 4/Figs/02. Infection/02. IFIT3/02. box_i3_a549.pdf}
        \caption[]{IB Area per Phenotype}
    \end{subfigure}
    \caption[i3 a549 hrsv plots]{i3 a549 hrsv plots}
    \label{fig:i3 a549 hrsv plots}
\end{figure}

\begin{figure}
    \centering
    \includegraphics[width=1\linewidth]{09. Chapter 4/Figs/02. Infection/02. IFIT3/03. a549 i3.pdf}
    \caption[i3 a549 hrsv]{i3 a549 hrsv}
    \label{fig:i3 a549 hrsv}
\end{figure}

%beas2b hrsv
Detecting magenta: endogenous human IFIT3 \newline
Detecting cyan: human IB \newline
Cell Line: BEAS2B \newline
Treatment: hRSV \newline

\begin{figure}
    \begin{subfigure}{0.5\textwidth}
        \includegraphics[width=1\linewidth]{09. Chapter 4/Figs/02. Infection/02. IFIT3/04. bar_i3_beas2b.pdf} 
        \caption[]{Occurance of Phenotypes}
    \end{subfigure}
    \begin{subfigure}{0.5\textwidth}
        \includegraphics[width=1\linewidth]{09. Chapter 4/Figs/02. Infection/02. IFIT3/05. box_i3_beas2b.pdf}
        \caption[]{IB Area per Phenotype}
    \end{subfigure}
    \caption[i3 beas2b hrsv plots]{i3 beas2b hrsv plots}
    \label{fig:i3 beas2b hrsv plots}
\end{figure}

\begin{figure}
    \centering
    \includegraphics[width=1\linewidth]{09. Chapter 4/Figs/02. Infection/02. IFIT3/06. beas2b i3.pdf}
    \caption[i3 beas2b hrsv]{i3 beas2b hrsv}
    \label{fig:i3 beas2b hrsv}
\end{figure}

%mdbk brsv
Detecting magenta: endogenous bovine IFIT3 \newline
Detecting cyan: bovine IB \newline
Cell Line: MDBK \newline
Treatment: bRSV dSH + bIFNa \newline

In this experiment the nascent bovine IFIT3 was consistently concentrated inside IBs. In some IBs the IFIT3 signal showed signs of sub-concentrations within the inclusions (bottom panel; highlighted with arrows), resembling IBAGs (inclusion body associated granules).

Subsequent experiments did not recapitulate the IFIT3 inclusions. It however shows signal which is almost identical to what is observed with IFIT5 in MDBKs during bRSV infection i.e., decrease, but not complete abolishment of IFIT3 signal inside the IB structure (top panel), and IB boundary exclusion while maintaining similar levels of signal intensity between cytoplasmic and intra IB stain.

\begin{figure}
    \begin{subfigure}{0.5\textwidth}
        \includegraphics[width=1\linewidth]{09. Chapter 4/Figs/02. Infection/02. IFIT3/07. bar_i3_mdbk.pdf} 
        \caption[]{Occurance of Phenotypes}
    \end{subfigure}
    \begin{subfigure}{0.5\textwidth}
        \includegraphics[width=1\linewidth]{09. Chapter 4/Figs/02. Infection/02. IFIT3/08. box_i3_mdbk.pdf}
        \caption[]{IB Area per Phenotype}
    \end{subfigure}
    \caption[i3 mdbk brsv plots]{i3 mdbk brsv plots}
    \label{fig:i3 mdbk brsv plots}
\end{figure}

\begin{figure}
    \centering
    \includegraphics[width=1\linewidth]{09. Chapter 4/Figs/02. Infection/02. IFIT3/09. mdbk i3.pdf}
    \caption[i3 mdbk brsv]{i3 mdbk brsv}
    \label{fig:i3 mdbk brsv}
\end{figure}

\subsubsection{Infection IFIT5}
%a549 hrsv
Detecting magenta: endogenous human IFIT5 \newline
Detecting cyan: human IB \newline
Cell Line: A549 \newline
Treatment: hRSV \newline

hIFIT5 seems to be excluded from hRSV IBs. There is a hint of accumulation of IFIT5 on the outside of IB (bottom panel; no z stacks to confirm this). 

\begin{figure}
    \begin{subfigure}{0.5\textwidth}
        \includegraphics[width=1\linewidth]{09. Chapter 4/Figs/02. Infection/03. IFIT5/01. bar_i5_a549.pdf} 
        \caption[]{Occurance of Phenotypes}
    \end{subfigure}
    \begin{subfigure}{0.5\textwidth}
        \includegraphics[width=1\linewidth]{09. Chapter 4/Figs/02. Infection/03. IFIT5/02. box_i5_a549.pdf}
        \caption[]{IB Area per Phenotype}
    \end{subfigure}
    \caption[i5 a549 hrsv plots]{i5 a549 hrsv plots}
    \label{fig:i5 a549 hrsv plots}
\end{figure}

\begin{figure}
    \centering
    \includegraphics[width=1\linewidth]{09. Chapter 4/Figs/02. Infection/03. IFIT5/03. a549 i5.pdf}
    \caption[i5 a549 hrsv]{i5 a549 hrsv}
    \label{fig:i5 a549 hrsv}
\end{figure}

%beas2b hrsv
Detecting magenta: endogenous human IFIT5 \newline
Detecting cyan: human IB \newline
Cell Line: BEAS2B \newline
Treatment: hRSV \newline

\begin{figure}
    \begin{subfigure}{0.5\textwidth}
        \includegraphics[width=1\linewidth]{09. Chapter 4/Figs/02. Infection/03. IFIT5/04. bar_i5_beas2b.pdf} 
        \caption[]{Occurance of Phenotypes}
    \end{subfigure}
    \begin{subfigure}{0.5\textwidth}
        \includegraphics[width=1\linewidth]{09. Chapter 4/Figs/02. Infection/03. IFIT5/05. box_i5_beas2b.pdf}
        \caption[]{IB Area per Phenotype}
    \end{subfigure}
    \caption[i5 beas2b hrsv plots]{i5 beas2b hrsv plots}
    \label{fig:i5 beas2b hrsv plots}
\end{figure}

\begin{figure}
    \centering
    \includegraphics[width=1\linewidth]{09. Chapter 4/Figs/02. Infection/03. IFIT5/06. beas2b i5.pdf}
    \caption[i5 beas2b hrsv]{i5 beas2b hrsv}
    \label{fig:i5 beas2b hrsv}
\end{figure}

%mdbk brsv
Detecting magenta: endogenous bovine IFIT5 \newline
Detecting cyan: bovine IB \newline
Cell Line: MDBK \newline
Treatment: bRSV dSH + bIFNa \newline

The distribution of bIFIT5 is equal between cytoplasmic stain and inside of the IB (with a hint of concentrations/substructures in top and bottom panel; no z stack data available). All 3 panels show a depression of IFIT5 signal at the side of IB boundary (highlighted by arrows).

\begin{figure}
    \begin{subfigure}{0.5\textwidth}
        \includegraphics[width=1\linewidth]{09. Chapter 4/Figs/02. Infection/03. IFIT5/07. bar_i5_mdbk.pdf} 
        \caption[]{Occurance of Phenotypes}
    \end{subfigure}
    \begin{subfigure}{0.5\textwidth}
        \includegraphics[width=1\linewidth]{09. Chapter 4/Figs/02. Infection/03. IFIT5/08. box_i5_mdbk.pdf}
        \caption[]{IB Area per Phenotype}
    \end{subfigure}
    \caption[i5 mdbk brsv plots]{i5 mdbk brsv plots}
    \label{fig:i5 mdbk brsv plots}
\end{figure}

\begin{figure}
    \centering
    \includegraphics[width=1\linewidth]{09. Chapter 4/Figs/02. Infection/03. IFIT5/09. mdbk i5.pdf}
    \caption[i5 mdbk brsv]{i5 mdbk brsv}
    \label{fig:i5 mdbk brsv}
\end{figure}

\subsubsection{Summary} \label{Summary-infection}
In the context of infection, endogenous human IFIT1 concentrates within the human RSV IB structure; colocalises to the edge of the IB; is diffused through the structure and cytoplasm equally; or is excluded from the structure. This suggests that the interaction between human IFIT1 and hRSV IB is dynamic and depends on factors that we do not understand yet. In the case of endogenous bovine IFIT1 in the context of bRSV IBs, IFIT1 is either excluded from the structure; excluded from the IB inner edge but concentrated inside; or excluded from the centre of IB structure but concentrated on the inner edge of the structure. 

Nascent human IFIT3 during hRSV infection is either excluded from IB structure or is diffused through the structure. Occasionally it colocalises to the IB ring. Nascent bIFIT3 during bRSV infection either siphons inside IBs and shows sub-IB granules or is excluded from the IB boundary with slightly decreased signal inside of the IB.

In human cells during hRSV infection IFIT5 is mainly excluded from the IBs but seems to concentrate on their edge. Once we saw colocalization with the IB ring and a concentration of IFIT5 inside it. In bovine cells IFIT5 is always excluded from the IB boundary and the signal inside is either slightly decreased or equal compared to cytoplasmic IFIT5. 
