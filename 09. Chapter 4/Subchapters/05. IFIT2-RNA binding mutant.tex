\subsection{Confocal of RNA-binding mutant}
Using published data about hIFIT2 rna-binding mutant

Difficulty of using alpha-fold with IFIT2 due to the swap domain

Using SWISS-MODEL to predict bIFIT2 structure from published hIFIT2 structures

Alignment of both structures, assessment of electrostatic charges and establishment of residues to be mutated

Primer design and mutagenesis procedure based on published hIFIT2 RNA-binding mutant paper

\begin{figure}
    \centering
    \includegraphics[width=1\linewidth]{09. Chapter 4/Figs/05. IFIT2-RNA binding mutant/01. structure.png}
    \caption[ifit2 mutant structure]{ifit2 mutant structure}
    \label{ifit2 mutant structure}
\end{figure}

\subsubsection{i2f-24}
\myparagraph{bi2f24}
Cell Line: VERO
Treatment: bIFIT2-FLAG-RBM
Detecting magenta: exogenous bovine IFIT2 RBM
Detecting cyan: background

Exogenous bovine IFIT2 RNA-binding mutant (RBM) seems to have the same distribution and effect on the cell as human IFIT2-FLAG overexpression.

\begin{figure}
    \centering
    \includegraphics[width=1\linewidth]{09. Chapter 4/Figs/05. IFIT2-RNA binding mutant/02. bi2f24.png}
    \caption[bi2f24]{bi2f24}
    \label{bi2f24}
\end{figure}

\subsubsection{pIB}
\myparagraph{bi2f24 + hnhp}
Cell Line: VERO
Treatment: hNhP + bIFIT2-FLAG-RBM
Detecting magenta: exogenous bovine IFIT2 RBM
Detecting cyan: human pIB

In the second experiment we see consistent colocalization and/or inclusion of bovine IFIT2 RNA-binding mutant with human pseudo inclusion bodies.

\begin{figure}
    \centering
    \includegraphics[width=0.5\linewidth]{09. Chapter 4/Figs/05. IFIT2-RNA binding mutant/03. bi2f hnhp.png}
    \caption[bi2f24 + hnhp]{bi2f24 + hnhp}
    \label{bi2f24 + hnhp}
\end{figure}

\subsubsection{Summary}
We have described how bovine IFIT2 RNA-binding mutant was designed based on the published human IFIT2 RNA-binding mutant data (needs to be annotated more). Overexpression of bovine IFIT2 RNA-binding mutant yields cellular distribution and morphology similar to what was observed with overexpressing human IFIT2-FLAG, suggesting that the mutant proteins are not toxic to the cells. In the first experiment where we were looking at interaction between bovine IFIT2 RNA-binding mutant and human pseudo inclusion bodies we saw several phenotypes. We observed bovine IFIT2 RNA-binding mutant being excluded from small and big pIBs and pIB associated filamentous network, while fully or partially colocalising with other pIBs. In a subsequent experiment we observed only colocalization and inclusion formation. When assessing the interaction between bovine IFIT2 RNA-binding mutant and human pIBs formed using wild-type human RSV P and GFP-tagged human RSV N, we observed consistently in two experiments that bovine IFIT2 RNA-binding mutant colocalises to the pIB structures.